\section{More on Modular Arithmetic}

For a positive integer $m$ and integers $a$ and $b$, we say that $a\equiv b\pmod{m}$, read as ``$a$ is congruent to $b$ modulo $m$,'' if $a$ and $b$ have the same remainder when dividing by $m$. Equivalently, $a\equiv b\pmod{m}$ if and only if $a - b$ is divisible by $m$. From the latter, we can prove that ``congruence modulo $m$'' is an equivalence relation on the integers, meaning that
\begin{enumerate}[label=(\alph*)]
\item (reflexive) $a\equiv a\pmod{m}$ for all integers $a$,
\item (symmetric) if $a\equiv b\pmod{m}$, then $b\equiv a\pmod{m}$,
\item (transitive) if $a\equiv b\pmod{m}$ and $b\equiv c\pmod{m}$, then $a\equiv c\pmod{m}$.
\end{enumerate}
What makes congruence useful, beyond just keeping track of remainders, is that we can still do arithmetic. More specifically, if $a\equiv b\pmod{m}$, then for any integer $c$,
\begin{enumerate}[label=(\alph*)]\setcounter{enumi}{3}
\item $a + c\equiv b + c\pmod{m}$,
\item $a - c\equiv b - c\pmod{m}$,
\item $ac\equiv bc\pmod{m}$.
\end{enumerate}
Combining some of the above properties allows us to show more generally that if $a\equiv b\pmod{m}$ and $c\equiv d\pmod{m}$, then
\begin{enumerate}[label=(\alph*)]\setcounter{enumi}{6}
\item $a + c\equiv b + d\pmod{m}$,
\item $a - c\equiv b - d\pmod{m}$,
\item $ac\equiv bd\pmod{m}$.
\end{enumerate}

In what follows, we develop some further important results in modular arithmetic, both for contest math and for continued studies of number theory.

\textit{These notes are also available at \url{https://azhou5849.github.io/teaching/}.}


\subsection{Multiplicative inverses}

In the rational and real number systems, division is formally defined as multiplication by the reciprocal. In modular arithmetic, we are limited to integers, but we can look for integers which fulfill the same role.

\begin{definition}[Multiplicative inverse]
Let $m$ be a positive integer and $a$ be an integer. A \emph{(multiplicative) inverse of $a$ modulo $m$} is an integer $b$ with the property that $ab\equiv 1\pmod{m}$.
\end{definition}

Inverses are unique modulo $m$, meaning that if $b$ and $c$ are both inverses of $a$ modulo $m$, then $b\equiv c\pmod{m}$. To prove this, we compute
\begin{equation*}
b\equiv b\cdot 1\equiv b\cdot (ac)\equiv (ba)\cdot c\equiv 1\cdot c\equiv c\pmod{m}.
\end{equation*}
This means that when an inverse of $a$ modulo $m$ exists, we can use the notation $a^{-1}$ for any such inverse (while we work modulo $m$).

The matter of existence is less straightforward than with reciprocals of rational or real numbers, however. For example, suppose $x$ is an inverse to 2 modulo 6. This means that $2x\equiv 1\pmod{6}$, so there is an integer $k$ such that $2x = 6k + 1$. However, this is impossible: the left hand side is divisible by 2 while the right hand side is not.

The key issue here is that 2 and 6 share a common factor greater than 1. More generally, if $\gcd(a,m) = d > 1$, then $ax\equiv 1\pmod{m}$ has no solutions, as in the corresponding integer equation $ax = km + 1$, both $ax$ and $km$ are divisible by $d$ while 1 is not.

On the other hand, if $\gcd(a,m) = 1$, then by B\'{e}zout's lemma (Lemma~\ref{lem:bezout}), there exist integers $x$ and $y$ such that $ax + my = 1$. This means that $ax\equiv 1\pmod{m}$, so the integer $x$ that comes from B\'{e}zout's lemma is an inverse of $a$ modulo $m$.

Putting this together, we have
\begin{theorem}\label{thm:existence-of-inverses}
Let $m$ be a positive integer and $a$ be an integer. An inverse of $a$ modulo $m$ exists if and only if $\gcd(a,m) = 1$.
\end{theorem}



\subsection{Complete residue sets}

A \emph{complete residue set modulo $m$} is a set of integers $S$ with the property that for every integer $a$, there is exactly one $x\in S$ for which $x\equiv a\pmod{m}$. We often use the set of remainders, $\{0, 1, 2, \ldots, m - 1\}$, as a complete residue set, but there are other ones we can use at times. Numerically, another convenient choice is to take values centered at 0, e.g. $\{-3, -2, -1, 0, 1, 2, 3\}$ when working modulo 7.

Closely related is the notion of a \emph{complete multiplicative residue set modulo $m$}, which is a set of integers $S$ with the properties (i) every $x\in S$ satisfies $\gcd(x,m) = 1$, and (ii) for every integer $a$ satisfying $\gcd(a,m) = 1$, there is exactly one $x\in S$ for which $x\equiv a\pmod{m}$.

\begin{enumerate}
\item How many elements are there in a complete residue set modulo $m$?
\item Let $S$ be a complete residue set modulo $24$. Is it possible for the sum of the elements of $S$ to be $2024$?
\end{enumerate}


\subsection{Division in modular arithmetic}

In class, we saw an example which shows that $ab\equiv ac\pmod{m}$ does not necessarily imply that $b\equiv c\pmod{m}$ (even when we require $a\not\equiv 0\pmod{m}$). To derive a variant which does work, we generalize Proposition~\ref{prop:prime} to cover non-prime divisors.

\begin{lemma}\label{lem:divisibility-product}
If $m$ divides $ab$, then $m' = \frac{m}{\gcd(a,m)}$ divides $b$.
\end{lemma}
\begin{proof}
Let $d = \gcd(a,m)$, so that $m = dm'$. By B\'{e}zout's lemma, there exist integers $x$ and $y$ such that $xa + ym = d$. Multiplying through by $b/m$,
\begin{equation*}
\frac{bd}{m} = \frac{xab}{m} + yb = x\cdot\frac{ab}{m} + yb
\end{equation*}
is an integer. Therefore, $bd/m = b/m'$ is an integer and $b$ is divisible by $m'$.
\end{proof}

\begin{proposition}[Division in modular arithmetic]\label{prop:mod-arith-divide}
If $ab\equiv ac\pmod{m}$, then
\begin{equation*}
b\equiv c\pmod{m'},\qquad m' = \frac{m}{\gcd(a,m)}.
\end{equation*}
\end{proposition}
\begin{proof}
If $ab\equiv ac\pmod{m}$, then $m$ divides $ab - ac = a(b - c)$. By Lemma~\ref{lem:divisibility-product}, $m'$ divides $b - c$, so $b\equiv c\pmod{m'}$ as required.
\end{proof}

As an example, to solve $9x\equiv 6\pmod{15}$, we can divide both sides by $3$, making sure to divide the modulus by $\gcd(3,15) = 3$ as well, to get $3x\equiv 2\pmod{5}$. Then, we can repeatedly add 5 to the right hand side until we get something divisible by 3. In this case, we successfully find $3x\equiv 12\pmod{5}$. Dividing by 3 again, this time dividing the modulus by $\gcd(3,5) = 1$, we get $x\equiv 4\pmod{5}$. Alternatively, we could have subtracted $5$ once to get $3x\equiv -3\pmod{5}$, so then $x\equiv -1\pmod{5}$. Since $4\equiv -1\pmod{5}$, we found the same solution.

\subsubsection*{Exercises}

\begin{enumerate}
\item What is the second smallest positive integer satisfying $12n\equiv 30\pmod{126}$?
\item The last two digits of $15n$ are $25$. If $100\leq n < 200$, what are the possible values of $n$?
\end{enumerate}


\subsection{Multiplicative inverses}

A particularly important case of Proposition~\ref{prop:mod-arith-divide} is that when $\gcd(a,m) = 1$, we can safely cancel $a$ from $ab\equiv ac\pmod{m}$ to get $b\equiv c\pmod{m}$.

\begin{theorem}\label{thm:complete-residue-set}
If $\gcd(a,m) = 1$, then the multiples
\begin{equation*}
0,\quad a,\quad 2a,\quad 3a,\quad \ldots,\quad (m - 1)a
\end{equation*}
form a complete residue set modulo $m$, meaning that for every integer $N$, we can find some multiple $ka$ such that $ka\equiv N\pmod{m}$, and the choice of $k\in\{0, 1, \ldots, m - 1\}$ is unique.
\end{theorem}
\begin{proof}
The above note shows that these $m$ multiples of $a$ lie in distinct residue classes. As there are only $m$ residue classes in total, all of the residue classes are represented exactly once.
\end{proof}

\begin{corollary}\label{cor:invertible-mod-m}
If $\gcd(a,m) = 1$, then there is an integer $b$ for which $ab\equiv 1\pmod{m}$, and this $b$ is unique modulo $m$. Conversely, if $ab\equiv 1\pmod{m}$, then $\gcd(a,m) = 1$.
\end{corollary}
\begin{proof}
For the forward direction, let $N = 1$ in Theorem~\ref{thm:complete-residue-set}. For the converse, let $d = \gcd(a,m)$. If $ab\equiv 1\pmod{m}$, then $ab - 1 = km$ for some integer $k$. Both $ab$ and $km$ are divisible by $d$, so $d$ divides $1$ and hence $d = 1$.
\end{proof}

Since the integer $b$ above is unique modulo $m$, we denote any such value by $a^{-1}$ and call it a \emph{multiplicative inverse of $a$ modulo $10$}. For instance, when working modulo $10$, we have $3^{-1}\equiv 7$ because $3\cdot 7\equiv 1\pmod{10}$. Multiplicative inverses have the same role for modular arithmetic that reciprocals do for real numbers: if we want to divide by a number in modular arithmetic, we can multiply by a multiplicative inverse (if it exists).

\subsubsection*{Exercises}

\begin{enumerate}
\item Given that 17 is a multiplicative inverse of 13 modulo 22, solve $13x\equiv 18\pmod{22}$.
\item Compute multiplicative inverses of 1, 2, 3, 4, 5, and 6 modulo 7.
\end{enumerate}


\subsection{Systems of linear congruences}

In this section, we consider systems of congruences of the form
\begin{align*}
x &\equiv a\pmod{m}, \\
x &\equiv b\pmod{n}.
\end{align*}
For example, suppose we have the system
\begin{align*}
x &\equiv 4\pmod{8}, \\
x &\equiv 2\pmod{21}.
\end{align*}
To solve this, we can rewrite one of the congruences as an equation using an extra integer-valued variable. The second congruence states that $x = 21k + 2$ for some integer $k$. Substituting into the first congruence,
\begin{equation*}
21k + 2\equiv 4\pmod{8}\implies 5k\equiv 2\pmod{8}.
\end{equation*}
Either by finding a multiplicative inverse of $5$ modulo $8$, which is doable since $\gcd(5,8) = 1$, or by adding or subtracting a multiple of $8$ to the right hand side to get something divisible by $5$, we find that $k\equiv 2\pmod{8}$. In turn writing this as $k = 8\ell + 2$ for some integer $\ell$,
\begin{equation*}
x = 21(8\ell + 2) + 2 = 168\ell + 44.
\end{equation*}
This shows that $\boxed{x\equiv 44\pmod{168}}$. Conversely, $x\equiv 44\pmod{168}$ satisfies the two original congruences, as can be seen by simplify $168\ell + 44$ modulo $8$ and $21$, so we have found the exact solution set. The process generalizes.

\begin{theorem}[Chinese remainder theorem]\label{thm:crt}
If $\gcd(m,n) = 1$, then for any integers $a$ and $b$, the solution set to the system of congruences
\begin{align*}
x &\equiv a\pmod{m}, \\
x &\equiv b\pmod{n}
\end{align*}
is a residue class modulo $mn$.
\end{theorem}

\subsubsection*{Exercises}

\begin{enumerate}
\item (Week 4 slides) There are several cookies in a jar. Sharing the cookies evenly among $3$ children leaves $2$ cookies left. Sharing the cookies among $7$ children leaves $3$ cookies left. Find all the possible amounts of cookies in the jar.
\item (Week 4 extensions) The eighth graders are taking buses to visit Sea World today. Each large bus can take $56$ students and each small bus can take $36$ students. If we try to let all students take a large bus, then all will be full except for one bus with only $9$ students. If we try to let all students take a small bus, then all will be full except for one bus with $21$ students. What are the possible amounts of eighth graders?
\end{enumerate}


\subsection{Euler's Totient Theorem}

By Corollary~\ref{cor:invertible-mod-m}, the integers relatively prime to $m$ are precisely those that have multiplicative inverses modulo $m$. As we can compute $(xy)^{-1}\equiv x^{-1}y^{-1}\pmod{m}$ whenever $x$ and $y$ are invertible modulo $m$, this tells us (with an admittedly roundabout proof) that the product of any two integers relatively prime to $m$ is also relatively prime to $m$.

Now consider the complete multiplicative residue set modulo $m$
\begin{equation*}
\Phi_m = \{a\mid 1\leq a\leq m\text{ and }\gcd(a,m) = 1\}.
\end{equation*}
(Here $\Phi$ is the capital Greek letter Phi. The notation is non-standard.) The number of elements of $\Phi_m$ is the \emph{Euler totient function} evaluated at $m$, and this function is denoted $\phi$ or $\varphi$ (the lowercase Greek letter phi).

Analogously to Theorem~\ref{thm:complete-residue-set}, we have 
\begin{theorem}\label{thm:complete-multiplicative-residue-set}
If $\gcd(a,m) = 1$, then
\begin{equation*}
a\cdot\Phi_m = \{ab\mid b\in\Phi_m\}
\end{equation*}
is also a complete multiplicative residue set modulo $m$.
\end{theorem}
\section{More on Modular Arithmetic}

\textit{These notes are also available at \url{https://azhou5849.github.io/teaching/}.}

For a positive integer $m$ and integers $a$ and $b$, we say that $a\equiv b\pmod{m}$, read as ``$a$ is congruent to $b$ modulo $m$,'' if $a - b$ is divisible by $m$.

\subsection{Addition and multiplication in modular arithmetic}

We saw in class that we can add and multiply congruences without any issue. Here we write out a proof of this claim.

\begin{proposition}
Suppose $a\equiv b\pmod{m}$ and $c\equiv d\pmod{m}$.
\begin{enumerate}
\item $a + c\equiv b + d\pmod{m}$
\item $ac\equiv bd\pmod{m}$
\end{enumerate}
\end{proposition}
\begin{proof}
Since $a\equiv b\pmod{m}$, we can write $a - b = xm$ and $c - d = ym$ for some integers $x$ and $y$. Then $a = b + xm$ and $c = d + ym$.
\begin{enumerate}
\item We compute
\begin{equation*}
(a + c) - (b + d) = [(b + xm) + (d + ym)] - (b + d) = (x + y)m.
\end{equation*}
This is a multiple of $m$, so $a + c\equiv b + d\pmod{m}$.
\item We compute 
\begin{equation*}
(ac) - (bd) = (b + xm)(d + ym) - bd = (bd + bym + dxm + xym^2) - bd = (by + dx + xym)m.
\end{equation*}
This is a multiple of $m$, so $ac\equiv bd\pmod{m}$.
\end{enumerate}
\end{proof}

\subsubsection*{Exercises}

\begin{enumerate}
\item Show that we can subtract congruences: $a - c\equiv b - d\pmod{m}$.
\end{enumerate}


\subsection{Complete residue sets}

A \emph{complete residue set modulo $m$} is a set of integers $S$ with the property that for every integer $a$, there is exactly one $x\in S$ for which $x\equiv a\pmod{m}$. We often use the set of remainders, $\{0, 1, 2, \ldots, m - 1\}$, as a complete residue set, but there are other ones we can use at times. Numerically, another convenient choice is to take values centered at 0, e.g. $\{-3, -2, -1, 0, 1, 2, 3\}$ when working modulo 7.

Closely related is the notion of a \emph{complete multiplicative residue set modulo $m$}, which is a set of integers $S$ with the properties (i) every $x\in S$ satisfies $\gcd(x,m) = 1$, and (ii) for every integer $a$ satisfying $\gcd(a,m) = 1$, there is exactly one $x\in S$ for which $x\equiv a\pmod{m}$.

\begin{enumerate}
\item How many elements are there in a complete residue set modulo $m$?
\item Let $S$ be a complete residue set modulo $24$. Is it possible for the sum of the elements of $S$ to be $2024$?
\end{enumerate}


\subsection{Division in modular arithmetic}

In class, we saw an example which shows that $ab\equiv ac\pmod{m}$ does not necessarily imply that $b\equiv c\pmod{m}$ (even when we require $a\not\equiv 0\pmod{m}$). To derive a variant which does work, we generalize Proposition~\ref{prop:prime} to cover non-prime divisors.

\begin{lemma}\label{lem:divisibility-product}
If $m$ divides $ab$, then $m' = \frac{m}{\gcd(a,m)}$ divides $b$.
\end{lemma}
\begin{proof}
Let $d = \gcd(a,m)$, so that $m = dm'$. By B\'{e}zout's lemma, there exist integers $x$ and $y$ such that $xa + ym = d$. Multiplying through by $b/m$,
\begin{equation*}
\frac{bd}{m} = \frac{xab}{m} + yb = x\cdot\frac{ab}{m} + yb
\end{equation*}
is an integer. Therefore, $bd/m = b/m'$ is an integer and $b$ is divisible by $m'$.
\end{proof}

\begin{proposition}[Division in modular arithmetic]\label{prop:mod-arith-divide}
If $ab\equiv ac\pmod{m}$, then
\begin{equation*}
b\equiv c\pmod{m'},\qquad m' = \frac{m}{\gcd(a,m)}.
\end{equation*}
\end{proposition}
\begin{proof}
If $ab\equiv ac\pmod{m}$, then $m$ divides $ab - ac = a(b - c)$. By Lemma~\ref{lem:divisibility-product}, $m'$ divides $b - c$, so $b\equiv c\pmod{m'}$ as required.
\end{proof}

As an example, to solve $9x\equiv 6\pmod{15}$, we can divide both sides by $3$, making sure to divide the modulus by $\gcd(3,15) = 3$ as well, to get $3x\equiv 2\pmod{5}$. Then, we can repeatedly add 5 to the right hand side until we get something divisible by 3. In this case, we successfully find $3x\equiv 12\pmod{5}$. Dividing by 3 again, this time dividing the modulus by $\gcd(3,5) = 1$, we get $x\equiv 4\pmod{5}$. Alternatively, we could have subtracted $5$ once to get $3x\equiv -3\pmod{5}$, so then $x\equiv -1\pmod{5}$. Since $4\equiv -1\pmod{5}$, we found the same solution.

\subsubsection*{Exercises}

\begin{enumerate}
\item What is the second smallest positive integer satisfying $12n\equiv 30\pmod{126}$?
\item The last two digits of $15n$ are $25$. If $100\leq n < 200$, what are the possible values of $n$?
\end{enumerate}


\subsection{Multiplicative inverses}

A particularly important case of Proposition~\ref{prop:mod-arith-divide} is that when $\gcd(a,m) = 1$, we can safely cancel $a$ from $ab\equiv ac\pmod{m}$ to get $b\equiv c\pmod{m}$.

\begin{theorem}\label{thm:complete-residue-set}
If $\gcd(a,m) = 1$, then the multiples
\begin{equation*}
0,\quad a,\quad 2a,\quad 3a,\quad \ldots,\quad (m - 1)a
\end{equation*}
form a complete residue set modulo $m$, meaning that for every integer $N$, we can find some multiple $ka$ such that $ka\equiv N\pmod{m}$, and the choice of $k\in\{0, 1, \ldots, m - 1\}$ is unique.
\end{theorem}
\begin{proof}
The above note shows that these $m$ multiples of $a$ lie in distinct residue classes. As there are only $m$ residue classes in total, all of the residue classes are represented exactly once.
\end{proof}

\begin{corollary}
If $\gcd(a,m) = 1$, then there is an integer $b$ for which $ab\equiv 1\pmod{m}$, and this $b$ is unique modulo $m$. Conversely, if $ab\equiv 1\pmod{m}$, then $\gcd(a,m) = 1$.
\end{corollary}
\begin{proof}
For the forward direction, let $N = 1$ in Theorem~\ref{thm:complete-residue-set}. For the converse, let $d = \gcd(a,m)$. If $ab\equiv 1\pmod{m}$, then $ab - 1 = km$ for some integer $k$. Both $ab$ and $km$ are divisible by $d$, so $d$ divides $1$ and hence $d = 1$.
\end{proof}

Since the integer $b$ above is unique modulo $m$, we denote any such value by $a^{-1}$ and call it the \emph{multiplicative inverse of $a$ modulo $10$}. For instance, when working modulo $10$, we have $3^{-1}\equiv 7$ because $3\cdot 7\equiv 1\pmod{10}$. Multiplicative inverses have the same role for modular arithmetic that reciprocals do for real numbers: if we want to divide by a number in modular arithmetic, we can multiply by the multiplicative inverse (if it exists).

\subsubsection*{Exercises}

\begin{enumerate}
\item Given that the multiplicative inverse of 13 modulo 22 is 17, solve the congruence $13x\equiv 18\pmod{22}$ for $x$.
\item Compute the multiplicative inverse of 4 modulo 9, 27, and 81.
\item Compute the multiplicative inverses of all non-zero residues modulo 7.
\end{enumerate}


\subsection{Multiplicative groups}

For a positive integer $m$, let $\Phi_m$ denote the set of residue classes which are relatively prime to $m$. (The symbol $\Phi$ is the capital Greek letter ``Phi.'') Equivalently, $\Phi_m$ consists of all residue classes which are invertible modulo $m$. This means that the product of any two elements of $\Phi_m$ is an element of $\Phi_m$, as we can directly check that if $x$ and $y$ have multiplicative inverses, then $x^{-1}y^{-1}$ is the multiplicative inverse of $xy$.

\begin{definition}
The set $\Phi_m$ with multiplication modulo $m$ is the \emph{multiplicative group modulo $m$}.
\end{definition}


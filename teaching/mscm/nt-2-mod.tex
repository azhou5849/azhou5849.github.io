\section{More on Modular Arithmetic}

\textit{These notes are also available at \url{https://azhou5849.github.io/teaching/}.}

For a positive integer $m$ and integers $a$ and $b$, we say that $a\equiv b\pmod{m}$, read as ``$a$ is congruent to $b$ modulo $m$,'' if $a - b$ is divisible by $m$.

\subsection{Proving that addition, subtraction, and multiplication work}

\begin{proposition}
Suppose $a\equiv b\pmod{m}$ and $c\equiv d\pmod{m}$.
\begin{enumerate}
\item $a + c\equiv b + d\pmod{m}$
\item $ac\equiv bd\pmod{m}$
\item $a - c\equiv b - d\pmod{m}$
\end{enumerate}
\end{proposition}
\begin{proof}
Since $a\equiv b\pmod{m}$, we can write $a - b = xm$ and $c - d = ym$ for some integers $x$ and $y$. Then $a = b + xm$ and $c = d + ym$.
\begin{enumerate}
\item We compute
\begin{equation*}
(a + c) - (b + d) = [(b + xm) + (d + ym)] - (b + d) = (x + y)m.
\end{equation*}
This is a multiple of $m$, so $a + c\equiv b + d\pmod{m}$.
\item We compute 
\begin{equation*}
(ac) - (bd) = (b + xm)(d + ym) - bd = (bd + bym + dxm + xym^2) - bd = (by + dx + xym)m.
\end{equation*}
This is a multiple of $m$, so $ac\equiv bd\pmod{m}$.
\item We multiply both sides of $c\equiv d\pmod{m}$ by $-1$ to get $-c\equiv -d\pmod{m}$. Then, adding to $a\equiv b\pmod{m}$ gives us $a - c\equiv b - d\pmod{m}$.
\end{enumerate}
\end{proof}


\subsection{Division in modular arithmetic}

In class, we saw an example which shows that $ab\equiv ac\pmod{m}$ does not necessarily imply that $b\equiv c\pmod{m}$ (even when we require $a\not\equiv 0\pmod{m}$). For example, $2\cdot 4\equiv 2\cdot 9\pmod{10}$, but $4\not\equiv 9\pmod{10}$. To derive a variant which does work, we generalize Proposition~\ref{prop:prime} to cover non-prime divisors.

\begin{lemma}\label{lem:divisibility-product}
If $m$ divides $ab$, then $m' = \frac{m}{\gcd(a,m)}$ divides $b$.
\end{lemma}
\begin{proof}
Let $d = \gcd(a,m)$, so that $m = dm'$. By B\'{e}zout's lemma, there exist integers $x$ and $y$ such that $xa + ym = d$. Multiplying through by $b/m$,
\begin{equation*}
\frac{bd}{m} = \frac{xab}{m} + yb = x\cdot\frac{ab}{m} + yb
\end{equation*}
is an integer. Therefore, $bd/m = b/m'$ is an integer and $b$ is divisible by $m'$.
\end{proof}

\begin{proposition}[Division in modular arithmetic]\label{prop:mod-arith-divide}
If $ab\equiv ac\pmod{m}$, then
\begin{equation*}
b\equiv c\pmod{m'},\qquad m' = \frac{m}{\gcd(a,m)}.
\end{equation*}
\end{proposition}
\begin{proof}
If $ab\equiv ac\pmod{m}$, then $m$ divides $ab - ac = a(b - c)$. By Lemma~\ref{lem:divisibility-product}, $m'$ divides $b - c$, so $b\equiv c\pmod{m'}$ as required.
\end{proof}

As an example, to solve $4x\equiv 8\pmod{10}$, we divide both sides by $4$, dividing the modulus by $\gcd(4,10) = 2$, to get $\boxed{x\equiv 2\pmod{5}}$.

\subsubsection*{Exercises}

\begin{enumerate}
\item What is the second smallest positive integer satisfying $12n\equiv 30\pmod{126}$
\item The last two digits of $15n$ are $25$. If $100\leq n < 200$, what are the possible values of $n$?
\end{enumerate}


\subsection{Multiplicative inverses}
\section{Proving the Fundamental Theorem of Arithmetic}

The \textbf{fundamental theorem of arithmetic} (FTA) states that every positive integer can be written as a product of primes, and that moreover, there is only one way to do this. (Here we consider a rearrangement of the factors to be the same product.) For example, 2024 and 2025 can be written as
\begin{equation*}
2024 = 2\cdot 2\cdot 2\cdot 11\cdot 23\quad\text{and}\quad 2025 = 3\cdot 3\cdot 3\cdot 3\cdot 5\cdot 5.
\end{equation*}
For a given positive integer, the corresponding product of primes is its \emph{prime factorization}.

In this chapter, we prove the FTA.


\subsection{Every positive integer is a product of primes}

The fact that there is \underline{a} prime factorization requires no new number theory, although we will use \href{https://en.wikipedia.org/wiki/Mathematical_induction}{mathematical induction}.

\begin{theorem}
Every positive integer is prime or can be written as a product of primes.
\end{theorem}
\begin{proof}
If $n = 1$, it is the ``empty product'' of primes, and if $n$ is prime, it is the product of one prime (itself). With these special cases established, by induction, it suffices to show for any composite $n$ that if all positive integers $k < n$ can be written as a product of primes, then so can $n$. Since $n$ is composite, we can write $n = ab$ with $1 < a,b < n$. Then, $a$ and $b$ can be written as products of primes by the inductive hypothesis. Concatenating these products gives us $n$ as a product of primes, as desired.
\end{proof}

What remains is to prove that there is only one prime factorization, and the rest of the chapter is dedicated to this goal.


\subsection{The euclidean algorithm}

Given integers $a$ and $b$, any common divisor $d$ also divides $a + b$ and $a - b$. From this, it follows that we always have $\gcd(a,b) = \gcd(a - b, b)$. For example,
\begin{align*}
\gcd(70,91) &= \gcd(70,21) \\
&= \gcd(49,21) = \gcd(28,21) = \gcd(7,21) \\
&= \gcd(7,14) = \gcd(7,7) = \boxed{7}.
\end{align*}
This method for calculating GCD is the \emph{euclidean algorithm}.

Instead of repeatedly subtracting the same number over and over again, as in the second line above, we can jump to the end by dividing and extracting the remainder. The same example would then look like
\begin{align*}
\gcd(70,91) &= \gcd(70,21) & 91 &= 1\cdot 70 + 21 \\
&= \gcd(7,21) & 70 &= 3\cdot 21 + 7 \\
&= \gcd(7,0) & 21 &= 3\cdot 7 + 0 \\
&= \boxed{7}.
\end{align*}

\begin{remark}
Most GCD calculations in middle school competitions involve numbers small enough that we could use factorization instead. The advantage of the euclidean algorithm emerges when we deal with large numbers that are difficult to factor efficiently, so this is usually how GCD is implemented in computers. For instance,
\begin{align*}
\gcd(582133, 268381) &= \gcd(45371, 268381) & 582133 &= 2\cdot 268381 + 45371 \\
&= \gcd(45371, 41526) & 268381 &= 5\cdot 45371 + 41526 \\
&= \gcd(3845, 41526) & 45371 &= 1\cdot 41526 + 3845 \\
&= \gcd(3845, 3076) & 41526 &= 10\cdot 3845 + 3076 \\
&= \gcd(769, 3076) & 3845 &= 1\cdot 3076 + 769 \\
&= \gcd(769, 0) & 3076 &= 4\cdot 769 + 0 \\
&= \boxed{769}.
\end{align*}
For reference, the prime factorizations are $582133 = 757\cdot 769$ and $268381 = 349\cdot 769$.
\end{remark}


\subsection{B\'{e}zout's lemma}

The euclidean algorithm lets us write the GCD in terms of the original integers. More specifically,

\begin{lemma}[B\'{e}zout]\label{lem:bezout}
Let $a$ and $b$ be positive integers. There exist integers $m$ and $n$ such that
\begin{equation*}
ma + nb = \gcd(a,b).
\end{equation*}
\end{lemma}
Using the example of $70$ and $91$, we have
\begin{equation*}
7 = 70 - 3\cdot 21,
\end{equation*}
from the last step of the euclidean algorithm that involves a non-zero remainder. The previous step gives us $21 = 91 - 1\cdot 70$, and we substitute this in to get
\begin{equation*}
7 = 70 - 3\cdot (91 - 1\cdot 70) = 4\cdot 70 - 3\cdot 91,
\end{equation*}
i.e. $4\cdot 70 + (-3)\cdot 91 = \gcd(70,91)$. A more involved example is that
\begin{align*}
769 &= 1\cdot 3845 - 1\cdot 3076 = 1\cdot 3845 - (41526 - 10\cdot 3845) \\
&= 11\cdot 3845 - 1\cdot 41526 = 11\cdot (45371 - 1\cdot 41526) - 1\cdot 41526 \\
&= 11\cdot 45371 - 12\cdot 41526 = 11\cdot 45371 - 12\cdot (268381 - 5\cdot 45371) \\
&= 71\cdot 45371 - 12\cdot 268381 = 71\cdot (582133 - 2\cdot 268381) - 12\cdot 268381 \\
&= 71\cdot 582133 - 154\cdot 268381.
\end{align*}


\subsection{Divisibility by a prime}

\begin{proposition}\label{prop:prime}
Let $p$ be a prime number and let $a$ and $b$ be integers. If $p$ divides $ab$, then $p$ divides $a$ or $p$ divides $b$.
\end{proposition}
\begin{proof}
It suffices to show that if $p$ divides $ab$ but $p$ does not divide $a$, then $p$ divides $b$. Since the only divisors of $p$ are $1$ and $p$, we must have $\gcd(a,p) = 1$ or $\gcd(a,p) = p$. The latter implies $p$ divides $a$, so it must be the case that $\gcd(a,p) = 1$. Now, B\'{e}zout's lemma implies that we can find integers $m$ and $n$ such that
\begin{equation*}
ma + np = 1.
\end{equation*}
Multiplying through by $b$,
\begin{equation*}
m(ab) + (nb)p = b.
\end{equation*}
Both terms of the left hand side are divisible by $p$, so the right hand side $b$ is divisible by $p$.
\end{proof}

It follows that if a prime $p$ divides a product of integers of any length (not necessarily just two factors), then $p$ divides at least one of the integers in the product.


\subsection{Prime factorization is unique}

We can now prove that no positive integer can have two genuinely different prime factorizations. Suppose that the positive integer $n$ has prime factorizations
\begin{equation*}
n = p_1p_2p_3\cdots p_r = q_1q_2q_3\cdots q_s,
\end{equation*}
where $p_1\leq p_2\leq p_3\leq\cdots\leq p_r$ and $q_1\leq q_2\leq q_3\leq\cdots\leq q_s$ are primes. Suppose $p_1\leq q_1$. By the previous section's result, the prime number $p_1$ divides the product $q_1q_2q_3\cdots q_s$, so $p_1$ must divide one of the factors $q_1, q_2, q_3, \ldots, q_s$. Since those factors are all prime, the only way this can happen is for $p_1$ to be equal to one of those factors. Since $p_1\leq q_1$, which is the smallest of the factors, we must have $p_1 = q_1$. If we suppose instead that $q_1\leq p_1$, we reach the same conclusion with the same argument. Thus we have $p_1 = q_1$ in any case, and we can cancel this common factor to get
\begin{equation*}
p_2p_3\cdots p_r = q_2q_3\cdots q_s.
\end{equation*}
We then continue this process to show $p_2 = q_2$, then $p_3 = q_3$, and so on. If one side ever runs out of factors before the other, then we end up with one side of the equation being 1 and the other side being a product of one or more primes, which is impossible. Thus the two factorizations have the same length and use the same primes.
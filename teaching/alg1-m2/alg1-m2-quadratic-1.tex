\section{Quadratics (I)}

\subsection{Review problems}

\begin{enumerate}
\item Solve each of the following equations for all real solutions.
\begin{enumerate}
\item $(x + 1)(x + 3) = 0$
\item $4(x + 3)(x^2 - 3) = 0$
\item $(x - 1)(\sqrt{x} - 3)(x - 4) = 0$
\end{enumerate}
\item Factor each of the following expressions as much as possible. All coefficients and constants in the factorisations should be integers.
\begin{enumerate}
\item $x^2 + 5x$
\item $6x^2 + 2x$
\item $x^2 - 16$
\item $x^2 + 3x + 2$
\item $x^2 + 8x + 16$
\item $4x^2 - 18x + 8$
\end{enumerate}
\item For each of the following quadratics, find the sum and product of all real roots.
\begin{enumerate}
\item $x^2 - 25$
\item $x^2 + x$
\item $x^2 + 3x + 2$
\item $4x^2 + 14x - 8$
\item $16x^2 + 8x + 1$
\end{enumerate} 
\item For each of the following descriptions, find the quadratic $ax^2 + bx + c$ fitting the description. The coefficients $a,b,c$ should be integers with $a > 0$ and $\gcd(a,b,c) = 1$.
\begin{enumerate}
\item The roots are $-1$ and $5$.
\item The roots are $2$ and $-2$.
\item The roots are $-2/3$ and $0$.
\item The only root is $-1/5$.
\item The sum of the roots is $3$ and the product of the roots is $2$.
\item The sum of the roots is $-9/2$ and the product of the roots is $-5/2$.
\end{enumerate}
\item Two real numbers have a sum of 18.
\begin{enumerate}
\item What is their largest possible product?
\item If their product is $-63$, what are the two numbers?
\end{enumerate}
\item The quadratic $x^2 - x - 1$ has roots $r$ and $s$. Compute $r^2 + s^2$.
\end{enumerate}


\subsection{Challenge problems}

\begin{enumerate}[resume]
\item Suppose $a + b = 7$ and $ab = 3$.
\begin{enumerate}
\item Find a quadratic whose roots are $a$ and $b$.
\item Find a quadratic whose roots are $a^2$ and $b^2$.
\item Find a quadratic whose roots are $\frac{1}{a}$ and $\frac{1}{b}$.
\end{enumerate}
\item \begin{enumerate}
\item Factor $x^4 + x^2 + 1$.
\item (Sophie-Germain) Factor $x^4 + 4$.
\end{enumerate}
\item Suppose $x$ is a real number satisfying $x + \frac{1}{x} = 3$.
\begin{enumerate}
\item Compute $x^2 + \frac{1}{x^2}$.
\item Compute $x^3 + \frac{1}{x^3}$.
\end{enumerate}
\item Find all real solutions of
\begin{equation*}
2x^4 - x^3 - 6x^2 - x + 2 = 0.
\end{equation*}
\end{enumerate}


\newpage
\subsection{Answers}

\begin{enumerate}
\item \begin{enumerate}
\item $-3$, $-1$
\item $-3$, $-\sqrt{3}$, $\sqrt{3}$
\item $1$, $4$, $9$
\end{enumerate}
\item \begin{enumerate}
\item $x(x + 5)$
\item $2x(3x + 1)$
\item $(x - 4)(x + 4)$
\item $(x + 1)(x + 2)$
\item $(x + 4)^2$
\item $2(2x - 1)(x - 4)$
\end{enumerate}
\item \begin{enumerate}
\item The sum is $0$ and the product is $-25$.
\item The sum is $-1$ and the product is $0$.
\item The sum is $-3$ and the product is $2$.
\item The sum is $-14/4 = -7/2$ and the product is $-8/4 = -2$.
\item The sum is $-1/4$ and the product is $-1/4$.\par
\emph{Vieta's formulas would give us a sum of $-8/16 = -1/2$ and a product of $1/16$, but this quadratic only has one root $-1/4$ (which is a double root).}
\end{enumerate}
\item \begin{enumerate}
\item $(x + 1)(x - 5) = \boxed{x^2 - 4x - 5}$
\item $(x - 2)(x + 2) = \boxed{x^2 - 4}$
\item $(3x + 2)\cdot x = \boxed{3x^2 + 2x}$
\item $(5x + 1)(5x + 1) = \boxed{25x^2 + 10x + 1}$
\item Using Vieta's formulas, $\boxed{x^2 - 3x + 2}$ would work.
\item Using Vieta's formulas, $x^2 + \frac{9}{2}x - \frac{5}{2}$ has the desired properties, but we need integer coefficients. Multiplying by 2 does not change the roots and gives $\boxed{2x^2 + 9x - 5}$.
\end{enumerate}
\item \begin{enumerate}
\item Let the two numbers be $9 + d$ and $9 - d$, where $d\geq 0$. Their product is
\begin{equation*}
(9 + d)(9 - d) = 81 - d^2\leq 81.
\end{equation*}
Therefore, the maximum possible product is $\boxed{81}$, which is attained when $d = 0$ and the two numbers are $9$ and $9$.
\item If the product is $-63$, then we have
\begin{equation*}
81 - d^2 = -63\implies d = 12.
\end{equation*}
The numbers are $9 + 12 = \boxed{21}$ and $9 - 12 = \boxed{-3}$.
\end{enumerate}
\item By Vieta's formulas, $r + s = 1$ and $rs = -1$. Then,
\begin{equation*}
r^2 + s^2 = (r + s)^2 - 2rs = 1^2 - 2\cdot (-1) = \boxed{3}.
\end{equation*}
\item \begin{enumerate}
\item By Vieta's formulas, $\boxed{x^2 - 7x + 3}$ would work.
\item The sum of the roots would be
\begin{equation*}
a^2 + b^2 = (a + b)^2 - 2ab = 7^2 - 2\cdot 3 = 43
\end{equation*}
and the product of the roots would be
\begin{equation*}
a^2b^2 = (ab)^2 = 3^2 = 9.
\end{equation*}
Hence we could take $\boxed{x^2 - 43x + 9}$.
\item The sum of the roots would be 
\begin{equation*}
\frac{1}{a} + \frac{1}{b} = \frac{a + b}{ab} = \frac{7}{3}
\end{equation*}
and the product of the roots would be 
\begin{equation*}
\frac{1}{ab} = \frac{1}{3}.
\end{equation*}
Hence we could take $x^2 - \frac{7}{3}x + \frac{1}{3}$, or after rescaling, $\boxed{3x^2 - 7x + 1}$.\par 
Alternatively, if $x = 1/y$, then the solutions for $y$ in the equation
\begin{equation*}
\left(\frac{1}{y}\right)^2 - 7\cdot\frac{1}{y} + 3 = 0
\end{equation*}
would be $y = 1/a$ and $y = 1/b$, by part (a). Multiplying through by $y^2$ gives us $1 - 7y + 3y^2 = 0$. This does not have $y = 0$ as a solution, so we did not introduce any new solutions after multiplying by $y^2$. Hence this quadratic has the required roots (in agreement with the first method).
\end{enumerate}
\item \begin{enumerate}
\item $x^4 + x^2 + 1 = (x^4 + 2x^2 + 1) - x^2 = (x^2 + 1)^2 - x^2 = \boxed{(x^2 + 1 - x)(x^2 + 1 + x)}$
\item $x^4 + 4 = (x^4 + 4x^2 + 4) - 4x^2 = (x^2 + 2)^2 - (2x)^2 = \boxed{(x^2 + 2 - 2x)(x^2 + 2 + 2x)}$
\end{enumerate}
\item \begin{enumerate}
\item We have
\begin{equation*}
9 = \left(x + \frac{1}{x}\right)^2 = x^2 + 2\cdot x\cdot\frac{1}{x} + \frac{1}{x^2} = x^2 + 2 + \frac{1}{x^2},
\end{equation*}
so $x^2 + \dfrac{1}{x^2} = \boxed{7}$.
\item We have
\begin{equation*}
27 = \left(x + \frac{1}{x}\right)^3 = \left(x^2 + 2 + \frac{1}{x^2}\right)\left(x + \frac{1}{x}\right) = x^3 + 3x + \frac{3}{x} + \frac{1}{x^3},
\end{equation*}
so 
\begin{equation*}
x^3 + \frac{1}{x^3} = 27 - 3\left(x + \frac{1}{x}\right) = 27 - 3\cdot 3 = \boxed{18}.
\end{equation*}
\end{enumerate}
\item First, $x = 0$ is not a root. Thus we can safely divide both sides by $x^2$ to get
\begin{equation*}
2\left(x^2 + \frac{1}{x^2}\right) - \left(x + \frac{1}{x}\right) - 6 = 0.
\end{equation*}
Letting $y = x + \frac{1}{x}$, we get $x^2 + \frac{1}{x^2} = y^2 - 2$, so our equation becomes
\begin{equation*}
2(y^2 - 2) - y - 6 = 0,
\end{equation*}
or $2y^2 - y - 10 = 0$. This factors as
\begin{equation*}
(2y - 5)(y + 2) = 0,
\end{equation*}
so $y = 5/2$ and $y = -2$ are the two solutions for $y$.\par 
When $y = -2$, we have
\begin{equation*}
x + \frac{1}{x} = -2\implies x^2 + 2x + 1 = 0.
\end{equation*}
This factors as $(x + 1)^2 = 0$, so the lone solution is $x = -1$.\par 
When $y = 5/2$, we have 
\begin{equation*}
x + \frac{1}{x} = \frac{5}{2}\implies 2x^2 - 5x + 2 = 0.
\end{equation*}
This factors as $(2x - 1)(x - 2) = 0$, so the solutions are $x = 1/2$ and $x = 2$.\par
In conclusion, the solutions are $\boxed{-1, 1/2, 2}$.\par 
\emph{Remark: This sort of trick can be used whenever the list of coefficients is palindromic, i.e. reads the same forwards or backwards.}
\end{enumerate}
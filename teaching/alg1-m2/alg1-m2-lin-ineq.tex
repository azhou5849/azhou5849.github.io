\section{Linear Inequalities}

\subsection{Review problems}

\begin{enumerate}
\item Solve each of the following inequalities, expressing the solution set in interval notation and graphing the solution set on a number line.
\begin{enumerate}
\item $x > -1$
\item $2x + 5\leq 7$
\item $7 - 3x < 5x - 9$
\item $(x + 4)^2 + x^2\geq (x + 2)^2 + (x + 1)^2$
\end{enumerate}
\item For each of the following combinations of inequalities, write the solution set in interval notation and graph the solution set on a number line.
\begin{enumerate}
\item $2\leq 3 + x < 6$
\item $x\geq 2 - x$ or $5 - 3x\geq x + 9$
\item $-9x + 4 < 8x + 2 < 2x + 7$
\item $-6x - 8\leq 4$ and $-x - 10\leq -7$
\item $-6x - 8\leq 4$ or $-x - 10\leq -7$
\end{enumerate}
\emph{In mathematics, the convention in almost all scenarios is that ``or'' is inclusive: when we join two separate statements with an ``or,'' the combined statement is true when \underline{at least} one of the original statements is true. For example, ``$x > 1$ or $x > 2$'' is true when $x = 3$.}
\item \begin{enumerate}
\item Write down a pair of linear inequalities, each involving a single variable $x$, for which no real number satisfies both of the inequalities simultaneously. In this case, the solution set for ``[first inequality] and [second inequality]'' is the \emph{empty set}, denoted $\emptyset$.
\item Write down a pair of linear inequalities, each involving a single variable $x$, for which every real number satisfies at least one of the inequalities. In this case, the solution set for ``[first inequality] or [second inequality]'' is the entire set of real numbers, denoted $\mathbb{R}$. We could also write $(-\infty, \infty)$.
\item Write down a pair of linear inequalities, each involving a single variable $x$, for which the only real number which does not satisfy at least one of the inequalities is $6.626$. Express this solution set in interval notation as a union ($\cup$) of two intervals.
\end{enumerate}
\item Solve each of the following inequalities, expressing the solution set in interval notation and graphing the solution set on a number line.
\begin{enumerate}
\item $\lvert x\rvert < 4$
\item $\lvert x - 4\rvert\geq 2$
\end{enumerate}
\item Graph the solution set of each of the following inequalities.
\begin{enumerate}
\item $x + y < -2$
\item $3x - y\geq 7$
\item $(x - 1)^2 + y^2\leq (x - 5)^2 + (y - 2)^2$
\item $\lvert x\rvert + \lvert y\rvert < 1$
\end{enumerate}
\item Graph the solution set of each of the following combinations of inequalities.
\begin{enumerate}
\item $x + y\leq -2$ and $3x - y > 7$
\item $x + y < -2$ or $3x - y > 7$
\item $2x + 3y\geq 4$ and $6y\leq 12 - 4x$
\item $3x + 3y < -5x < 5y$
\end{enumerate}
\item Supposing $a\leq b$ and $c\leq d$, which of the following must also be true?
\begin{enumerate}
\item $a + c\leq b + d$
\item $a - c\leq b - d$
\item $ac\leq bd$
\item $a/c\leq b/d$
\end{enumerate}
For the statements that were not guaranteed to be true, which ones are guaranteed to be true when we also assume $a\geq 0$ and $c\geq 0$?
\end{enumerate}


\subsection{Challenge problems}

\begin{enumerate}
\item Write down three inequalities for which the set of all points $(x,y)$ satisfying all of the three inequalities is the triangle with vertices $A = (0,0)$, $B = (14,0)$, and $C = (5,12)$, with vertex $A$ included in the solution set but vertices $B$ and $C$ excluded.
\item Hayasaka Ayaka
\end{enumerate}

% linear programming
% abs value of difference of abs values, find solution set


\subsection{Answers}
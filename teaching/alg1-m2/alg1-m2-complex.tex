\section{Complex Numbers}

Given a complex number $z = x + yi$, the \emph{magnitude} of $z$, denoted $\lvert z\rvert$, is $\sqrt{x^2 + y^2}$. For example,
\begin{equation*}
\lvert -1 + 3i\rvert = \sqrt{(-1)^2 + 3^2} = \sqrt{10}.
\end{equation*}

\subsection{Review problems}

\begin{enumerate}
\item Let $z = 2 + i$. Compute each of the following:
\begin{enumerate}
\item $\Re z$, the real part of $z$
\item $\Im z$, the imaginary part of $z$
\item $\bar{z}$, the complex conjugate of $z$
\item $\lvert z\rvert$, the magnitude of $z$
\end{enumerate}
\item Let $z = 5 + 3i$ and $w = 7 - 9i$. Compute each of the following:
\begin{enumerate}
\item $z + w$
\item $z - w$
\item $zw$
\item $z/w$
\end{enumerate}
\item \begin{enumerate}
\item Compute $(1 + i)^2$.
\item Compute $(1 + i)^{2022}$.
\end{enumerate}
\item \begin{enumerate}
\item Find the two complex numbers whose square is $-180$.
\item Find the four complex numbers whose fourth power is $16$.
\end{enumerate}
\item Which of the following statements is always true (whenever both sides are defined)?
\begin{enumerate}[label=(\Roman*)]
\item $\bar{z + w} = \bar{z} + \bar{w}$
\item $\bar{z - w} = \bar{z} - \bar{w}$
\item $\bar{z\cdot w} = \bar{z}\cdot\bar{w}$
\item $\bar{z/w} = \bar{z}/\bar{w}$
\end{enumerate}
\item Which of the following statements is always true (whenever both sides are defined)?
\begin{enumerate}[label=(\Roman*)]
\item $\lvert z + w\rvert = \lvert z\rvert + \lvert w\rvert$
\item $\lvert z - w\rvert = \lvert z\rvert - \lvert w\rvert$
\item $\lvert z\cdot w\rvert = \lvert z\rvert\cdot\lvert w\rvert$
\item $\lvert z/w\rvert = \lvert z\rvert / \lvert w\rvert$
\end{enumerate}
\end{enumerate}


\subsection{Challenge problems}

\begin{enumerate}[resume]
\item Let $\omega = \frac{1}{2} + \frac{\sqrt{3}}{2}i$. Compute $\omega^{2024}$.
\item Every complex number can be represented as a point in the plane by associating $a + bi$ with the point $(a,b)$. (When we do this, we get the \emph{complex plane} or \emph{Argand diagram}.)
\begin{enumerate}
\item Complex numbers with negative real part and positive imaginary part are represented by points in which quadrant?
\item What geometric transformation is represented by complex conjugation?
\item What geometric transformation is represented by multiplication by $i$?
\end{enumerate}
\item Let $ax^2 + bx + c$ be a quadratic with \underline{real} coefficients. Show that if $r$ is a root, then so is $\bar{r}$.
\item Find a complex number $z$ such that $z^2 = -15 - 8i$.\par
\emph{One can show that in fact, every complex number has a complex number square root: we do not need to add any new numbers besides $i$!}
\end{enumerate}


\newpage
\subsection{Answers}

\begin{enumerate}
\item \begin{enumerate}
\item $2$
\item $1$
\item $2 - i$
\item $\sqrt{5}$
\end{enumerate}
\item \begin{enumerate}
\item $12 - 6i$
\item $-2 + 12i$
\item $62 - 24i$
\item $\frac{4}{65} + \frac{33}{65}i$
\end{enumerate}
\item \begin{enumerate}
\item $(1 + i)^2 = 1 + 2i + i^2 = \boxed{2i}$
\item $(1 + i)^{2022} = (2i)^{1011} = 2^{1011}i^{1011} = 2^{1011}i^3 = \boxed{-2^{1011}i}$
\end{enumerate}
\item \begin{enumerate}
\item The two square roots of $-180$ are
\begin{equation*}
\pm\sqrt{-180} = \pm\sqrt{180}i = \boxed{\pm 6\sqrt{5}i}.
\end{equation*}
\item Every fourth root is a square root of a square root, so we can start by finding the square roots of $16$, which are $4$ and $-4$. The square roots of $4$ are $\pm 2$, while the square roots of $-4$ are $\pm 2i$, so the final list is $\boxed{\pm 2, \pm 2i}$.
\end{enumerate}
\item All four statements (I), (II), (III), (IV) are always true.
\item Statements (III) and (IV) are always true, while a counterexample for (I) and (II) can be found by taking $z = 1$ and $w = i$.
\item Computing several powers of $\omega$, we can eventually find that $\omega^6 = 1$. Therefore,
\begin{equation*}
\omega^{2024} = \omega^{2022}\cdot\omega^2 = \omega^2 = \boxed{-\frac{1}{2} + \frac{\sqrt{3}}{2}i}.
\end{equation*}
\item 
\end{enumerate}
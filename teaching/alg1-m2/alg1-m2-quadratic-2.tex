\section{Quadratics (II)}

\subsection{Review problems}

\begin{enumerate}
\item Find all solutions to the following equations. You may use any method.
\begin{enumerate}
\item $x^2 = 2x + 80$
\item $8x = 4x^2 + 1$
\item $x^2 + 25 = 10x$
\item $-6x^2 - 1 - 5x = 0$
\item $x^2 + x + 10 = 0$
\end{enumerate}
\item Complete the square in each of the following expressions. That is, write each of the following in the form $a(x - h)^2 + k$ for constants $a,h,k$.
\begin{enumerate}
\item $x^2 - 18x + 81$
\item $x^2 + 16x + 59$
\item $-6x^2 + 24x + 5$
\item $9x^2 - 7x + 5$
\item $-\frac{x^2}{2s} - 2\pi tx$, where $s$ and $t$ are constants.
\end{enumerate}
\item For what real values of $c$ does the quadratic $x^2 + 8x + c$ have
\begin{enumerate}
\item two non-real roots?
\item one real root?
\item two real roots?
\item For what positive integers $c$ does the quadratic have two rational roots?
\end{enumerate}
\item \begin{enumerate}
\item A parabola is given by $y = x^2 + 18x + 81$. Find the intercepts and the vertex.
\item A parabola has $x$-intercepts $(-8,0)$ and $(10,0)$ and $y$-intercept $(0, 560)$. Find an equation for the parabola and find the vertex.
\item A parabola has vertex $(-8,-4)$ and passes through the point $(6, 192)$. Find an equation for the parabola and find the intercepts.
\end{enumerate}
\item When Linda sells boxes of cookies, she finds that when she prices each box at $\$x$, she manages to sell $56 - 4x$ boxes.
\begin{enumerate}
\item What price should she set in order to maximize revenue?
\item It costs Linda $\$2$ to make each box. What price should she set to maximize profit?
\end{enumerate}
\item \begin{enumerate}
\item Write down an equation for a circle centered at $(-5, 1)$ with radius $7$.
\item Find the center and radius of the circle with equation $x^2 + y^2 + 6x + 10y + 3 = 0$.
\item Find the value of $E$ for which the graph of the equation $x^2 + y^2 - 4x - 4y = E$ is a single point. What is that point?
\end{enumerate}
\item Find the two points where the circles $x^2 + y^2 = 125$ and $x^2 - 16x + y^2 - 12y = 25$ intersect.
\end{enumerate}


\subsection{Challenge problems}

\begin{enumerate}[resume]
\item \begin{enumerate}
\item Write down a quadratic with \underline{real} coefficients whose discriminant is a perfect square integer but whose roots are irrational.
\item Write down a quadratic with \underline{complex} coefficients whose discriminant is a negative real number but whose roots are real.
\end{enumerate}
\item Every parabola has a point $F$, called the \emph{focus}, and a line $\ell$, called the \emph{directrix}, with the property that the points $P$ on the parabola are precisely those for which the distance from $P$ to $F$ is the same as the distance from $P$ to $\ell$. The line through $F$ perpendicular to $\ell$ is the axis of symmetry of the parabola.
\begin{enumerate}
\item Let $F$ be the point $(1,3)$ and let $\ell$ be the line $y = -1$. For the parabola with focus $F$ and directrix $\ell$, what point is the vertex?
\item Find the two points on the line $y = 3$ that lie on the parabola.
\item Write down an equation for the parabola.
\end{enumerate}
\item (Even more challenging than usual) Find all real numbers $x$ for which
\begin{equation*}
\sqrt{5 - x} = 5 - x^2.
\end{equation*}
\end{enumerate}


\newpage
\subsection{Answers}

\begin{enumerate}
\item \begin{enumerate}
\item $-8$ and $10$
\item $1\pm\frac{\sqrt{3}}{2}$
\item $5$
\item $-1/2$ and $-1/3$
\item $\frac{-1\pm\sqrt{39}i}{2}$
\end{enumerate}
\item \begin{enumerate}
\item $(x - 9)^2$
\item $(x + 8)^2 - 5$
\item $-6(x - 2)^2 + 29$
\item $\displaystyle 9\left(x - \frac{7}{18}\right)^2 + \frac{131}{36}$
\item $\displaystyle -\frac{1}{2s}(x + 2\pi st)^2 + 4\pi^2 st^2$
\end{enumerate}
\item In this problem, we are interested in the discriminant $\Delta = 8^2 - 4\cdot 1\cdot c = 64 - 4c$. Note that all coefficients are real.
\begin{enumerate}
\item We have two non-real roots if and only if $\Delta < 0$, so
\begin{equation*}
64 - 4c < 0\implies\boxed{c > 16}.
\end{equation*}
\item We have one real root if and only if $\Delta = 0$, so 
\begin{equation*}
64 - 4c = 0\implies\boxed{c = 16}.
\end{equation*}
\item We have two real roots if and only if $\Delta > 0$, so
\begin{equation*}
64 - 4c > 0\implies\boxed{c < 16}.
\end{equation*}
\item All coefficients are integers, so we get two rational roots if and only if $\Delta$ is a positive perfect square. Here we have that $\Delta$ is even and $\Delta < 64$ since $c$ must be a positive integer, so the solutions come from setting
\begin{align*}
64 - 4c = 4 &\implies c = \boxed{15}, \\
64 - 4c = 16 &\implies c = \boxed{12}, \\
64 - 4c = 36 & \implies c = \boxed{7}.
\end{align*}
\end{enumerate}
\item \begin{enumerate}
\item $x$-intercept $(-9,0)$; $y$-intercept $(0,81)$; vertex $(-9,0)$
\item equation $y = -7(x + 8)(x - 10)$; vertex $(1, 567)$
\item equation $y = (x + 8)^2 - 4$; $x$-intercepts $(-10,0)$ and $(-6,0)$; $y$-intercept $(0,60)$
\end{enumerate}
\item \begin{enumerate}
\item Linda's revenue, in dollars, is $x(56 - 4x)$. This is maximised when the value of $x$ is the vertex of the parabola $y = x(56 - 4x)$. The quadratic here has roots $x = 0$ and $x = 14$, so the vertex has $x$-coordinate $(0 + 14)/2 = \boxed{7}$.
\item Linda's profit, in dollars, is $(x - 2)(56 - 4x)$, as we profit $x - 2$ from each box. This time, the vertex has $x$-coordinate $(2 + 14)/2 = \boxed{8}$.
\end{enumerate}
\item \begin{enumerate}
\item $(x + 5)^2 + (y - 1)^2 = 49$
\item center $(-3, -5)$; radius $\sqrt{31}$
\item Completing the square, the equation becomes
\begin{equation*}
(x - 2)^2 + (y - 2)^2 = E + 8.
\end{equation*}
This has exactly one solution precisely when $E + 8 = 0$, in which case $\boxed{E = -8}$. The lone solution in that case is the point $\boxed{(2,2)}$.
\end{enumerate}
\item Subtracting the two given equations from each other yields
\begin{equation*}
16x + 12y = 100, \tag{$\dagger$}
\end{equation*}
or $y = -\frac{4}{3}x + \frac{25}{3}$. Substituting this into the first given equation,
\begin{align*} 
x^2 + \left(-\frac{4}{3}x + \frac{25}{3}\right)^2 &= 125, \\
9x^2 + (-4x + 25)^2 &= 1125, \\
9x^2 + 16x^2 - 200x + 625 &= 1125, \\
25x^2 - 200x &= 500, \\
x^2 - 8x - 20 &= 0, \\
(x - 10)(x + 2) &= 0.
\end{align*}
Thus the two possible $x$-coordinates for a point of intersection are $x = 10$ and $x = -2$. Substituting $x = 10$ into $(\dagger)$ yields $y = -5$, while substituting $x = -2$ yields $y = 11$, so the two points of intersection are $\boxed{(10, -5)}$ and $\boxed{(-2, 11)}$.
\item \begin{enumerate}
\item $x^2 - 2\sqrt{2}x + 1$
\item $ix^2 - i$
\end{enumerate}
\item \begin{enumerate}
\item The line perpendicular to $\ell$ through $F$ is $x = 1$, so it hits $\ell$ at the point $A = (1,-1)$. The vertex is then the midpoint of $F$ and $A$, which is $\boxed{(1,1)}$.
\item Let $P = (x,3)$ be a point on the parabola. Then the distance from $P$ to $\ell$ is $3 - (-1) = 4$, while $PF = \lvert x - 1\rvert$, so we have
\begin{equation*}
\lvert x - 1\rvert = 4.
\end{equation*}
The two solutions are $x = -3$ and $x = 5$, so the points are $\boxed{(-3,3)}$ and $\boxed{(5,3)}$.
\item Since the vertex is $(1,1)$, we can write down an equation
\begin{equation*}
y = a(x - 1)^2 - 1
\end{equation*}
for some constant $a$. Plugging in the point $(5,3)$ gives us
\begin{equation*}
3 = a(5 - 1)^2 - 1,
\end{equation*}
from which it follows that $a = 1/4$. Hence our equation is $\boxed{y = \frac{1}{4}(x - 1)^2 - 1}$.
\end{enumerate}
\item Squaring both sides,
\begin{equation*}
5 - x = 5^2 - 2\cdot 5\cdot x^2 + x^4,
\end{equation*}
which rearranges to
\begin{equation*}
5^2 - (2x^2 + 1)\cdot 5 + (x^4 + x) = 0.
\end{equation*}
Using the quadratic formula to solve for $5$ (?!?!),
\begin{align*}
5 &= \frac{(2x^2 + 1)\pm\sqrt{(2x^2 + 1)^2 - 4\cdot 1\cdot (x^4 + x)}}{2} \\
&= \frac{(2x^2 + 1)\pm\sqrt{(4x^4 + 4x^2 + 1) - (4x^4 + 4x)}}{2} \\
&= \frac{(2x^2 + 1)\pm\sqrt{4x^2 - 4x + 1}}{2} \\
&= \frac{(2x^2 + 1)\pm (2x - 1)}{2}.
\end{align*}
We now consider two cases based on the $\pm$ sign. For the ``plus'' case,
\begin{align*} 
\frac{(2x^2 + 1) + (2x - 1)}{2} &= 5, \\
x^2 + x &= 5, \\
x^2 + x - 5 &= 0, \\
x &= \frac{-1\pm\sqrt{1^2 - 4\cdot 1\cdot (-5)}}{2} \\
&= \frac{-1\pm\sqrt{21}}{2}.
\end{align*}
The conditions for $x$ to be a valid solution to the initial equation are that $x\leq 5$, so that $\sqrt{5 - x}$ is defined, and that $x^2\leq 5$, so that $5 - x^2$ is non-negative. Of the two solutions found in this ``plus'' case, only $\displaystyle x = \boxed{\frac{-1 + \sqrt{21}}{2}}$ satisfies this.\par 
For the ``minus'' case,
\begin{align*} 
\frac{(2x^2 + 1) - (2x - 1)}{2} &= 5, \\
x^2 - x + 1 &= 5, \\
x^2 - x - 4 &= 0, \\
x &= \frac{1\pm\sqrt{(-1)^2 - 4\cdot 1\cdot (-4)}}{2} \\
&= \frac{1\pm\sqrt{17}}{2}.
\end{align*}
This time, the solution that satisfies the initial equation is $\displaystyle x = \boxed{\frac{1 - \sqrt{17}}{2}}$.
\end{enumerate}
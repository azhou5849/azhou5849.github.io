\section{Parametric and polar equations}

\subsection{Review problems}

\begin{enumerate}
\item \emph{Graphing parametric curves.} Graph each of the following and write down equivalent equations using $x$ and $y$ only.
\begin{enumerate}
\item $x(t) = 2 + 3t$ and $y(t) = 4 - t$
\item $x(t) = \cos t$ and $y(t) = \sin t$
\item $x(t) = t$ and $y(t) = t^3$
\item $x(t) = 1 + 2\sin t$ and $y(t) = 3 - 2\cos t$
\end{enumerate}
\item \emph{Parameterizing standard curves.} Write down parametric equations for each of the following curves in the plane.
\begin{enumerate}
\item The line passing through $(3,2)$ and $(6,7)$
\item The circle with center $(-2,3)$ and radius 4
\item The ray emanating from $(0,1)$ passing through $(-4,-4)$
\end{enumerate}
\item \emph{Plane polar coordinates.} Make each of the following conversions.
\begin{enumerate}
\item Polar $(3,0)$ to cartesian (rectangular, $(x,y)$)
\item Polar $(2,2\pi/3)$ to cartesian
\item Cartesian $(-4,0)$ to polar
\item Cartesian $(1,1)$ to polar
\end{enumerate}
\item \emph{Lines and circles.} Graph each of the following and write down equivalent equations using cartesian coordinates ($x$ and $y$).
\begin{enumerate}
\item $r = 5$
\item $\theta = \arctan(1/4)$, where $r$ is allowed to be any real number
\item $r = -2\sin\theta$
\item $r = 2\cos\theta + 4\sin\theta$
\item $r = \frac{7}{3\cos\theta - 2\sin\theta}$
\end{enumerate}
\item \emph{Lima\c{c}ons and roses.} Match each equation with its corresponding graph. (Not every graph has a corresponding equation.)
\begin{table}[H]
\centering 
\begingroup\setlength{\tabcolsep}{40pt}
\begin{tabular}{cc}
Equation & Graph \\ \hline
$r = 1 + 3\cos\theta$ & Lima\c{c}on with indentation \\
$r = 3 + 3\cos\theta$ & Rose with 3 petals \\
$r = 5 + 3\cos\theta$ & Lima\c{c}on with inner loop \\
$r = 7 + 3\cos\theta$ & Convex lima\c{c}on\\
$r = \cos\theta$ & Rose with 4 petals \\
$r = \cos(2\theta)$ & Circle \\
$r = \cos(3\theta)$ & Rose with 2 petals \\
{} & Rose with 6 petals \\
{} & Cardioid (lima\c{c}on with cusp)
\end{tabular}
\endgroup
\end{table}
\item \emph{Arc length parameterization.} An \emph{arc length parameterization} of a curve in the plane is a parameterization $(x(s), y(s))$ with the property that whenever a particle moves along the curve from $s = s_1$ to $s = s_2$, where $s_1 < s_2$, the distance it travels is $s_2 - s_1$.
\begin{enumerate}
\item Prove that $x(s) = 1 + \frac{3}{5}s$ and $y(s) = 1 + \frac{4}{5}s$ is an arc length parameterization for the line passing through $(1,1)$ and $(4,5)$.
\item Find an arc length parameterization for the circle centered at $(1,-1)$ with radius 2 that traverses the circle clockwise starting at $(3,-1)$ when $s = 0$.
\end{enumerate}
\item \emph{Roulettes.} Let $P$ be a point on a circle of radius 1. Parameterize the path traced by $P$ as the circle rolls along each of the following, assuming that $P$ is at the point of contact at time $t = 0$. (You are free to choose exactly where the initial point of contact is.)
\begin{enumerate}
\item \emph{Cycloid.} Rolling on top of the $x$-axis
\item \emph{Cardioid.} Rolling on the outside of a circle of radius 1
\item \emph{Nephroid.} Rolling on the outside of a circle of radius 2
\item \emph{Deltoid.} Rolling on the inside of a circle of radius 3
\item \emph{Astroid.} Rolling on the inside of a circle of radius 4
\end{enumerate}
\emph{Remark:} The cardioid and nephroid are examples of \emph{epicycloids} while the deltoid and astroid are examples of \emph{hypocycloids}.
\end{enumerate}


\subsection{Challenge problems}

\begin{enumerate}\setcounter{enumi}{7}
\item Let $e > 0$ and $\ell > 0$ be given. The conic section with eccentricity $e$ whose focus is the point $F = (0,0)$ and whose directrix is the line $x = -\ell$ is the set of all points satisfying
\begin{equation*}
\frac{\text{distance from \ensuremath{P} to \ensuremath{F}}}{\text{distance from \ensuremath{P} to the directrix}} = e.
\end{equation*}
Note that when $e = 1$, this matches the usual focus-directrix definition of the parabola.
\begin{enumerate}
\item Show that the above curve has polar equation
\begin{equation*}
r = \frac{e\ell}{1 - e\cos\theta}.
\end{equation*}
\item Convert the polar equation to cartesian form.
\item Show that when $0 < e < 1$, the conic is an ellipse. What is the other focus of the ellipse?
\item Show that when $e > 1$, the conic is a hyperbola. In terms of $e$ and/or $\ell$, what are the slopes of the asymptotes?
\item Suppose the conic passes through $(-p,0)$, where $p > 0$. Express $\ell$ in terms of $e$ and $p$.
\item Holding $p$ fixed, what happens to the conic and the directrix as $e$ approaches $0$?
\end{enumerate}\newpage
\item Parameterization works for curves in three dimensions as well.
\begin{enumerate}
\item Graph the line $x(t) = t$; $y(t) = 1 + 2t$; $z(t) = 2 + t$.
\item Graph the \emph{helix} $x(t) = \cos t$; $y(t) = \sin t$; $z(t) = t$.
\item Find the intersection point (if it exists) of the lines
\begin{equation*}
x_1(t) = 2 - t;\quad y_1(t) = 2 + t;\quad z_1(t) = 3t
\end{equation*}
and
\begin{equation*}
x_2(t) = -1 + t;\quad y_2(t) = 7 - 2t;\quad z_2(t) = 2 + t.
\end{equation*}
\end{enumerate}
\item By using two parameters, we can describe surfaces in three dimensions. For each pair of values $(u,v)$, we get a corresponding point on the surface.
\begin{enumerate}
\item Graph the plane $x(u,v) = u + v$, $y(u,v) = u - v$, and $z(u,v) = 1 + 4u + 6v$. Write down an equation for this plane using only $x$, $y$, and $z$.
\item Graph the (infinite) cylinder $x(u,v) = 2\cos u$, $y(u,v) = 2\sin u$, and $z(u,v) = v$.
\item Assuming the Earth is a perfect sphere of radius $R$ centered at the origin with the prime meridian and equator intersecting at $(R,0,0)$, let $\phi$ denote latitude, ranging from $-90^{\circ}$ at the south pole to $90^{\circ}$ at the north pole, and let $\theta$ denote longitude, increasing from $-180^{\circ}$ to $180^{\circ}$ moving eastward with the prime meridian at $\theta = 0^{\circ}$. Parameterize points on the surface of the Earth using $\theta$ and $\phi$.
\end{enumerate}
\end{enumerate}


\newpage
\subsection{Answers}

\begin{enumerate}
\item \href{https://help.desmos.com/hc/en-us/articles/4406906208397-Parametric-Equations}{Checking parametric graphs with Desmos}
\begin{enumerate}
\item $x + 3y = 14$
\item $x^2 + y^2 = 1$
\item $y = x^3$
\item $(x - 1)^2 + (y - 3)^2 = 4$
\end{enumerate}
\item \begin{enumerate}
\item $x(t) = 3 + 4t$ and $y(t) = 2 + 5t$
\item $x(t) = -2 + 4\cos t$ and $y(t) = 3 + 4\sin t$
\item $x(t) = -4t$ and $y(t) = 1 - 4t$ with $t\geq 0$
\end{enumerate}
\item \begin{enumerate}
\item $(3,0)$
\item $(-1,\sqrt{3})$
\item $(4,\pi)$
\item $(\sqrt{2},\pi/4)$
\end{enumerate}
\item \href{https://help.desmos.com/hc/en-us/articles/4406895312781-Polar-Graphing}{Checking polar graphs with Desmos}
\begin{enumerate}
\item $x^2 + y^2 = 25$
\item $y = x/4$
\item $x^2 + (y + 1)^2 = 1$
\item $(x - 1)^2 + (y - 2)^2 = 5$
\item $3x - 2y = 7$
\end{enumerate}
\item ``Rose with 2 petals'' and ``Rose with 6 petals'' are unused.
\begin{table}[H]
\centering 
\begingroup\setlength{\tabcolsep}{40pt}
\begin{tabular}{cc}
Equation & Graph \\ \hline
$r = 1 + 3\cos\theta$ & Lima\c{c}on with inner loop \\
$r = 3 + 3\cos\theta$ & Cardioid (lima\c{c}on with cusp) \\
$r = 5 + 3\cos\theta$ & Lima\c{c}on with indentation \\
$r = 7 + 3\cos\theta$ & Convex lima\c{c}on \\
$r = \cos\theta$ & Circle \\
$r = \cos(2\theta)$ & Rose with 4 petals \\
$r = \cos(3\theta)$ & Rose with 3 petals
\end{tabular}
\endgroup
\end{table}
\item \begin{enumerate}
\item Let $s_1 < s_2$. The parametric equations given for $x(s)$ and $y(s)$ describe a line, so we can compute distance according to the distance formula (Pythagorean theorem). The distance traveled from $(x(s_1), y(s_1))$ to $(x(s_2), y(s_2))$ along the line is
\begin{align*}
& \sqrt{(x(s_2) - x(s_1))^2 + (y(s_2) - y(s_1))^2} \\
&= \sqrt{\left(1 + \frac{3}{5}s_2 - 1 - \frac{3}{5}s_1\right)^2 + \left(1 + \frac{4}{5}s_2 - 1 - \frac{4}{5}s_1\right)^2} \\
&= \sqrt{(s_2 - s_1)^2\left(\left(\frac{3}{5}\right)^2 + \left(\frac{4}{5}\right)^2\right)} \\
&= s_2 - s_1.
\end{align*}
\item The ``usual'' parameterization $x(t) = 1 + 2\cos t$ and $y(t) = -1 + 2\sin t$ starts at $(3,-1)$ and travels counterclockwise. We travel 1 radian per unit time and the radius is 2, so the arc length traveled per unit time is $1\cdot 2 = 2$. Therefore, to get a clockwise traversal at unit speed, we need to go backwards and take twice as long. Hence we set $s = -2t$, or $t = -s/2$, to get a new parameterization $x(s) = 1 + 2\cos(s/2)$ and $y(s) = -1 - 2\sin(s/2)$.
\end{enumerate}
\item \begin{enumerate}
\item $x(t) = t - \sin t$ and $y(t) = 1 - \cos t$
\item $x(t) = 2\cos t - \cos(2t)$ and $y(t) = 2\sin t - \sin(2t)$
\item $x(t) = 3\cos t - \cos(3t)$ and $y(t) = 3\sin t - \sin(3t)$
\item $x(t) = 2\cos t + \cos(2t)$ and $y(t) = 2\sin t - \sin(2t)$
\item $x(t) = 3\cos t + \cos(3t)$ and $y(t) = 3\sin t - \sin(3t)$
\end{enumerate}
\item \begin{enumerate}
\item The distance from $P$ to $F = (0,0)$ is just $r$, while the distance from $P$ to the directrix is $x + \ell = r\cos\theta + \ell$, so
\begin{equation*}
\frac{r}{r\cos\theta + \ell} = e.
\end{equation*}
Solving for $r$ in terms of $e, \ell, \theta$ gives the desired result.
\item Writing the above equation in the form $r = e(x + \ell)$, we can square both sides and rearrange to get
\begin{equation*}
(1 - e^2)x^2 - 2e^2\ell x + y^2 = e^2\ell^2.
\end{equation*}
\item When $0 < e < 1$, the coefficients of $x^2$ and $y^2$ have the same sign and there is no $xy$ term, so we get an ellipse. The center of the ellipse has $x$-coordinate $\frac{e^2\ell}{1 - e^2}$ and $y$-coordinate $0$, while one focus is $F = (0,0)$, so the other focus must be $F' = (\frac{2e^2\ell}{1 - e^2}, 0)$.
\item When $e > 1$, the coefficients of $x^2$ and $y^2$ have opposite sign, so we get a hyperbola. The asymptotes have slope $\pm\sqrt{e^2 - 1}$: when we divide both sides by $x^2$ and consider very large values of $x$, we have $(y/x)^2 + (1 - e^2)\approx 0$. Alternatively, the asymptotes correspond to when the denominator is $0$ in the polar equation, and when $1 - e\cos\theta = 0$ we have $\tan\theta = \pm\sqrt{e^2 - 1}$.
\item We set $\theta = \pi$ and $r = p$ in the polar equation to get $p = \frac{e\ell}{1 + e}$, or $\ell = p(1 + e)/e$.
\item Holding $p$ fixed, as $e$ approaches $0$ we see that $\ell$ grows without bound, so the directrix slides to infinity to the left. In terms of $p$ and $e$ we have $r = \frac{p(1 + e)}{1 - e\cos\theta}$, and as $e$ approaches 0 with $p$ fixed, the equation approaches $r = p$, the circle of radius $p$.
\end{enumerate}
\item \begin{enumerate}
\item \href{https://www.desmos.com/3d/cordytfslf}{Desmos graph}
\item \href{https://www.desmos.com/3d/im6ctd6w9e}{Desmos graph}
\item If there is an intersection point, suppose that for the first line it corresponds to parameter value $t = u$ and that in the second line it corresponds to parameter value $t = v$ (where $u$ and $v$ may be but are not necessarily equal). Matching coordinates, we have the system
\begin{align*}
2 - u &= -1 + v, \\
2 + u &= 7 - 2v, \\
3u &= 2 + v.
\end{align*}
This system does not have a solution, so there is no intersection point.\par
\emph{Remark:} If the second line had $z_2(t) = 1 + t$ instead, then we would get the solution $u = 1$ and $v = 2$, corresponding to the point $(1,3,3)$. Note that we would not find this point just by setting $x_1(t) = x_2(t)$, etc., so we do need to use a new variable for each line when setting up the system. If we imagine particles traveling according along their respective lines according to the given parametric equations, then they both pass through the same point but at different times (so the particles would not meet, but their trajectories do).
\end{enumerate}
\item \begin{enumerate}
\item \href{https://www.desmos.com/3d/ydcud6diqh}{Desmos graph}, $z = 1 + 5x - y$
\item \href{https://www.desmos.com/3d/qpweomovuq}{Desmos graph}
\item $x(\phi,\theta) = R\cos\phi\cos\theta$\par
$y(\phi,\theta) = R\cos\phi\sin\theta$\par
$z(\phi,\theta) = R\sin\phi$
\end{enumerate}
\end{enumerate}
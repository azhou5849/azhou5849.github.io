\section{Trig (III): Identities}

\subsection{Review problems}

\begin{enumerate}
\item \emph{Angle sum and difference identities.}
\begin{enumerate}
\item $\sin(\alpha + \beta) = $
\item $\sin(\alpha - \beta) = $
\item $\cos(\alpha + \beta) = $
\item $\cos(\alpha - \beta) = $
\item $\tan(\alpha + \beta) = $
\item $\tan(\alpha - \beta) = $
\end{enumerate}
\item \emph{Calculation.}
\begin{enumerate}
\item Compute $\cos(75^{\circ})$.
\item Supposing $\sin(\alpha) = 5/13$ and $\sin(\beta) = 3/5$, compute $\sin(\alpha + \beta)$.
\end{enumerate}
\item \emph{Double angle identities.} (There are three useful expressions for $\cos(2\theta)$.)
\begin{enumerate}
\item $\sin(2\theta) = $
\item $\cos(2\theta) = $
\item $\cos(2\theta) = $
\item $\cos(2\theta) = $
\item $\tan(2\theta) = $
\end{enumerate}
\item \emph{Half angle calculations.}
\begin{enumerate}
\item If $\cos\theta = 3/5$, what are the possible values of $\cos(\theta/2)$?
\item Calculate $\tan(\pi/8)$.
\end{enumerate}
\item \emph{The tangent half-angle substitution.} Let $t = \tan(\theta/2)$. Show that
\begin{equation*}
\cos\theta = \frac{1 - t^2}{1 + t^2}\quad\text{and}\quad\sin\theta = \frac{2t}{1 + t^2}.
\end{equation*}
These are sometimes used in calculus to change expressions involving trig functions into expressions involving rational functions.
\item \emph{Product-to-sum and sum-to-product identities.}
\begin{enumerate}
\item $\cos\alpha\cos\beta = $
\item $\sin\alpha\sin\beta = $
\item $\sin\alpha\cos\beta = $
\item $\sin\theta + \sin\phi = $
\item $\cos\theta + \cos\phi = $
\end{enumerate}
\item \emph{Some equations.} Find all real solutions to the following equations. (These will generally involve an integer parameter and may involve inverse trig functions.) % latter two based on dec 2019 gbml
\begin{enumerate}
\item $\sin\alpha = 1$
\item $2\cos(2t) + 5 = 8\cos t$
\item $\sin\theta = \cos^2(2\pi/9) - \sin^2(2\pi/9)$
\item $3\sin x + 5\cos x = 23/4$
\end{enumerate}
\end{enumerate}


\subsection{Challenge problems}

\begin{enumerate}\setcounter{enumi}{7}
\item For each non-negative integer $n$, the \emph{degree-$n$ Chebyshev polynomial of the first kind}, denoted $T_n(X)$, is defined by the property that $T_n(\cos\theta) = \cos(n\theta)$ for all real $\theta$. Thus $T_0(X) = 1$ and $T_1(X) = X$.
\begin{enumerate}
\item Compute $T_2(X)$, $T_3(X)$, and $T_4(X)$.
\item Find the roots of $T_n(X)$ for $n = 0, 1, 2, 3, 4$.
\item Prove that $T_{n + 1}(X) = 2X\cdot T_n(X) - T_{n - 1}(X)$ for all positive integers $n$.
\end{enumerate}
\item Show that for all positive integers $n$,
\begin{equation*}
1 + 2\cos\theta + 2\cos(2\theta) + \cdots + 2\cos(n\theta) = \frac{\sin((n + \frac{1}{2})\theta)}{\sin(\frac{1}{2}\theta)}.
\end{equation*}
\emph{Remark:} If we denote either side of this equation by $D_n(\theta)$, then the sequence of functions $D_0, D_1, D_2, \ldots$ is known as the \emph{Dirichlet kernel}.
\item Prove the following for a triangle $ABC$:
\begin{enumerate}
\item $\tan A + \tan B + \tan C = \tan A\tan B\tan C$
\item $\cot(\frac{A}{2}) + \cot(\frac{B}{2}) + \cot(\frac{C}{2}) = \cot(\frac{A}{2})\cot(\frac{B}{2})\cot(\frac{C}{2})$
\item $\sin(2A) + \sin(2B) + \sin(2C) = 4\sin A\sin B\sin C$
\end{enumerate}
\end{enumerate}


\newpage
\subsection{Answers}

\begin{enumerate}
\item \begin{enumerate}
\item $\sin\alpha\cos\beta + \cos\alpha\sin\beta$
\item $\sin\alpha\cos\beta - \cos\alpha\sin\beta$
\item $\cos\alpha\cos\beta - \sin\alpha\sin\beta$
\item $\cos\alpha\cos\beta + \sin\alpha\sin\beta$
\item $\frac{\tan\alpha + \tan\beta}{1 - \tan\alpha\tan\beta}$
\item $\frac{\tan\alpha - \tan\beta}{1 + \tan\alpha\tan\beta}$
\end{enumerate}
\item \begin{enumerate}
\item $\frac{\sqrt{6} - \sqrt{2}}{4}$
\item $56/65$
\end{enumerate}
\item The answers for (b), (c), and (d) can be rearranged.
\begin{enumerate}
\item $2\sin\theta\cos\theta$
\item $\cos^2\theta - \sin^2\theta$
\item $2\cos^2\theta - 1$
\item $1 - 2\sin^2\theta$
\item $\frac{2\tan\theta}{1 - \tan^2\theta}$
\end{enumerate}
\item \begin{enumerate}
\item $\pm 2/\sqrt{5}$
\item $\sqrt{2} - 1$
\end{enumerate}
\item We have $\sec^2(\theta/2) = 1 + \tan^2(\theta/2) = 1 + t^2$, so $\cos^2(\theta/2) = \frac{1}{1 + t^2}$. Then,
\begin{equation*}
\cos\theta = 2\cos^2(\theta/2) - 1 = \frac{1 - t^2}{1 + t^2}.
\end{equation*}
To find $\sin\theta$, we compute
\begin{equation*}
\tan\theta = \frac{2\tan(\theta/2)}{1 - \tan^2(\theta/2)} = \frac{2t}{1 - t^2}
\end{equation*}
and hence $\sin\theta = \cos\theta\tan\theta = \frac{2t}{1 + t^2}$.\par
\emph{Remark: Another method to solve this is to set up the situation in Section 1 Problem 10(a).}
\item \begin{enumerate}
\item $\frac{\cos(\alpha - \beta) + \cos(\alpha + \beta)}{2}$
\item $\frac{\cos(\alpha - \beta) - \cos(\alpha + \beta)}{2}$
\item $\frac{\sin(\alpha + \beta) + \sin(\alpha - \beta)}{2}$
\item $2\sin(\frac{\theta + \phi}{2})\sin(\frac{\theta - \phi}{2})$
\item $2\cos(\frac{\theta + \phi}{2})\cos(\frac{\theta - \phi}{2})$
\end{enumerate}
\item For all of the below, $n$ ranges over all integers.
\begin{enumerate}
\item $\alpha = \frac{\pi}{2} + 2\pi n$
\item By the double-angle identity $\cos(2t) = 2\cos^2 t - 1$,
\begin{equation*}
4\cos^2 t + 3 = 8\cos t\quad\iff\quad (2\cos t - 1)(2\cos t - 3) = 0.
\end{equation*}
Only $\cos t = 1/2$ is possible, and we get the solutions $t = \pi/3 + 2\pi n$ and $t = -\pi/3 + 2\pi n$.
\item The right hand side is $\cos(4\pi/9) = \sin(\pi/18)$, so $\sin\theta = \sin(\pi/18)$. Hence $\theta = \pi/18 + 2\pi n$ or $\theta = 17\pi/18 + 2\pi n$.
\item Dividing both sides by $\sqrt{3^2 + 5^2} = \sqrt{34}$ and setting $\varphi = \arccos(\frac{3}{\sqrt{34}})$ and $C = \frac{23}{4\sqrt{34}}$, we have $\sin x\cos\alpha + \cos x\sin\alpha = C$, so $\sin(x + \alpha) = C$. Hence
\begin{equation*}
x = \arcsin C - \alpha + 2\pi n\quad\text{or}\quad x = \pi - \arcsin C - \alpha + 2\pi n.
\end{equation*}
\end{enumerate}
\item \begin{enumerate}
\item $T_2(X) = 2X^2 - 1$\par
$T_3(X) = 4X^3 - 3X$\par
$T_4(X) = 8X^4 - 8X^2 + 1$
\item $T_0(X)$ has no roots\par 
$T_1(X)$ has one root, $0$\par
$T_2(X)$ has two roots, $\pm 1/\sqrt{2}$\par
$T_3(X)$ has three roots, $0$ and $\pm\sqrt{3}/2$\par
$T_4(X)$ has four roots: letting $Y = X^2$, then $8Y^2 - 8Y + 1 = 0$ when $Y = 4\pm 2\sqrt{2}$, which in turn gives us $X = \pm\sqrt{4\pm 2\sqrt{2}}$
\item It suffices to show that $\cos((n + 1)\theta) = 2\cos\theta\cos(n\theta) - \cos((n - 1)\theta)$, or equivalently,
\begin{equation*}
\cos((n + 1)\theta) + \cos((n - 1)\theta) = 2\cos\theta\cos(n\theta).
\end{equation*}
This follows by expanding $\cos(\alpha + \beta)$ and $\cos(\alpha - \beta)$ with $\alpha = n\theta$ and $\beta = \theta$, or (since the work was already done before) using by a sum-to-product or product-to-sum identity.
\end{enumerate}
\item Multiplying through by $\sin(\theta/2)$ and expanding the left hand side using product-to-sum,
\begin{align*}
&\sin\left(\frac{1}{2}\theta\right)\left[1 + 2\cos\theta + \cdots + 2\cos (n\theta)\right] \\
&= \sin\left(\frac{1}{2}\theta\right) + \left[\sin\left(\frac{3}{2}\theta\right) - \sin\left(\frac{1}{2}\theta\right)\right] + \cdots + \left[\sin\left(\left(n + \frac{1}{2}\right)\theta\right) - \sin\left(\left(n - \frac{1}{2}\right)\theta\right)\right] \\
&= \sin\left(\left(n + \frac{1}{2}\right)\theta\right).
\end{align*}
\item \begin{enumerate}
\item We compute
\begin{equation*}
\tan C = \tan(\pi - (A + B)) = -\tan(A + B) = \frac{\tan A + \tan B}{\tan A\tan B - 1}.
\end{equation*}
With this, both sides can be seen to simplify to $\frac{\tan A\tan B(\tan A + \tan B)}{\tan A\tan B - 1}$.
\item 
\end{enumerate}
\end{enumerate}
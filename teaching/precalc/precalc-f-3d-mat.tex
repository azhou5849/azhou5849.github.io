\section{Matrices in 3D}

\emph{Problems and solutions can be found at \url{https://azhou5849.github.io/teaching/}}

\subsection{Review Problems}

Throughout this section, let
\begin{align*}
\mathsf{A} &= \begin{pmatrix} 3 & 3 & 0 \\ 2 & -1 & 2 \\ 3 & 2 & 3 \end{pmatrix}, & \mathsf{B} &= \begin{pmatrix} 2 & 2 & -1 \\ -3 & 3 & -3 \\ -3 & -1 & 1 \end{pmatrix}, & \mathsf{U} &= \begin{pmatrix} -3 & -2 & 0 \\ 0 & -1 & -1 \\ 0 & 0 & 2 \end{pmatrix}, \\
\vec{u} &= \begin{pmatrix} 5 \\ 4 \\ 0 \end{pmatrix}, & \vec{v} &= \begin{pmatrix} 0 \\ -1 \\ 1 \end{pmatrix}, & \vec{w} &= \begin{pmatrix} -7 \\ -3 \\ 8 \end{pmatrix}.
\end{align*}

\begin{enumerate}
\item \emph{Matrix-vector calculations.}
\begin{enumerate}
\item Compute $\mathsf{A}\vec{u}$, $\mathsf{B}\vec{u}$, $\mathsf{U}\vec{u}$, $\mathsf{A}\vec{v}$, and $\mathsf{B}\vec{w}$.
\item Compute $2(\mathsf{A}\vec{u}) + 3(\mathsf{A}\vec{v})$ and $\mathsf{A}(2\vec{u} + 3\vec{v})$.
\item How does $\mathsf{B}\vec{w}$ relate to $\vec{w}$? Use this to compute $\mathsf{B}^5\vec{w}$.
\end{enumerate}
\item \emph{Matrix-matrix calculations.}
\begin{enumerate}
\item Compute $3\mathsf{A}$ and $\mathsf{B} - \mathsf{U}$.
\item Compute $(3\mathsf{A} + \mathsf{B} - \mathsf{U})\vec{u}$ and $3(\mathsf{A}\vec{u}) + \mathsf{B}\vec{u} - \mathsf{U}\vec{u}$.
\item Find all triples $(r,s,t)$ of real numbers such that $r\mathsf{A} + s\mathsf{B} + t\mathsf{U} = \mathsf{0}$.
\end{enumerate}
\item \emph{Matrix products.}
\begin{enumerate}
\item Compute $\mathsf{AB}$, $\mathsf{BA}$, and $\mathsf{AU}$.
\item Compute $\mathsf{AB} + \mathsf{AU}$ and $\mathsf{A}(\mathsf{B} + \mathsf{U})$.
\item Compute $(\mathsf{AB})\mathsf{U}$ and $\mathsf{A}(\mathsf{BU})$.
\end{enumerate}
\item \emph{Geometric transformations.}
\begin{enumerate}
\item Compute the matrix for scaling the $x$ and $y$ coordinates by $3/2$.
\item Compute the matrix for projecting onto $\vec{w}$.
\item Compute the matrix for reflecting across the plane $2x + 2y + 3z = 0$.
\item Compute the matrix for rotating around the $z$-axis with the property that $(-3,4,-12)$ is rotated to $(5,0,-12)$.
\item ($*$) Compute the matrix for rotating by an angle of $\pi$ around the axis passing through the origin and the point $(-3,4,-12)$.
\end{enumerate}
\newpage
\item \emph{Determinants.}
\begin{enumerate}
\item Compute $\det\mathsf{A}$, $\det\mathsf{B}$, and $\det\mathsf{U}$.
\item Compute $\det(\mathsf{AB})$.
\end{enumerate}
\item \emph{Inverses.}
\begin{enumerate}
\item Compute the inverses of $\mathsf{A}$ and $\mathsf{U}$.
\item Show that $\mathsf{B} - 4\mathsf{I}$ is not invertible.
\item Identify a non-zero vector $\vec{x}$ with the property that $(\mathsf{B} - 4\mathsf{I})\vec{x} = \vec{0}$.
\end{enumerate}
\item \emph{Cross products and matrices.} Given a matrix $\mathsf{M}$, let $[\mathsf{M}]_{ij}$ denote the entry in the $i$-th row and $j$-th column. Recall that the \textbf{transpose} of $\mathsf{M}$ is the matrix $\mathsf{M}^T$ with the property that $[\mathsf{M}^T]_{ij} = [\mathsf{M}]_{ji}$, i.e. the rows of $\mathsf{M}^T$ are the columns of $\mathsf{M}$ and vice-versa.
\begin{enumerate}
\item A square matrix $\mathsf{M}$ is called \textbf{symmetric} if $\mathsf{M}^T = \mathsf{M}$ and \textbf{skew-symmetric} if $\mathsf{M}^T = -\mathsf{M}$.\par
Show that every square matrix $\mathsf{M}$ can be written as the sum of a symmetric matrix and a skew-symmetric matrix.
\item Show that for every vector $\vec{a}$, there is a corresponding skew-symmetric matrix $\mathsf{R}_{\vec{a}}$ such that $\mathsf{R}_{\vec{a}}\vec{x} = \vec{a}\times\vec{x}$ for all vectors $\vec{x}$.
\item Conversely, show that for every skew-symmetric matrix $\mathsf{M}$, there is a corresponding vector $\vec{a}_{\mathsf{M}}$ for which $\mathsf{M}\vec{x} = \vec{a}_{\mathsf{M}}\times\vec{x}$ for all vectors $\vec{x}$.
\item Show that $\mathsf{R}_{\vec{a}\times\vec{b}} = \mathsf{R}_{\vec{a}}\mathsf{R}_{\vec{b}} - \mathsf{R}_{\vec{b}}\mathsf{R}_{\vec{a}}$.
\end{enumerate}
\end{enumerate}


\newpage
\subsection{Challenge Problems}

\begin{enumerate}\setcounter{enumi}{7}
\item % projection via matrices
\item % adjugate matrix
\item % cayley-hamilton
\end{enumerate}


\newpage
\subsection{Answers}
\section{Matrices in 3D}

\subsection{Review Problems}

Throughout this section, let
\begin{align*}
\mathsf{A} &= \begin{pmatrix} 3 & 3 & 0 \\ 2 & -1 & 2 \\ 3 & 2 & 3 \end{pmatrix}, & \mathsf{B} &= \begin{pmatrix} 2 & 2 & -1 \\ -3 & 3 & -3 \\ -3 & -1 & 1 \end{pmatrix}, & \mathsf{U} &= \begin{pmatrix} -3 & -2 & 0 \\ 0 & -1 & -1 \\ 0 & 0 & 2 \end{pmatrix}, \\
\vec{u} &= \begin{pmatrix} 5 \\ 4 \\ 0 \end{pmatrix}, & \vec{v} &= \begin{pmatrix} 0 \\ -1 \\ 1 \end{pmatrix}, & \vec{w} &= \begin{pmatrix} -7 \\ -3 \\ 8 \end{pmatrix}.
\end{align*}

\begin{enumerate}
\item \emph{Matrix-vector calculations.}
\begin{enumerate}
\item Compute $\mathsf{A}\vec{u}$, $\mathsf{B}\vec{u}$, $\mathsf{U}\vec{u}$, $\mathsf{A}\vec{v}$, and $\mathsf{B}\vec{w}$.
\item Compute $2(\mathsf{A}\vec{u}) + 3(\mathsf{A}\vec{v})$ and $\mathsf{A}(2\vec{u} + 3\vec{v})$.
\item How does $\mathsf{B}\vec{w}$ relate to $\vec{w}$? Use this to compute $\mathsf{B}^5\vec{w}$.
\end{enumerate}
\item \emph{Matrix-matrix calculations.}
\begin{enumerate}
\item Compute $3\mathsf{A}$ and $\mathsf{B} - \mathsf{U}$.
\item Compute $(3\mathsf{A} + \mathsf{B} - \mathsf{U})\vec{v}$ and $3(\mathsf{A}\vec{v}) + \mathsf{B}\vec{v} - \mathsf{U}\vec{v}$.
\item Find all triples $(r,s,t)$ of real numbers such that $r\mathsf{A} + s\mathsf{B} + t\mathsf{U} = \mathsf{0}$.
\end{enumerate}
\item \emph{Matrix products.}
\begin{enumerate}
\item Compute $\mathsf{AB}$, $\mathsf{BA}$, and $\mathsf{AU}$.
\item Compute $\mathsf{AB} + \mathsf{AU}$ and $\mathsf{A}(\mathsf{B} + \mathsf{U})$.
\item Compute $(\mathsf{AB})\mathsf{U}$ and $\mathsf{A}(\mathsf{BU})$.
\end{enumerate}
\item \emph{Geometric transformations.}
\begin{enumerate}
\item Compute the matrix for scaling the $x$ and $y$ coordinates by $3/2$.
\item Compute the matrix for projecting onto $\vec{w}$.
\item Compute the matrix for reflecting across the plane $2x + 2y + 3z = 0$.
\item Compute the matrix for rotating around the $z$-axis with the property that $(-3,4,-12)$ is rotated to $(0,5,-12)$.
\item (Calculator permitted, $*$) Compute the matrix for rotating by an angle of $\pi$ around the axis passing through the origin and the point $(-3,4,-12)$.
\end{enumerate}
\newpage
\item \emph{Determinants.}
\begin{enumerate}
\item Compute $\det\mathsf{A}$, $\det\mathsf{B}$, and $\det\mathsf{U}$.
\item Compute $\det(\mathsf{AB})$ and $(\det\mathsf{A})(\det\mathsf{B})$.
\end{enumerate}
\item \emph{Inverses.}
\begin{enumerate}
\item Compute the inverses of $\mathsf{A}$ and $\mathsf{U}$.
\item Show that $\mathsf{B} - 4\mathsf{I}$ is not invertible.
\item Identify a non-zero vector $\vec{x}$ with the property that $(\mathsf{B} - 4\mathsf{I})\vec{x} = \vec{0}$.
\end{enumerate}
\item \emph{Cross products and matrices.} Given a matrix $\mathsf{M}$, let $[\mathsf{M}]_{ij}$ denote the entry in the $i$-th row and $j$-th column. Recall that the \textbf{transpose} of $\mathsf{M}$ is the matrix $\mathsf{M}^T$ with the property that $[\mathsf{M}^T]_{ij} = [\mathsf{M}]_{ji}$, i.e. the rows of $\mathsf{M}^T$ are the columns of $\mathsf{M}$ and vice-versa. A square matrix $\mathsf{M}$ is called \textbf{symmetric} if $\mathsf{M}^T = \mathsf{M}$ and \textbf{skew-symmetric} if $\mathsf{M}^T = -\mathsf{M}$.
\begin{enumerate}
\item Show that for every vector $\vec{a}$, there is a corresponding skew-symmetric matrix $\mathsf{R}_{\vec{a}}$ such that $\mathsf{R}_{\vec{a}}\vec{x} = \vec{a}\times\vec{x}$ for all vectors $\vec{x}$.
\item Conversely, show that for every skew-symmetric matrix $\mathsf{M}$, there is a corresponding vector $\vec{a}_{\mathsf{M}}$ for which $\mathsf{M}\vec{x} = \vec{a}_{\mathsf{M}}\times\vec{x}$ for all vectors $\vec{x}$.
\item Show that $\mathsf{R}_{\vec{a}\times\vec{b}} = \mathsf{R}_{\vec{a}}\mathsf{R}_{\vec{b}} - \mathsf{R}_{\vec{b}}\mathsf{R}_{\vec{a}}$.
\item Independently of the previous parts, show that every square matrix $\mathsf{M}$ can be written as the sum of a symmetric matrix and a skew-symmetric matrix.
\end{enumerate}
\end{enumerate}


\newpage
\subsection{Challenge Problems}

The results in this section are stated for $\mathbb{R}^3$ and for $3\times 3$ matrices, but they generalise directly to any (finite) dimension.

\begin{enumerate}\setcounter{enumi}{7}
\item In general, if $\mathsf{A}$ is an $\ell\times m$ matrix and $\mathsf{B}$ is an $m\times n$ matrix, then $\mathsf{AB}$ is an $\ell\times n$ matrix. For $1\leq i\leq\ell$ and $1\leq j\leq n$, the entry of $\mathsf{AB}$ in row $i$ and column $j$ is found by taking the ``dot product'' of the $i$-th row of $\mathsf{A}$ with the $j$-th column of $\mathsf{B}$. This also applies to matrix-vector multiplication if we regard an $m$-component vector as an $m\times 1$ matrix.
\begin{enumerate}
\item Describe the matrices $\begin{pmatrix} 1 & 0 & 0 \\ 0 & 1 & 0 \end{pmatrix}$ and $\begin{pmatrix} 1 & 0 \\ 0 & 1 \\ 0 & 0 \end{pmatrix}$ geometrically.
\item Let $\vec{u}\in\mathbb{R}^3$ be a vector of norm 1. Show that projection onto the line generated by $\vec{u}$ is given by the matrix $\vec{uu}^T$.
\item Let $\vec{u},\vec{v}\in\mathbb{R}^3$ be vectors of norm 1 which are orthogonal to each other, and let $\mathsf{Q}$ be the matrix whose columns are $\vec{u}$ and $\vec{v}$. Show that projection onto the plane generated by $\vec{u}$ and $\vec{v}$ is given by the matrix $\mathsf{QQ}^T$.
\item More generally, let $\vec{u},\vec{v}\in\mathbb{R}^3$ be any two linearly independent vectors and let $\mathsf{A}$ be the matrix whose columns are $\vec{u}$ and $\vec{v}$. Derive a formula in terms of $\mathsf{A}$ for the matrix that represents projection onto the plane generated by $\vec{u}$ and $\vec{v}$.
\end{enumerate}
\item Let $\mathsf{A}$ be a $3\times 3$ matrix. For each pair of indices $1\leq i,j\leq 3$, the $(i,j)$-th \textbf{minor} $\mathsf{M}_{ij}$ of $\mathsf{A}$ is the determinant of the $2\times 2$ matrix formed by deleting the $i$-th row and $j$-th column of $\mathsf{A}$. The \textbf{adjugate matrix} of $\mathsf{A}$ is the $3\times 3$ matrix $\adj\mathsf{A}$ whose $(i,j)$-th entry is $(-1)^{i + j}M_{ji}$.
\begin{enumerate}
\item Show that $\mathsf{A}(\adj\mathsf{A}) = (\adj\mathsf{A})\mathsf{A} = (\det\mathsf{A})\mathsf{I}$.
\item Supposing $\det\mathsf{A}\neq 0$, express $\mathsf{A}^{-1}$ in terms of $\adj\mathsf{A}$ and $\det\mathsf{A}$.
\end{enumerate}
\item Fix a matrix $\mathsf{A} = \begin{pmatrix} a_{11} & a_{12} & a_{13} \\ a_{21} & a_{22} & a_{23} \\ a_{31} & a_{32} & a_{33} \end{pmatrix}$. Given any polynomial $f(X) = b_0 + b_1X + \cdots + b_nX^n$, we can define scalar multiplication of a vector by a polynomial according to the formula
\begin{equation*}
f(X)\vec{v} = (b_0\mathsf{I} + b_1\mathsf{A} + b_2\mathsf{A}^2 + \cdots + b_n\mathsf{A}^n)\vec{v}.
\end{equation*}
\begin{enumerate}
\item Show that
\begin{equation*}
\begin{pmatrix} a_{11} - X & a_{12} & a_{13} \\ a_{21} & a_{22} - X & a_{23} \\ a_{31} & a_{32} & a_{33} - X \end{pmatrix}\begin{pmatrix} \unit{i} \\ \unit{j} \\ \unit{k} \end{pmatrix} = \begin{pmatrix} \vec{0} \\ \vec{0} \\ \vec{0} \end{pmatrix}.
\end{equation*}
\item Show that $p_{\mathsf{A}}(X) = \det(\mathsf{A} - X\mathsf{I})$, the \textbf{characteristic polynomial} of $\mathsf{A}$, satisfies the relation $p_{\mathsf{A}}(X)\vec{v} = \vec{0}$ for all vectors $\vec{v}$.
\item Supposing $p_{\mathsf{A}}(X) = c_0 + c_1X + c_2X^2 + c_3X^3$, show that $c_0\mathsf{I} + c_1\mathsf{A} + c_2\mathsf{A}^2 + c_3\mathsf{A}^3 = \mathsf{0}$. This is the \textbf{Cayley-Hamilton theorem} ($3\times 3$ case), sometimes written as $p_{\mathsf{A}}(\mathsf{A}) = \mathsf{0}$.
\end{enumerate}
\end{enumerate}


\newpage
\subsection{Answers}

\begin{enumerate}
\item \begin{enumerate}
\item \begin{multicols}{2}
$\mathsf{A}\vec{u} = \begin{pmatrix} 27 \\ 6 \\ 23 \end{pmatrix}$\par
$\mathsf{B}\vec{u} = \begin{pmatrix} 18 \\ -3 \\ -19 \end{pmatrix}$\par
$\mathsf{U}\vec{u} = \begin{pmatrix} -23 \\ -4 \\ 0 \end{pmatrix}$\par
$\mathsf{A}\vec{v} = \begin{pmatrix} -3 \\ 3 \\ 1 \end{pmatrix}$\par
$\mathsf{B}\vec{w} = \begin{pmatrix} -28 \\ -12 \\ 32 \end{pmatrix}$
\end{multicols}
\item The two computations should give the same result, as
\begin{equation*}
\mathsf{A}(2\vec{u} + 3\vec{v}) = \mathsf{A}(2\vec{u}) + \mathsf{A}(3\vec{v}) = 2(\mathsf{A}\vec{u}) + 3(\mathsf{A}\vec{v}).
\end{equation*}
From what we computed in part (a),
\begin{equation*}
2(\mathsf{A}\vec{u}) + 3(\mathsf{A}\vec{v}) = 2\begin{pmatrix} 27 \\ 6 \\ 23 \end{pmatrix} + 3\begin{pmatrix} -3 \\ 3 \\ 1 \end{pmatrix} = \begin{pmatrix} 45 \\ 21 \\ 49 \end{pmatrix}.
\end{equation*}
\item We observe that $\mathsf{B}\vec{w} = 4\vec{w}$. Therefore,
\begin{equation*}
\mathsf{B}^5\vec{w} = 4^5\vec{w} = 1024\begin{pmatrix} -7 \\ -3 \\ 8 \end{pmatrix} = \begin{pmatrix} -7168 \\ -3072 \\ 8192 \end{pmatrix}.
\end{equation*}
\end{enumerate}
\item \begin{enumerate}
\item \begin{multicols}{2}
$3\mathsf{A} = \begin{pmatrix} 9 & 9 & 0 \\ 6 & -3 & 6 \\ 9 & 6 & 9 \end{pmatrix}$\par
$\mathsf{B} - \mathsf{U} = \begin{pmatrix} 5 & 4 & -1 \\ -3 & 4 & -2 \\ -3 & -1 & -1 \end{pmatrix}$
\end{multicols}
\item The two computations should give the same result, as
\begin{equation*}
(3\mathsf{A} + \mathsf{B} - \mathsf{U})\vec{v} = (3\mathsf{A})\vec{v} + \mathsf{B}\vec{v} - \mathsf{U}\vec{v} = 3(\mathsf{A}\vec{v}) + \mathsf{B}\vec{v} - \mathsf{U}\vec{v}.
\end{equation*}
From what we computed in part (a),
\begin{equation*}
3\mathsf{A} + \mathsf{B} - \mathsf{U} = \begin{pmatrix} 14 & 13 & -1 \\ 3 & 1 & 4 \\ 6 & 5 & 8 \end{pmatrix},
\end{equation*}
so then
\begin{equation*}
(3\mathsf{A} + \mathsf{B} - \mathsf{U})\vec{v} = \begin{pmatrix} 14 & 13 & -1 \\ 3 & 1 & 4 \\ 6 & 5 & 8 \end{pmatrix}\begin{pmatrix} 0 \\ -1 \\ 1 \end{pmatrix} = \begin{pmatrix} -14 \\ 3 \\ 3 \end{pmatrix}.
\end{equation*}
\item The upper right entry of $\mathsf{M} = r\mathsf{A} + s\mathsf{B} + t\mathsf{U} = \mathsf{0}$ is $-s$, so we need $s = 0$. Then, the lower left entry of $\mathsf{M} = r\mathsf{A} + t\mathsf{U} = \mathsf{0}$ is $3r$, so we need $r = 0$. Finally, $\mathsf{M} = t\mathsf{U} = \mathsf{0}$ forces $t = 0$, so the only solution is $(r,s,t) = (0,0,0)$.
\end{enumerate}
\newpage
\item \begin{enumerate}
\item \begin{multicols}{2}
$\mathsf{AB} = \begin{pmatrix} -3 & 15 & -12 \\ 1 & -1 & 3 \\ -9 & 9 & -6 \end{pmatrix}$\par
$\mathsf{BA} = \begin{pmatrix} 7 & 2 & 1 \\ -12 & -18 & -3 \\ -8 & -6 & 1 \end{pmatrix}$\par
$\mathsf{AU} = \begin{pmatrix} -9 & -9 & -3 \\ -6 & -3 & 5 \\ -9 & -8 & 4 \end{pmatrix}$
\end{multicols}
\item $\mathsf{A}(\mathsf{B} + \mathsf{U}) = \mathsf{AB} + \mathsf{AU} = \begin{pmatrix} -12 & 6 & -15 \\ -5 & -4 & 8 \\ -18 & 1 & -2 \end{pmatrix}$
\item $\mathsf{A}(\mathsf{BU}) = (\mathsf{AB})\mathsf{U} = \begin{pmatrix} -3 & 15 & -12 \\ 1 & -1 & 3 \\ -9 & 9 & -6 \end{pmatrix}\begin{pmatrix} -3 & -2 & 0 \\ 0 & -1 & -1 \\ 0 & 0 & 2 \end{pmatrix} = \begin{pmatrix} 9 & -9 & -39 \\ -3 & -1 & 7 \\ 27 & 9 & -21 \end{pmatrix}$
\end{enumerate}
\item \begin{enumerate}
\item $\begin{pmatrix} 3/2 & 0 & 0 \\ 0 & 3/2 & 0 \\ 0 & 0 & 1 \end{pmatrix}$
\item We compute
\begin{equation*}
\proj_{\vec{w}}(x,y,z) = \frac{(x,y,z)\cdot\vec{w}}{\|\vec{w}\|^2}\vec{w} = \frac{-7x - 3y + 8z}{122}\vec{w} = \frac{-7}{122}x\vec{w} + \frac{-3}{122}y\vec{w} + \frac{4}{61}z\vec{w}.
\end{equation*}
Therefore, the matrix of this projection is
\begin{equation*}
\begin{pmatrix} {} & {} & {} \\ \frac{-7}{122}\vec{w} & \frac{-3}{122}\vec{w} & \frac{4}{61}\vec{w} \\ {} & {} & {} \end{pmatrix} = \begin{pmatrix} 49/122 & 21/122 & -28/61 \\ 21/122 & 9/122 & -12/61 \\ -28/61 & -12/61 & 32/61 \end{pmatrix}.
\end{equation*}
\item If $\mathsf{R}\vec{x}$ is the reflection of vector $\vec{x}$ across the plane, then $\vec{x} - \mathsf{R}\vec{x} = 2\proj_{\vec{n}}(\vec{x})$, where $\vec{n} = (2,2,3)$ is a normal vector to the plane. We compute
\begin{equation*}
\proj_{\vec{n}}(x,y,z) = \frac{(x,y,z)\cdot\vec{n}}{\|\vec{n}\|^2}\vec{n} = \frac{2x + 2y + 3z}{17}\vec{n} = \frac{2}{17}x\vec{n} + \frac{2}{17}y\vec{n} + \frac{3}{17}z\vec{n},
\end{equation*}
so the matrix of the projection onto $\vec{n}$ is
\begin{equation*}
\mathsf{P}_{\vec{n}} = \begin{pmatrix} {} & {} & {} \\ \frac{2}{17}\vec{n} & \frac{2}{17}\vec{n} & \frac{3}{17}\vec{n} \\ {} & {} & {} \end{pmatrix} = \begin{pmatrix} 4/17 & 4/17 & 6/17 \\ 4/17 & 4/17 & 6/17 \\ 6/17 & 6/17 & 9/17 \end{pmatrix}.
\end{equation*}
The matrix of the reflection is then
\begin{equation*}
\mathsf{R} = \mathsf{I} - 2\mathsf{P}_{\vec{n}} = \begin{pmatrix} 9/17 & -8/17 & -12/17 \\ -8/17 & 9/17 & -12/17 \\ -12/17 & -12/17 & -1/17 \end{pmatrix}.
\end{equation*}
\emph{Remark:} We can also compute the matrix $\mathsf{P}$ for projection onto the plane, then compute $\mathsf{R} = 2\mathsf{P} - \mathsf{I}$ to get the reflection across the plane.
\newpage
\item Since this rotation is around the $z$-axis, it takes the form $\mathsf{A} = \begin{pmatrix} \cos\theta & -\sin\theta & 0 \\ \sin\theta & \cos\theta & 0 \\ 0 & 0 & 1 \end{pmatrix}$, where $\theta$ is chosen so that $(-3,4)$ rotates to $(0,5)$. We compute
\begin{equation*}
\begin{pmatrix} \cos\theta & -\sin\theta \\ \sin\theta & \cos\theta \end{pmatrix}\begin{pmatrix} -3 \\ 4 \end{pmatrix} = \begin{pmatrix} -3\cos\theta - 4\sin\theta \\ -3\sin\theta + 4\cos\theta \end{pmatrix},
\end{equation*}
so we require
\begin{equation*}
-3\cos\theta - 4\sin\theta = 0\quad\text{and}\quad -3\sin\theta + 4\cos\theta = 5.
\end{equation*}
Solving gives $\sin\theta = -3/5$ and $\cos\theta = 4/5$, so $\mathsf{A} = \begin{pmatrix} 4/5 & 3/5 & 0 \\ -3/5 & 4/5 & 0 \\ 0 & 0 & 1 \end{pmatrix}$.
\item We start by rotating the given axis to the $z$-axis. The rotation from part (d) sends the $(-3,4,-12)$-axis to the $(0,5,-12)$-axis, and then by similar reasoning to part (d), the rotation given by $\mathsf{B} = \begin{pmatrix} 1 & 0 & 0 \\ 0 & -12/13 & -5/13 \\ 0 & 5/13 & -12/13 \end{pmatrix}$ sends the $(0,5,-12)$-axis to the $z$-axis. Rotation by $\pi$ around the $z$-axis is given by the matrix $\mathsf{R} = \begin{pmatrix} -1 & 0 & 0 \\ 0 & -1 & 0 \\ 0 & 0 & 1 \end{pmatrix}$. Finally, we undo the transformation of the axis. Therefore, the overall rotation matrix is
\begin{equation*}
\mathsf{A}^{-1}\mathsf{B}^{-1}\mathsf{RBA} = \frac{1}{169}\begin{pmatrix} -151 & -24 & 72 \\ -24 & -137 & -96 \\ 72 & -96 & 119 \end{pmatrix}.
\end{equation*}
\end{enumerate}
\item \begin{enumerate}
\item For $\mathsf{A}$, expanding along the first row gives
\begin{equation*}
\det\mathsf{A} = 3\cdot\det\begin{pmatrix} -1 & 2 \\ 2 & 3 \end{pmatrix} - 3\cdot\begin{pmatrix} 2 & 2 \\ 3 & 3 \end{pmatrix} = 3\cdot [(-1)\cdot 3 - 2\cdot 2] - 3\cdot [2\cdot 3 - 2\cdot 3] = -21.
\end{equation*}
For $\mathsf{B}$, we can add the third row to the first row and add $3$ times the third row to the second row to produce the matrix $\mathsf{B}' = \begin{pmatrix} -1 & 1 & 0 \\ -12 & 0 & 0 \\ -3 & -1 & 1 \end{pmatrix}$ with $\det\mathsf{B}' = \det\mathsf{B}$. We can now expand down the third column to get
\begin{equation*}
\det\mathsf{B} = \det\mathsf{B}' = 1\cdot\begin{pmatrix} -1 & 1 \\ -12 & 0 \end{pmatrix} = 1\cdot [(-1)\cdot 0 - 1\cdot (-12)] = 12.
\end{equation*}
Alternatively, we can expand along the second row to get
\begin{equation*}
\det\mathsf{B} = \det\mathsf{B}' = -(-12)\cdot\begin{pmatrix} 1 & 0 \\ -1 & 1 \end{pmatrix} = 12.
\end{equation*}
Finally, $\mathsf{U}$ is upper triangular, so $\det\mathsf{U} = (-3)\cdot (-1)\cdot 2 = 6$.
\item $\det(\mathsf{AB}) = (\det\mathsf{A})(\det\mathsf{B}) = (-21)\cdot 12 = -252$
\end{enumerate}
\newpage
\item \begin{enumerate}
\item We proceed by \href{https://en.wikipedia.org/wiki/Gaussian_elimination#Finding_the_inverse_of_a_matrix}{row reduction}. For $\mathsf{A}$, one possible sequence of row operations is
\begin{align*}
\begin{pmatrix} 3 & 3 & 0 & \bigm| & 1 & 0 & 0 \\ 2 & -1 & 2 & \bigm| & 0 & 1 & 0 \\ 3 & 2 & 3 & \bigm| & 0 & 0 & 1 \end{pmatrix} &\rightarrow \begin{pmatrix} 1 & 1 & 0 & \bigm| & 1/3 & 0 & 0 \\ 2 & -1 & 2 & \bigm| & 0 & 1 & 0 \\ 3 & 2 & 3 & \bigm| & 0 & 0 & 1 \end{pmatrix} \\
&\rightarrow \begin{pmatrix} 1 & 1 & 0 & \bigm| & 1/3 & 0 & 0 \\ 0 & -3 & 2 & \bigm| & -2/3 & 1 & 0 \\ 0 & -1 & 3 & \bigm| & -1 & 0 & 1 \end{pmatrix} \\
&\rightarrow \begin{pmatrix} 1 & 1 & 0 & \bigm| & 1/3 & 0 & 0 \\ 0 & 1 & -3 & \bigm| & 1 & 0 & -1 \\ 0 & -3 & 2 & \bigm| & -2/3 & 1 & 0 \end{pmatrix} \\
&\rightarrow \begin{pmatrix} 1 & 1 & 0 & \bigm| & 1/3 & 0 & 0 \\ 0 & 1 & -3 & \bigm| & 1 & 0 & -1 \\ 0 & 0 & -7 & \bigm| & 7/3 & 1 & -3 \end{pmatrix} \\
&\rightarrow \begin{pmatrix} 1 & 1 & 0 & \bigm| & 1/3 & 0 & 0 \\ 0 & 1 & -3 & \bigm| & 1 & 0 & -1 \\ 0 & 0 & 1 & \bigm| & -1/3 & -1/7 & 3/7 \end{pmatrix} \\
&\rightarrow \begin{pmatrix} 1 & 1 & 0 & \bigm| & 1/3 & 0 & 0 \\ 0 & 1 & 0 & \bigm| & 0 & -3/7 & 2/7 \\ 0 & 0 & 1 & \bigm| & -1/3 & -1/7 & 3/7 \end{pmatrix} \\
&\rightarrow \begin{pmatrix} 1 & 0 & 0 & \bigm| & 1/3 & 3/7 & -2/7 \\ 0 & 1 & 0 & \bigm| & 0 & -3/7 & 2/7 \\ 0 & 0 & 1 & \bigm| & -1/3 & -1/7 & 3/7 \end{pmatrix},
\end{align*}
so $\mathsf{A}^{-1} = \begin{pmatrix} 1/3 & 3/7 & -2/7 \\ 0 & -3/7 & 2/7 \\ -1/3 & -1/7 & 3/7 \end{pmatrix}$. For $\mathsf{U}$, one possible sequence of row operations is
\begin{align*}
\begin{pmatrix} -3 & -2 & 0 & \bigm| & 1 & 0 & 0 \\ 0 & -1 & -1 & \bigm| & 0 & 1 & 0 \\ 0 & 0 & 2 & \bigm| & 0 & 0 & 1 \end{pmatrix} &\rightarrow \begin{pmatrix} 1 & 2/3 & 0 & \bigm| & -1/3 & 0 & 0 \\ 0 & 1 & 1 & \bigm| & 0 & -1 & 0 \\ 0 & 0 & 1 & \bigm| & 0 & 0 & 1/2 \end{pmatrix} \\
&\rightarrow \begin{pmatrix} 1 & 2/3 & 0 & \bigm| & -1/3 & 0 & 0 \\ 0 & 1 & 0 & \bigm| & 0 & -1 & -1/2 \\ 0 & 0 & 1 & \bigm| & 0 & 0 & 1/2 \end{pmatrix} \\
&\rightarrow \begin{pmatrix} 1 & 0 & 0 & \bigm| & -1/3 & 2/3 & 1/3 \\ 0 & 1 & 0 & \bigm| & 0 & -1 & -1/2 \\ 0 & 0 & 1 & \bigm| & 0 & 0 & 1/2 \end{pmatrix},
\end{align*}
so $\mathsf{U}^{-1} = \begin{pmatrix} -1/3 & 2/3 & 1/3 \\ 0 & -1 & -1/2 \\ 0 & 0 & 1/2 \end{pmatrix}$.
\newpage
\item We start by computing
\begin{equation*}
\mathsf{B} - 4\mathsf{I} = \begin{pmatrix} -2 & 2 & -1 \\ -3 & -1 & -3 \\ -3 & -1 & -3 \end{pmatrix}.
\end{equation*}
The last two rows are equal, so $\det(\mathsf{B} - 4\mathsf{I}) = 0$ and hence $\mathsf{B} - 4\mathsf{I}$ is not invertible.
\item In Problem 1, we observed that $\mathsf{B}\vec{w} = 4\vec{w}$, so $(\mathsf{B} - 4\mathsf{I})\vec{w} = \vec{0}$.
\end{enumerate}
\item \begin{enumerate}
\item If $\vec{a} = (a_1,a_2,a_3)$ and $\vec{x} = (x_1,x_2,x_3)$, then
\begin{equation*}
\vec{a}\times\vec{x} = \begin{pmatrix} a_2x_3 - a_3x_2 \\ a_3x_1 - a_1x_3 \\ a_1x_2 - a_2x_1 \end{pmatrix} = \begin{pmatrix} 0 & -a_3 & a_2 \\ a_3 & 0 & -a_1 \\ -a_2 & a_1 & 0\end{pmatrix}\begin{pmatrix} x_1 \\ x_2 \\ x_3 \end{pmatrix},
\end{equation*}
so we take $\mathsf{R}_{\vec{a}} = \begin{pmatrix} 0 & -a_3 & a_2 \\ a_3 & 0 & -a_1 \\ -a_2 & a_1 & 0\end{pmatrix}$.
\item If $\mathsf{M} = \begin{pmatrix} 0 & m_{12} & m_{13} \\ -m_{12} & 0 & m_{23} \\ -m_{13} & -m_{23} & 0 \end{pmatrix}$, then reversing the computations of part (a) shows that we can take $\vec{a}_{\mathsf{M}} = (-m_{23}, m_{13}, -m_{12})$.
\item We use Problem 5a from Week 31 extensions. For any $\vec{x}$,
\begin{align*}
(\mathsf{R}_{\vec{a}}\mathsf{R}_{\vec{b}} - \mathsf{R}_{\vec{b}}\mathsf{R}_{\vec{a}})\vec{x} &= \vec{a}\times(\vec{b}\times\vec{x}) - \vec{b}\times(\vec{a}\times\vec{x}) \\
&= [\vec{b}(\vec{a}\cdot\vec{x}) - \vec{x}(\vec{a}\cdot\vec{b})] - [\vec{a}(\vec{b}\cdot\vec{x}) - \vec{x}(\vec{a}\cdot\vec{b})] \\
&= \vec{b}(\vec{x}\cdot\vec{a}) - \vec{a}(\vec{x}\cdot\vec{b}) = \vec{x}\times(\vec{b}\times\vec{a}) \\
&= -(\vec{b}\times\vec{a})\times\vec{x} = (\vec{a}\times\vec{b})\times\vec{x} \\
&= \mathsf{R}_{\vec{a}\times\vec{b}}\vec{x}.
\end{align*}
Therefore, $\mathsf{R}_{\vec{a}\times\vec{b}} = \mathsf{R}_{\vec{a}}\mathsf{R}_{\vec{b}} - \mathsf{R}_{\vec{b}}\mathsf{R}_{\vec{a}}$.
\item We wish to find $\mathsf{A}$ symmetric and $\mathsf{B}$ skew-symmetric so that $\mathsf{M} = \mathsf{A} + \mathsf{B}$. Taking the transpose on both sides,
\begin{equation*}
\mathsf{M}^T = \mathsf{A}^T + \mathsf{B}^T = \mathsf{A} - \mathsf{B}.
\end{equation*}
Solving the system of equations
\begin{align*}
\mathsf{A} + \mathsf{B} &= \mathsf{M} \\
\mathsf{A} - \mathsf{B} &= \mathsf{M}^T
\end{align*}
gives us the required symmetric and skew-symmetric matrices, namely
\begin{equation*}
\mathsf{A} = \frac{1}{2}(\mathsf{M} + \mathsf{M}^T)\quad\text{and}\quad\mathsf{B} = \frac{1}{2}(\mathsf{M} - \mathsf{M}^T).
\end{equation*}
\end{enumerate}
\newpage 
\item \begin{enumerate}
\item The matrix $\begin{pmatrix} 1 & 0 & 0 \\ 0 & 1 & 0 \end{pmatrix}$ turns 3D vectors into 2D vectors by projecting them onto the $xy$-plane (i.e. throwing out the third component). The matrix $\begin{pmatrix} 1 & 0 \\ 0 & 1 \\ 0 & 0 \end{pmatrix}$ takes 2D vectors and turns them into 3D vectors by treating them as lying in the $xy$-plane (i.e. introducing a third component and setting it equal to 0).
\item Note that for any vectors $\vec{a}$ and $\vec{b}$, we have $\vec{a}\cdot\vec{b} = \vec{a}^T\vec{b} = \vec{b}^T\vec{a}$.\par
If $\vec{u}$ is a unit vector, $\proj_{\vec{u}}(\vec{x}) = (\vec{x}\cdot\vec{u})\vec{u} = \vec{u}(\vec{u}^T\vec{x}) = (\vec{uu}^T)\vec{x}$.
\item 
\end{enumerate}
\end{enumerate}
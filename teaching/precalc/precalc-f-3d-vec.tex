\section{Vectors in 3D}

\emph{Problems and solutions can be found at \url{https://azhou5849.github.io/teaching/}}

\subsection{Review Problems}

\begin{enumerate}
\item \emph{Operations.} Let
\begin{equation*}
\vec{a} = \begin{pmatrix} -2 \\ -1 \\ 2 \end{pmatrix},\quad\vec{b} = \begin{pmatrix} 3 \\ 0 \\ -4 \end{pmatrix},\quad\vec{c} = \begin{pmatrix} -1 \\ 1 \\ 5 \end{pmatrix}.
\end{equation*}
Compute each of the following. (Write ``Err'' or similar for any undefined expressions.)
\begin{multicols}{2}
\begin{enumerate}
\item $2\vec{a} + \vec{b} - \vec{c}$
\item $\|\vec{a}\| + \|\vec{b}\| - \|\vec{a} + \vec{b}\|$
\item $\vec{b}\cdot\vec{c}$
\item $\vec{a}\times\vec{b}$
\item $\vec{a}\cdot (\vec{b}\cdot\vec{c})$
\item $(\vec{a}\times\vec{b})\times\vec{c}$
\end{enumerate}
\end{multicols}
\item \emph{Distances and spheres.} 
\begin{enumerate}
\item Find the distance between the points $(2, -5, -2)$ and $(1, -5, 0)$.
\item Write down an equation for the sphere with center $(5, -1, 0)$ and radius 5.
\item Find the center and radius of the sphere with equation
\begin{equation*}
x^2 + y^2 + z^2 - 2x + 8y + 8z + 17 = 0.
\end{equation*}
\end{enumerate}
\item \emph{Angles.} Let $A = (-20,-2,1)$, $B = (-15,3,21)$, and $C = (-16,14,5)$. Compute $\angle BAC$.
\item \emph{Cross products.} Let $\vec{u} = \begin{pmatrix} 3 \\ 3 \\ -5 \end{pmatrix}$ and $\vec{v} = \begin{pmatrix} -3 \\ 2 \\ -5 \end{pmatrix}$.
\begin{enumerate}
\item Find all vectors orthogonal to both $\vec{u}$ and $\vec{v}$ with norm 1.
\item Find the area of the parallelogram with vertices at $\vec{0}, \vec{u}, \vec{v}, \vec{u} - \vec{v}$.
\item Let $\theta$ be the angle between $\vec{u}$ and $\vec{v}$. Compute $\sin\theta$.
\end{enumerate}
\item \emph{Planes.} Let $A = (4,-5,5)$, $B = (-2,5,-5)$, and $C = (3,-3,-3)$. Find an equation for the plane passing through $A$, $B$, and $C$
\begin{enumerate}
\item in parametric form;
\item in cartesian form $ax + by + cz = d$.
\end{enumerate}
Then find a parametric form for the intersection of this plane and the plane $x + 2y + 3z = 4$.
\newpage
\item \emph{Projections and reflections.}
\begin{enumerate}
\item What point on the line through $(-5,0,-2)$ and $(2,5,2)$ is closest to $(3,1,-4)$?
\item Find the reflection of the point $P = (4,-4,5)$ across the plane $-4x + 4y + 3z = 3$.
\item ($*$) Let $\mathcal{C}$ be the circle centered at $(0,0,1)$ of radius 1 lying in the plane $z = 1$ and let $\ell$ be the line passing through the origin and the point $(1,3,1)$. What is the shortest possible distance between a point on $\mathcal{C}$ and a point on $\ell$?
\end{enumerate}
\item \emph{Using cross products in 2D problems.}
\begin{enumerate}
\item Let $ABC$ be a triangle in the $xy$-plane with area 14. If the points $A,B,C$ are listed in clockwise order going around the triangle, what is $\overrightarrow{AB}\times\overline{AC}$?
\item ($*$) Let $ABCD$ be a convex quadrilateral and let points $P$ and $Q$ lie on segments $\overline{AB}$ and $\overline{CD}$ respectively so that $AP/AB = CQ/CD$. Let $R$ be the intersection of $\overline{AQ}$ and $\overline{PD}$ and let $S$ be the intersection of $\overline{BQ}$ and $\overline{PC}$. Show that
\begin{equation*}
[PSQR] = [ARD] + [BSC].
\end{equation*}
\end{enumerate}
\end{enumerate}


\newpage
\subsection{Challenge Problems}

\begin{enumerate}\setcounter{enumi}{7}
\item % projections onto planes in 4D
\item % covariance and correlation
\item % simple linear regression with intercept
\end{enumerate}


\newpage
\subsection{Answers}

\begin{enumerate}
\item \begin{enumerate}
\item $2\vec{a} + \vec{b} - \vec{c} = \begin{pmatrix} 2(-2) + 3 - (-1) \\ 2(-1) + 0 - 1 \\ 2(2) + (-4) - 5 \end{pmatrix} = \begin{pmatrix} 0 \\ -3 \\ -5 \end{pmatrix}$
\item $\|\vec{a}\| + \|\vec{b}\| - \|\vec{a} + \vec{b}\| = 3 + 5 - \left\|\begin{pmatrix} 1 \\ -1 \\ -2 \end{pmatrix}\right\| = 8 - \sqrt{6}$
\item $\vec{b}\cdot\vec{c} = 3\cdot (-1) + 0\cdot 1 + (-4)\cdot 5 = -23$
\item $\vec{a}\times\vec{b} = $
\item Err ($\vec{b}\cdot\vec{c}$ produces a real number, which cannot be dotted with $\vec{a}$)
\item 
\end{enumerate}
\end{enumerate}
\section{Vectors in 3D}

\emph{Problems and solutions can be found at \url{https://azhou5849.github.io/teaching/}}

\subsection{Review Problems}

\begin{enumerate}
\item \emph{Operations.} Let
\begin{equation*}
\vec{a} = \begin{pmatrix} -2 \\ -1 \\ 2 \end{pmatrix},\quad\vec{b} = \begin{pmatrix} 3 \\ 0 \\ -4 \end{pmatrix},\quad\vec{c} = \begin{pmatrix} -1 \\ 1 \\ 5 \end{pmatrix}.
\end{equation*}
Compute each of the following. (Write ``Err'' or similar for any undefined expressions.)
\begin{multicols}{2}
\begin{enumerate}
\item $2\vec{a} + \vec{b} - \vec{c}$
\item $\|\vec{a}\| + \|\vec{b}\| - \|\vec{a} + \vec{b}\|$
\item $\vec{b}\cdot\vec{c}$
\item $\vec{a}\times\vec{b}$
\item $\vec{a}\cdot (\vec{b}\cdot\vec{c})$
\item $(\vec{a}\times\vec{b})\times\vec{c}$
\end{enumerate}
\end{multicols}
\item \emph{Distances and spheres.} 
\begin{enumerate}
\item Find the distance between the points $(2, -5, -2)$ and $(1, -5, 0)$.
\item Write down an equation for the sphere with center $(5, -1, 0)$ and radius 5.
\item Find the center and radius of the sphere with equation
\begin{equation*}
x^2 + y^2 + z^2 - 2x + 8y + 8z + 17 = 0.
\end{equation*}
\end{enumerate}
\item \emph{Angles.} Let $A = (-20,-2,1)$, $B = (-15,3,21)$, and $C = (-16,14,5)$. Compute $\angle BAC$.
\item \emph{Cross products.} Let $\vec{u} = \begin{pmatrix} 3 \\ 3 \\ -5 \end{pmatrix}$ and $\vec{v} = \begin{pmatrix} -3 \\ 2 \\ -5 \end{pmatrix}$.
\begin{enumerate}
\item Find all vectors orthogonal to both $\vec{u}$ and $\vec{v}$ with norm 1.
\item Find the area of the parallelogram with vertices at $\vec{0}, \vec{u}, \vec{v}, \vec{u} - \vec{v}$.
\item Let $\theta$ be the angle between $\vec{u}$ and $\vec{v}$. Compute $\sin\theta$.
\end{enumerate}
\item \emph{Planes.} Let $A = (4,-5,5)$, $B = (-2,5,-5)$, and $C = (3,-3,-3)$. Find an equation for the plane passing through $A$, $B$, and $C$
\begin{enumerate}
\item in parametric form;
\item in cartesian form $ax + by + cz = d$.
\end{enumerate}
Then find a parametric form for the intersection of this plane and the plane $x + 2y + 3z = 4$.
\newpage
\item \emph{Projections and reflections.}
\begin{enumerate}
\item What point on the line through $(-5,0,-2)$ and $(2,5,2)$ is closest to $(3,1,-4)$?
\item Find the reflection of the point $P = (4,-4,5)$ across the plane $-4x + 4y + 3z = 3$.
\item ($*$) Let $\mathcal{C}$ be the circle centered at $(0,0,1)$ of radius 1 lying in the plane $z = 1$ and let $\mathcal{P}$ be the plane passing through the origin as well as the points $(5,1,1)$ and $(1,3,1)$. What is the shortest possible distance between a point on $\mathcal{C}$ and a point on $\mathcal{P}$?
\end{enumerate}
\item \emph{Using cross products in 2D problems.}
\begin{enumerate}
\item Let $ABC$ be a triangle in the $xy$-plane with area 14. If the points $A,B,C$ are listed in clockwise order going around the triangle, what is $\overrightarrow{AB}\times\overline{AC}$?
\item ($*$) Let $ABCD$ be a convex quadrilateral and let points $P$ and $Q$ lie on segments $\overline{AB}$ and $\overline{CD}$ respectively so that $AP/AB = CQ/CD$. Let $R$ be the intersection of $\overline{AQ}$ and $\overline{PD}$ and let $S$ be the intersection of $\overline{BQ}$ and $\overline{PC}$. Show that
\begin{equation*}
[PSQR] = [ARD] + [BCS].
\end{equation*}
\end{enumerate}
\end{enumerate}


\newpage
\subsection{Challenge Problems}

Suppose we sample $n$ members of a population and measure quantities $X$ and $Y$ for each of the $n$ observations. (For example, perhaps $X$ and $Y$ denote height and wingspan that we measure for several people.) The observed values of $X$ and $Y$ are stored in vectors
\begin{equation*}
\vec{x} = \begin{pmatrix} x_1 \\ x_2 \\ \vdots \\ x_n \end{pmatrix}\quad\text{and}\quad\vec{y} = \begin{pmatrix} y_1 \\ y_2 \\ \vdots \\ y_n \end{pmatrix}.
\end{equation*}

\begin{enumerate}\setcounter{enumi}{7}
\item In statistics, often more useful than the ordinary dot product is a rescaled version,
\begin{equation*}
\langle\vec{x},\vec{y}\rangle = \frac{1}{n}(\vec{x}\cdot\vec{y}) = \frac{x_1y_1 + x_2y_2 + \cdots + x_ny_n}{n}.
\end{equation*}
\begin{enumerate}
\item Let $\vec{1}$ (or $\vec{1}_n$) denote the vector with components that are all equal to 1. Express the sample mean $\bar{x}$ of the observed values of $X$ in terms of $\vec{x}$, $\vec{1}$, and $\langle\phantom{x},\phantom{x}\rangle$.
\item Let $\mathcal{H}$ be the ``hyperplane'' of all points in $n$-dimensional space with the property that the sum of the coordinates is 0. Show that the projection of $\vec{x}$ onto $\mathcal{H}$ is $\vec{x} - \bar{x}\vec{1}$.
\item The \emph{(uncorrected) sample variance} of the observed values of $X$ is
\begin{equation*}
s_x^2 = \langle\vec{x} - \bar{x}\vec{1}, \vec{x} - \bar{x}\vec{1}\rangle,
\end{equation*}
while the \emph{sample standard deviation} $s_x$ is the square root of the sample variance.\par
Show that $s_x^2 = \langle\vec{x},\vec{x}\rangle - (\bar{x})^2$.
\end{enumerate}
\item The \emph{sample covariance} of the observed values of $X$ and $Y$ is
\begin{equation*}
s_{xy} = \langle\vec{x} - \bar{x}\vec{1}, \vec{y} - \bar{y}\vec{1}\rangle = \langle\vec{x},\vec{y}\rangle - \bar{x}\cdot\bar{y},
\end{equation*}
and the \emph{sample correlation} is $r_{xy} = \frac{s_{xy}}{s_x\cdot s_y}$ (when $s_x,s_y\neq 0$).
\begin{enumerate}
\item Show that $-1\leq r\leq 1$.
\item When does $r = 1$? When does $r = -1$?
\item Given three quantities $A,B,C$, is it possible that $r_{ab} > 0$, $r_{bc} > 0$, but $r_{ac} < 0$?
\end{enumerate}
\item In \emph{simple linear regression}, we seek values $\beta_0,\beta_1$ so that the linear model $Y = \beta_0 + \beta_1X$ is ``best possible.'' This is usually taken to mean that the \emph{mean squared error}
\begin{equation*}
MSE = \langle\vec{y} - \unit{y}, \vec{y} - \unit{y}\rangle
\end{equation*}
should be as small as possible, where $\unit{y} = \beta_0\vec{1} + \beta_1\vec{x}$. Show, using projection or otherwise, that this is achieved when
\begin{equation*}
\beta_1 = r_{xy}\cdot\frac{s_y}{s_x}\quad\text{and}\quad\beta_0 = \bar{y} - \beta_1\bar{x}.
\end{equation*}
\end{enumerate}


\newpage
\subsection{Answers}

\begin{enumerate}
\item \begin{enumerate}
\item $2\vec{a} + \vec{b} - \vec{c} = \begin{pmatrix} 2(-2) + 3 - (-1) \\ 2(-1) + 0 - 1 \\ 2(2) + (-4) - 5 \end{pmatrix} = \begin{pmatrix} 0 \\ -3 \\ -5 \end{pmatrix}$
\item $\|\vec{a}\| + \|\vec{b}\| - \|\vec{a} + \vec{b}\| = 3 + 5 - \left\|\begin{pmatrix} 1 \\ -1 \\ -2 \end{pmatrix}\right\| = 8 - \sqrt{6}$
\item $\vec{b}\cdot\vec{c} = 3\cdot (-1) + 0\cdot 1 + (-4)\cdot 5 = -23$
\item $\vec{a}\times\vec{b} = \begin{pmatrix} (-1)\cdot (-4) - 2\cdot 0 \\ 2\cdot 3 - (-2)\cdot (-4) \\ (-2)\cdot 0 - (-1)\cdot 3 \end{pmatrix} = \begin{pmatrix} 4 \\ -2 \\ 3 \end{pmatrix}$
\item Err ($\vec{b}\cdot\vec{c}$ produces a real number, which cannot be dotted with $\vec{a}$)
\item $(\vec{a}\times\vec{b})\times\vec{c} = \begin{pmatrix} 4 \\ -2 \\ 3 \end{pmatrix}\times\begin{pmatrix} -1 \\ 1 \\ 5 \end{pmatrix} = \begin{pmatrix} (-2)\cdot 5 - 3\cdot 1 \\ 3\cdot (-1) - 4\cdot 5 \\ 4\cdot 1 - (-2)\cdot (-1) \end{pmatrix} = \begin{pmatrix} -13 \\ -23 \\ 2 \end{pmatrix}$
\end{enumerate}
\item \begin{enumerate}
\item $\sqrt{(2 - 1)^2 + ((-5) - (-5))^2 - ((-2) - 0)^2} = \sqrt{5}$
\item $(x - 5)^2 + (y + 1)^2 + z^2 = 25$
\item We complete the square:
\begin{align*}
x^2 + y^2 + z^2 - 2x + 8y + 8z + 17 &= 0, \\
(x^2 - 2x) + (y^2 + 8y) + (z^2 + 8z) &= -17, \\
(x^2 - 2x + {\color{red}1}) + (y^2 + 8y + {\color{green}16}) + (z^2 + 8z + {\color{blue}16}) &= -17 + {\color{red}1} + {\color{green}16} + {\color{blue}16} \\
(x - 1)^2 + (y + 4)^2 + (z + 4)^2 &= 16.
\end{align*}
This is a sphere with center $(1,-4,-4)$ and radius $\sqrt{16} = 4$.
\end{enumerate}
\item Let $\vec{u} = \overrightarrow{AB}$ and $\vec{v} = \overrightarrow{AC}$, so that $\theta = \angle BAC$ is the angle between $\vec{u}$ and $\vec{v}$. We compute
\begin{align*}
\vec{u} &= \begin{pmatrix} (-15) - (-20) \\ 3 - (-2) \\ 21 - 1 \end{pmatrix} = \begin{pmatrix} 5 \\ 5 \\ 20 \end{pmatrix}, \\
\vec{v} &= \begin{pmatrix} (-16) - (-20) \\ 14 - (-2) \\ 5 - 1 \end{pmatrix} = \begin{pmatrix} 4 \\ 16 \\ 4 \end{pmatrix},
\end{align*}
so then
\begin{align*}
\cos\theta &= \frac{\vec{u}\cdot\vec{v}}{\|\vec{u}\|\|\vec{v}\|} = \frac{5\cdot 4 + 5\cdot 16 + 20\cdot 4}{\sqrt{5^2 + 5^2 + 20^2}\cdot\sqrt{4^2 + 16^2 + 4^2}} \\
&= \frac{180}{5\sqrt{1^2 + 1^2 + 4^2}\cdot 4\sqrt{1^2 + 4^2 + 1^2}} = \frac{9}{18} = \frac{1}{2}.
\end{align*}
This means that $\theta = \pi/3 = 60^{\circ}$.
\newpage
\item \begin{enumerate}
\item Any vector orthogonal to both $\vec{u}$ and $\vec{v}$ must be parallel to
\begin{equation*}
\vec{n} = \vec{u}\times\vec{v} = \begin{pmatrix} 3\cdot (-5) - (-5)\cdot 2 \\ (-5)\cdot (-3) - 3\cdot (-5) \\ 3\cdot 2 - 3\cdot (-3) \end{pmatrix} = \begin{pmatrix} -5 \\ 30 \\ 15 \end{pmatrix}.
\end{equation*}
The two vectors parallel to $\vec{n}$ of length 1 are
\begin{equation*}
\frac{\pm 1}{\|\vec{n}\|}\vec{n} = \frac{\pm 1}{\sqrt{(-5)^2 + 30^2 + 15^2}}\begin{pmatrix} -5 \\ 30 \\ 15 \end{pmatrix} = \frac{\pm 1}{5\sqrt{46}}\begin{pmatrix} -5 \\ 30 \\ 15 \end{pmatrix} = \frac{\pm 1}{\sqrt{46}}\begin{pmatrix} -1 \\ 6 \\ 3 \end{pmatrix}.
\end{equation*}
\item Since $\vec{u} = \vec{v} + (\vec{u} - \vec{v})$, this parallelogram is the one defined by $\vec{v}$ and $\vec{u} - \vec{v}$. Its area is
\begin{equation*}
\|\vec{v}\times (\vec{u} - \vec{v})\| = \|\vec{v}\times\vec{u} - \vec{v}\times\vec{v}\| = \|\vec{v}\times\vec{u}\| = \|-\vec{n}\| = 5\sqrt{46}.
\end{equation*}
\item We compute
\begin{equation*}
\sin\theta = \frac{\|\vec{u}\times\vec{v}\|}{\|\vec{u}\|\|\vec{v}\|} = \frac{5\sqrt{46}}{\sqrt{3^2 + 3^2 + (-5)^2}\cdot\sqrt{(-3)^2 + 2^2 + (-5)^2}} = \frac{5\sqrt{23}}{\sqrt{817}}.
\end{equation*}
\end{enumerate}
\item \begin{enumerate}
\item If $P = (x,y,z)$ is an arbitrary point in the plane, then there exist $s$ and $t$ for which
\begin{equation*}
\overrightarrow{P} = \overrightarrow{A} + s(\overrightarrow{AB}) + t(\overrightarrow{AC}) = \begin{pmatrix} 4 \\ -5 \\ 5 \end{pmatrix} + s\begin{pmatrix} -6 \\ 10 \\ -10 \end{pmatrix} + t\begin{pmatrix} -1 \\ 2 \\ -8 \end{pmatrix}.
\end{equation*}
\item A normal vector to the plane is given by
\begin{equation*}
\vec{n} = \overrightarrow{AB}\times\overrightarrow{AC} = \begin{pmatrix} 10\cdot (-8) - (-10)\cdot 2 \\ (-10)\cdot (-1) - (-6)\cdot (-8) \\ (-6)\cdot 2 - 10\cdot (-1) \end{pmatrix} = \begin{pmatrix} -60 \\ -38 \\ -2 \end{pmatrix}.
\end{equation*}
Therefore, an equation for the plane is
\begin{equation*}
0 = \vec{n}\cdot (\overrightarrow{P} - \overrightarrow{A}) = -60(x - 4) - 38(y + 5) - 2(z - 5),
\end{equation*}
which can be rearranged to $30x + 19y + z = 30$.
\end{enumerate}
To find the intersection of this plane with $x + 2y + 3z = 4$, we eliminate $x$ to get
\begin{align*}
30(x + 2y + 3z) - (30x + 19y + z) &= 30\cdot 4 - 30, \\
41y + 89z &= 90.
\end{align*}
If $z = t$, then $\displaystyle y = \frac{90}{41} - \frac{89}{41}t$ and
\begin{equation*}
x = 4 - 2y - 3z =  4 - \left(\frac{90}{41} - \frac{89}{41}t\right) - 3t = -\frac{16}{41} + \frac{55}{41}t.
\end{equation*}
In vector parametric form,
\begin{equation*}
\overrightarrow{P} = \begin{pmatrix} -16/41 \\ 90/41 \\ 0 \end{pmatrix} + t\begin{pmatrix} 55/41 \\ -89/41 \\ 1 \end{pmatrix}.
\end{equation*}
\newpage
\item \begin{enumerate}
\item Translating $(-5,0,-2)$ to the origin, the problem is equivalent to finding the point on the line generated by $\vec{u} = \begin{pmatrix} 7 \\ 5 \\ 4 \end{pmatrix}$ closest to $\vec{v} = \begin{pmatrix} 8 \\ 1 \\ -2 \end{pmatrix}$. This is
\begin{equation*}
\proj_{\vec{u}}(\vec{v}) = \left(\frac{\vec{v}\cdot\vec{u}}{\vec{u}\cdot\vec{u}}\right)\vec{u} = \frac{8\cdot 7 + 1\cdot 5 + (-2)\cdot 4}{7\cdot 7 + 5\cdot 5 + 4\cdot 4}\begin{pmatrix} 7 \\ 5 \\ 4 \end{pmatrix} = \frac{53}{90}\begin{pmatrix} 7 \\ 5 \\ 4 \end{pmatrix}.
\end{equation*}
Translating back, the desired point is $\displaystyle\left(\frac{53}{90}\cdot 7 - 5, \frac{53}{90}\cdot 5, \frac{53}{90}\cdot 4 - 2\right) = \left(\frac{-79}{90}, \frac{53}{18}, \frac{16}{45}\right)$.
\item Let $Q$ be the desired reflection. Since $\overrightarrow{PQ}$ is normal to the plane, we can write
\begin{equation*}
\overrightarrow{Q} = \overrightarrow{P} + t\begin{pmatrix} -4 \\ 4 \\ 3 \end{pmatrix} = \begin{pmatrix} 4 - 4t \\ -4 + 4t \\ 5 + 3t \end{pmatrix}
\end{equation*}
for some value of $t$. The midpoint of $\overline{PQ}$ lies on the plane, so
\begin{equation*}
(-4)\cdot\frac{4 + (4 - 4t)}{2} + 4\cdot\frac{-4 + (-4 + 4t)}{2} + 3\cdot\frac{5 + (5 + 3t)}{2} = 3.
\end{equation*}
Solving this equation yields $t = 40/41$ and $\displaystyle Q = \left(\frac{4}{41}, \frac{-4}{41}, \frac{325}{41}\right)$.
\item A parameterization for $\mathcal{C}$ is given by $\vec{v}(\theta) = \begin{pmatrix} \cos\theta \\ \sin\theta\\ 1 \end{pmatrix}$. The shortest possible distance between an arbitrary point on $\mathcal{C}$ and a point on $\mathcal{P}$ can be found by projecting $\vec{v}(\theta)$ onto a normal vector for $\mathcal{P}$. One such normal vector is
\begin{equation*}
\vec{n} = \begin{pmatrix} 5 \\ 1 \\ 1 \end{pmatrix}\times\begin{pmatrix} 1 \\ 3 \\ 1 \end{pmatrix} = \begin{pmatrix} -2 \\ -4 \\ 14 \end{pmatrix},
\end{equation*}
so then the distance between $\vec{v}(\theta)$ and the plane $\mathcal{P}$ is
\begin{equation*}
\|\proj_{\vec{n}}(\vec{v}(\theta))\| = \frac{\lvert\vec{v}(\theta)\cdot\vec{n}\rvert}{\|\vec{n}\|} = \frac{\lvert -2\cos\theta - 4\sin\theta + 14\rvert}{6\sqrt{6}}.
\end{equation*}
We can write
\begin{equation*}
2\cos\theta + 4\sin\theta = 2\sqrt{5}\left(\frac{1}{\sqrt{5}}\cos\theta + \frac{2}{\sqrt{5}}\sin\theta\right) = 2\sqrt{5}\sin(\phi + \theta),
\end{equation*}
where $\sin\phi = 1/\sqrt{5}$ and $\cos\phi = 2/\sqrt{5}$. Therefore,
\begin{equation*}
\|\proj_{\vec{n}}(\vec{v}(\theta))\| = \frac{\lvert 14 - 2\sqrt{5}\sin(\phi + \theta)\rvert}{6\sqrt{6}}\geq\boxed{\frac{14 - 2\sqrt{5}}{6\sqrt{6}}},
\end{equation*}
with equality when $\sin(\phi + \theta) = 1$.
\end{enumerate}
\newpage
\item \begin{enumerate}
\item Since $ABC$ lies in the $xy$-plane, the cross product is parallel to $\unit{k}$. By the right-hand rule, $\overrightarrow{AB}\times\overrightarrow{AC}$ points downward since $A,B,C$ go around the triangle in clockwise order. The norm of $\overrightarrow{AB}\times\overrightarrow{AC}$ is twice the area of triangle $ABC$. Therefore, $\overrightarrow{AB}\times\overrightarrow{AC} = (0, 0, -28)$.
\item Without loss of generality, suppose that the vertices of $ABCD$ are in counterclockwise order and that the quadrilateral lies in the $xy$-plane. Then the $z$-coordinate of the vector
\begin{equation*}
\overrightarrow{AP}\times\overrightarrow{AD} + \overrightarrow{QB}\times\overrightarrow{QA} + \overrightarrow{BC}\times\overrightarrow{BP} \tag{$*$}
\end{equation*}
is precisely $2([ARD] - [PSQR] + [BSC])$, so it suffices to show that $(*)$ is $\vec{0}$. Let $\vec{a} = \overrightarrow{A}$, $\vec{b} = \overrightarrow{B}$, etc., so then $(*)$ becomes
\begin{equation*}
(\vec{p} - \vec{a})\times (\vec{d} - \vec{a}) + (\vec{b} - \vec{q})\times (\vec{a} - \vec{q}) + (\vec{c} - \vec{b})\times (\vec{p} - \vec{b}).
\end{equation*}
Let $r = AP/AB = CQ/CD$. Then
\begin{equation*}
\overrightarrow{P} = (1 - r)\vec{a} + r\vec{b}\quad\text{and}\quad\overrightarrow{Q} = (1 - r)\vec{c} + r\vec{d},
\end{equation*}
so
\begin{align*}
(\vec{p} - \vec{a})\times (\vec{d} - \vec{a}) &= r(\vec{b} - \vec{a})\times (\vec{d} - \vec{a}) \\
&= r(\vec{b} - \vec{a})\times\vec{d} - r\vec{b}\times\vec{a} \\
(\vec{b} - \vec{q})\times (\vec{a} - \vec{q}) &= \vec{b}\times\vec{a} - \vec{b}\times\vec{q} - \vec{q}\times\vec{a} \\
&= \vec{b}\times\vec{a} + \vec{q}\times (\vec{b} - \vec{a}) \\
&= \vec{b}\times\vec{a} + ((1 - r)\vec{c} + r\vec{d})\times (\vec{b} - \vec{a}) \\
&= \vec{b}\times\vec{a} + (1 - r)\vec{c}\times (\vec{b} - \vec{a}) - r(\vec{b} - \vec{a})\times\vec{d} \\
(\vec{c} - \vec{b})\times (\vec{p} - \vec{b}) &= (\vec{c} - \vec{b})\times (1 - r)(\vec{a} - \vec{b}) \\
&= -(1 - r)\vec{c}\times (\vec{b} - \vec{a}) - (1 - r)\vec{b}\times\vec{a}.
\end{align*}
Adding these up gives $\vec{0}$, as required.
\end{enumerate}
\end{enumerate}
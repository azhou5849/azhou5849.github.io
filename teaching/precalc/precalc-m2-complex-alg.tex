\section{Complex Numbers I: Algebra}

Throughout, $\mathbb{R}$ denotes the set of all real numbers and $\mathbb{C}$ denotes the set of all complex numbers.

\subsection{Review problems}

\begin{enumerate}
\item \emph{Arithmetic.} Let $z = -3 + 3i$ and $w = -4 - 2i$. Compute each of the following:
\begin{enumerate}
\item $\Re z$
\item $\Im w$
\item $z + w$
\item $z - w$
\item $zw$
\item $z/w$
\end{enumerate}
\item \emph{A quadratic with real coefficients.} Find all complex solutions to the equation $z^2 + 5 = 4z$.
\item \emph{Real-valued products.} Find all complex numbers $z$ for which $(-4 + 2i)z$ is real.
\item \emph{Powers of $i$ and periodic sequences.}
\begin{enumerate}
\item Show that $i^4 = 1$.
\item Find all complex solutions to the equation $z^4 = 1$ and write each one as a power of $i$.
\item Let $z_1, z_2, z_3, \ldots$ be a 4-periodic sequence of complex numbers, meaning that $z_{n + 4} = z_n$ for all positive integers $n$. Show that there exist complex numbers $a,b,c,d$ such that
\begin{equation*}
z_n = a + b\cdot i^n + c\cdot i^{2n} + d\cdot i^{3n}
\end{equation*}
for all $n$.
\end{enumerate}
\item \emph{Complex conjugation.} Given a complex number $z = x + yi$, the \textbf{complex conjugate} of $z$ is defined to be $\bar{z} = x - yi$.
\begin{enumerate}
\item Compute $\overline{943 - 319i}$.
\item Prove the following properties of complex conjugation:
\begin{enumerate}
\item $\bar{(\bar{z})} = z$ for all complex numbers $z$.
\item $\bar{z + w} = \bar{z} + \bar{w}$ for all complex numbers $z$ and $w$.
\item $\bar{z\cdot w} = \bar{z}\cdot\bar{w}$ for all complex numbers $z$ and $w$.
\item $\bar{1/z} = 1/\bar{z}$ for all complex numbers $z\neq 0$.
\item $\Re z = (z + \bar{z})/2$
\item $\Im z = (z - \bar{z})/2i$
\end{enumerate}
\emph{Remark:} From these, we can show that $\bar{z - w} = \bar{z} - \bar{w}$ for all complex numbers $z$ and $w$, that $\bar{z/w} = \bar{z}/\bar{w}$ for all complex numbers $z$ and $w\neq 0$, and that $\bar{z^n} = \bar{z}^n$ for all complex numbers $z$ and for all integers $n$ (with $z\neq 0$ when $n\leq 0$).
\end{enumerate}
\item \emph{Magnitude.} Given a complex number $z = x + yi$, the \textbf{magnitude} or \textbf{absolute value} of $z$ is defined to be $\lvert z\rvert = \sqrt{x^2 + y^2}$.
\begin{enumerate}
\item Compute $\lvert 21 + 20i\rvert$.
\item Prove the following properties of the magnitude:
\begin{enumerate}
\item $\lvert z\rvert^2 = z\cdot\bar{z}$ for all complex numbers $z$.
\item $\lvert zw\rvert = \lvert z\rvert\cdot\lvert w\rvert$ for all complex numbers $z$ and $w$.
\item $\lvert 1/z\rvert = 1/\lvert z\rvert$ for all complex numbers $z\neq 0$.
\end{enumerate}
\emph{Remark:} From these, it follows that $\lvert z/w\rvert = \lvert z\rvert / \lvert w\rvert$ for all complex numbers $z$ and $w\neq 0$, and that $\lvert z^n\rvert = \lvert z\rvert^n$ for all integers $n$ (with $z\neq 0$ when $n\leq 0$).
\end{enumerate}
\item \emph{Square roots of complex numbers.}
\begin{enumerate}
\item Find a complex number $w$ for which $w^2 = -16 + 30i$.
\item Find the two complex numbers $z$ satisfying $2z^2 - (8 + 4i)z + (14 - 7i) = 0$.
\item Prove that for every complex number $z$, there is a complex number $w$ for which $w^2 = z$.\par
\emph{Remark:} It follows from this that every quadratic polynomial with complex coefficients has complex roots (with roots given by the familiar quadratic formula).
\end{enumerate}
\end{enumerate}




%\item \begin{enumerate}
%\item Let $\ell_1$ be the line through $a = -4 - 3i$ and $b = 4 + i$, and let $\ell_2$ be the line through $c = -4i$ and $d = -3 + 2i$.
%\begin{enumerate}
%\item By considering slopes, or otherwise, show that $\ell_1$ and $\ell_2$ are perpendicular.
%\item Compute $\frac{d - c}{b - a}$.
%\end{enumerate}
%\item Show that in general, the line through $p\neq q$ is perpendicular to the line through $r\neq s$ if and only if $\frac{r - s}{p - q}$ is purely imaginary.
%\item Given two distinct complex numbers $a$ and $b$, the \emph{perpendicular bisector} of the line segment connecting $a$ and $b$ is the line perpendicular to this segment passing through the midpoint $m = \frac{a + b}{2}$.\par
%Show that $z$ lies on the perpendicular bisector of the line segment connecting $a$ and $b$ if and only if $\lvert z - a\rvert = \lvert z - b\rvert$.\par
%\emph{Hint:} Consider squared magnitudes and use the fact that a complex number $\alpha$ is purely imaginary if and only if $\alpha = -\bar{\alpha}$.
%\end{enumerate}
%\item If $\ell$ is a line in the complex plane, then \emph{reflection across $\ell$} is the function $f_{\ell}:\mathbb{C}\to\mathbb{C}$ defined the property that for any complex number $z$, line $\ell$ is the perpendicular bisector of the line segment connecting $z$ and $f_{\ell}(z)$. (When $z$ already lies on $\ell$, then we define $f(z) = z$.)
%\begin{enumerate}
%\item What complex number operation is equivalent to reflection across the $x$-axis?
%\item Let $\ell$ be the line passing through $0$ and $4 + 2i$. Find the reflection of $-5$ across $\ell$.
%\item More generally, let $\ell$ be the line passing through $0$ and $d$, where $d$ is a non-zero complex number. Find the reflection of $z$ across $\ell$, i.e. determine the function $f_{\ell}(z)$.
%\item Even more generally, let $\ell$ be the line passing through $a$ and $b$, where $a$ and $b$ are two distinct complex numbers. Find the reflection of $z$ across $\ell$.
%\item From the previous two parts, every reflection has the form $f(z) = \alpha\bar{z} + \beta$ where $\lvert\alpha\rvert = 1$. Conversely, show that any such function is a reflection composed with a translation where the translation is parallel to the line of reflection. (When the translation is non-zero, we call the overall transformation a \emph{glide reflection}.)
%\end{enumerate}

%\item Identify each of the following complex numbers.
%\begin{enumerate}
%\item The complex number corresponding to the point $(-5,-1)$.
%\item The two complex numbers of magnitude 2 whose real and imaginary parts are equal.
%\item The three complex numbers $z$ for which $0$, $3 - 2i$, $5 + 2i$, and $z$ are the vertices of a parallelogram (in some order).
%\end{enumerate}

    %\item An \emph{isometry} of the complex plane is a function $f:\mathbb{C}\to\mathbb{C}$ satisfying
%\begin{equation*}
%\lvert f(z) - f(w)\rvert = \lvert z - w\rvert
%\end{equation*}
%for all complex numbers $z$ and $w$. In other words, $f$ preserves distances between points.
%\begin{enumerate}
%\item Show that every translation and every reflection is an isometry.
%\item Let $f$ be an isometry satisfying $f(0) = 0$, $f(1) = 1$, and $f(i) = i$. Show that $f$ must be the identity map, i.e. $f(z) = z$ for all $z$.
%\item Prove that every isometry can be written as a composition of at most three reflections.
%\item Show that the composition of three reflections is either a reflection or a glide reflection.
%\end{enumerate}


\subsection{Challenge problems}

\begin{enumerate}\setcounter{enumi}{7}
\item \begin{enumerate}
\item Show that $\Re z\leq\lvert z\rvert$ for all complex numbers $z$. When does equality occur?
\item \emph{Triangle inequality.} Using part (a), or otherwise, show that
\begin{equation*}
\lvert z + w\rvert\leq\lvert z\rvert + \lvert w\rvert
\end{equation*}
for all complex numbers $z$ and $w$. When does equality occur?
\end{enumerate}
\item In this problem, we work through one formal construction of the complex numbers.\par 
Let $\mathcal{C}$ be the set of all ordered pairs of real numbers, and define operations $\oplus$ and $\otimes$ on $\mathcal{C}$ by
\begin{align*}
(a,b)\oplus (c,d) &= (a + c, b + d), \\
(a,b)\otimes (c,d) &= (ac - bd, ad + bc).
\end{align*}
We call $\oplus$ and $\otimes$ the addition and multiplication on $\mathcal{C}$, respectively.
\begin{enumerate}
\item Let $u = (-2,-4)$, $v = (-3,1)$, and $w = (0,4)$. Verify each of the following:
\begin{align*}
u\oplus (v\oplus w) &= (u\oplus v)\oplus w, \\
u\otimes (v\otimes w) &= (u\otimes v)\otimes w, \\
u\oplus v &= v\oplus u, \\
u\otimes v &= v\otimes u, \\
u\otimes (v\oplus w) &= (u\otimes v)\oplus (u\otimes w), \\
u\oplus (0,0) &= u, \\
v\otimes (1,0) &= v.
\end{align*}
Also find pairs $a_v$ and $m_u$ for which $v\oplus a_v = (0,0)$ and $m_u\otimes u = (1,0)$.
\end{enumerate}
One can show that these equalities hold in general and that for any $z\in\mathcal{C}$, we can find a unique $a_z\in\mathcal{C}$ for which $z\oplus a_z = (0,0)$. This $a_z$ is the \textbf{additive inverse} of $z$ and is denoted $-z$. Similarly, for any non-zero $z\in\mathcal{C}$, we can find a unique $m_z\in\mathcal{C}$ for which $z\otimes m_z = (1,0)$. This $m_z$ is the \textbf{multiplicative inverse} of $z$ and is denoted $z^{-1}$.\par
These properties, collectively called the ``field axioms,'' are enough to derive all of the usual algebraic facts that we are familiar with in the context of real number algebra. What remains is to check that $\mathcal{C}$ ``does what we expect the complex numbers to do.''
\begin{enumerate}\setcounter{enumii}{1}
\item Prove that for any two real numbers $x$ and $y$,
\begin{equation*}
(x,0)\oplus (y,0) = (x + y,0)\quad\text{and}\quad (x,0)\otimes (y,0) = (xy,0).
\end{equation*}
\end{enumerate}
This shows that the elements $(r,0)$ for $r\in\mathbb{R}$, with operations $\oplus$ and $\otimes$, ``act like'' the real numbers with the usual addition and multiplication operations $+$ and $\times$. As such, we can regard $\mathbb{R}$ as being contained within $\mathcal{C}$ by identifying $r\in\mathbb{R}$ with $(r,0)\in\mathcal{C}$, and then $\otimes$ and $\otimes$ extend $+$ and $\times$ from $\mathbb{R}$ to all of $\mathcal{C}$. As such, when $r$ is a real number we simply write $r$ instead of $(r,0)$. Moreover, from now on, we write $+$ and $\times$ (or $\cdot$) instead of $\oplus$ and $\otimes$. We also introduce the subtraction and division operations as $z - w = z + (-w)$ and $z/w = z\cdot w^{-1}$.
\begin{enumerate}\setcounter{enumii}{2}
\item Show that $(0,1)\times (0,1) = -1$ and $(0,-1)\times (0,-1) = -1$.
\end{enumerate}
This shows that $\mathcal{C}$ has square roots of $-1$, as expected. We can now recover the usual notation by defining $i = (0,1)$ and then observing that $(x,y) = x + y\cdot i$. Henceforth, we can forget about the underlying ordered pairs and replace $\mathcal{C}$ with the usual $\mathbb{C}$.
\item A function $f:\mathbb{C}\to\mathbb{C}$ is an \emph{$\mathbb{R}$-automorphism of $\mathbb{C}$} if
\begin{equation*}
f(z + w) = f(z) + f(w)\quad\text{and}\quad f(zw) = f(z)\cdot f(w)
\end{equation*}
for all $z,w\in\mathbb{C}$ and $f(r) = r$ for all $r\in\mathbb{R}$.
\begin{enumerate}
\item Show that if $f:\mathbb{C}\to\mathbb{C}$ is an $\mathbb{R}$-automorphism of $\mathbb{C}$, then $f(i) = i$ or $f(i) = -i$.
\item Show that the only two $\mathbb{R}$-automorphisms of $\mathbb{C}$ are the identity function $f(z) = z$ and the conjugation function $f(z) = \bar{z}$.
\end{enumerate}
\end{enumerate}


\newpage
\subsection{Answers}

\begin{enumerate}
\item 
\end{enumerate}
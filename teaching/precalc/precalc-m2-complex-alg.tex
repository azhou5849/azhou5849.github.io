\section{Complex Numbers I: Algebra}

Throughout, $\mathbb{R}$ denotes the set of all real numbers and $\mathbb{C}$ denotes the set of all complex numbers.

\subsection{Review problems}

\begin{enumerate}
\item \emph{Arithmetic.} Let $z = -3 + 3i$ and $w = -4 - 2i$. Compute each of the following:
\begin{enumerate}
\item $\Re z$
\item $\Im w$
\item $z + w$
\item $z - w$
\item $zw$
\item $z/w$
\end{enumerate}
\item \emph{A quadratic with real coefficients.} Find all complex solutions to the equation $z^2 + 5 = 4z$.
\item \emph{Real-valued products.} Find all complex numbers $z$ for which $(-4 + 2i)z$ is real.
\item \emph{Powers of $i$ and periodic sequences.}
\begin{enumerate}
\item Show that $i^4 = 1$.
\item Find all complex solutions to the equation $z^4 = 1$ and write each one as a power of $i$.
\item Let $z_1, z_2, z_3, \ldots$ be a 4-periodic sequence of complex numbers, meaning that $z_{n + 4} = z_n$ for all positive integers $n$. Show that there exist complex numbers $a,b,c,d$ such that
\begin{equation*}
z_n = a + b\cdot i^n + c\cdot i^{2n} + d\cdot i^{3n}
\end{equation*}
for all $n$.
\end{enumerate}
\item \emph{Complex conjugation.} Given a complex number $z = x + yi$, the \textbf{complex conjugate} of $z$ is defined to be $\bar{z} = x - yi$.
\begin{enumerate}
\item Compute $\overline{943 - 319i}$.
\item Prove the following properties of complex conjugation:
\begin{enumerate}
\item $\bar{(\bar{z})} = z$ for all complex numbers $z$.
\item $\bar{z + w} = \bar{z} + \bar{w}$ for all complex numbers $z$ and $w$.
\item $\bar{z\cdot w} = \bar{z}\cdot\bar{w}$ for all complex numbers $z$ and $w$.
\item $\Re z = (z + \bar{z})/2$
\item $\Im z = (z - \bar{z})/2i$
\end{enumerate}
\emph{Remark:} From these, we can show that $\bar{z - w} = \bar{z} - \bar{w}$ for all complex numbers $z$ and $w$, that $\bar{z/w} = \bar{z}/\bar{w}$ for all complex numbers $z$ and $w\neq 0$, and that $\bar{z^n} = \bar{z}^n$ for all complex numbers $z$ and for all integers $n$ (with $z\neq 0$ when $n\leq 0$).
\end{enumerate}
\item \emph{Magnitude.} Given a complex number $z = x + yi$, the \textbf{magnitude} or \textbf{absolute value} of $z$ is defined to be $\lvert z\rvert = \sqrt{x^2 + y^2}$.
\begin{enumerate}
\item Compute $\lvert 21 + 20i\rvert$.
\item Prove the following properties of the magnitude:
\begin{enumerate}
\item $\lvert z\rvert^2 = z\cdot\bar{z}$ for all complex numbers $z$.
\item $\lvert zw\rvert = \lvert z\rvert\cdot\lvert w\rvert$ for all complex numbers $z$ and $w$.
\end{enumerate}
\emph{Remark:} From these, it follows that $\lvert z/w\rvert = \lvert z\rvert / \lvert w\rvert$ for all complex numbers $z$ and $w\neq 0$, and that $\lvert z^n\rvert = \lvert z\rvert^n$ for all integers $n$ (with $z\neq 0$ when $n\leq 0$).
\end{enumerate}
\item \emph{Square roots of complex numbers.}
\begin{enumerate}
\item Find a complex number $w$ for which $w^2 = -16 + 30i$.
\item Find the two complex numbers $z$ satisfying $2z^2 - (8 + 4i)z + (14 - 7i) = 0$.
\item Prove that for every complex number $z$, there is a complex number $w$ for which $w^2 = z$.\par
\emph{Remark:} It follows from this that every quadratic polynomial with complex coefficients has complex roots (with roots given by the familiar quadratic formula).
\end{enumerate}
\end{enumerate}



\subsection{Challenge problems}

\begin{enumerate}\setcounter{enumi}{7}
\item \begin{enumerate}
\item Show that $\Re z\leq\lvert z\rvert$ for all complex numbers $z$. When does equality occur?
\item \emph{Triangle inequality.} Using part (a), or otherwise, show that
\begin{equation*}
\lvert z + w\rvert\leq\lvert z\rvert + \lvert w\rvert
\end{equation*}
for all complex numbers $z$ and $w$. When does equality occur?
\end{enumerate}
\item In this problem, we work through one formal construction of the complex numbers.\par 
Let $\mathcal{C}$ be the set of all ordered pairs of real numbers, and define operations $\oplus$ and $\otimes$ on $\mathcal{C}$ by
\begin{align*}
(a,b)\oplus (c,d) &= (a + c, b + d), \\
(a,b)\otimes (c,d) &= (ac - bd, ad + bc).
\end{align*}
We call $\oplus$ and $\otimes$ the addition and multiplication on $\mathcal{C}$, respectively.
\begin{enumerate}
\item Let $u = (-2,-4)$, $v = (-3,1)$, and $w = (0,4)$. Verify each of the following:
\begin{align*}
u\oplus (v\oplus w) &= (u\oplus v)\oplus w, \\
u\otimes (v\otimes w) &= (u\otimes v)\otimes w, \\
u\oplus v &= v\oplus u, \\
u\otimes v &= v\otimes u, \\
u\otimes (v\oplus w) &= (u\otimes v)\oplus (u\otimes w), \\
u\oplus (0,0) &= u, \\
v\otimes (1,0) &= v.
\end{align*}
Also find pairs $a_v$ and $m_u$ for which $v\oplus a_v = (0,0)$ and $m_u\otimes u = (1,0)$.
\end{enumerate}
One can show that these equalities hold in general and that for any $z\in\mathcal{C}$, we can find a unique $a_z\in\mathcal{C}$ for which $z\oplus a_z = (0,0)$. This $a_z$ is the \textbf{additive inverse} of $z$ and is denoted $-z$. Similarly, for any non-zero $z\in\mathcal{C}$, we can find a unique $m_z\in\mathcal{C}$ for which $z\otimes m_z = (1,0)$. This $m_z$ is the \textbf{multiplicative inverse} of $z$ and is denoted $z^{-1}$.\par
These properties, collectively called the ``field axioms,'' are enough to derive all of the usual algebraic facts that we are familiar with in the context of real number algebra. What remains is to check that $\mathcal{C}$ ``does what we expect the complex numbers to do.''
\begin{enumerate}\setcounter{enumii}{1}
\item Prove that for any two real numbers $x$ and $y$,
\begin{equation*}
(x,0)\oplus (y,0) = (x + y,0)\quad\text{and}\quad (x,0)\otimes (y,0) = (xy,0).
\end{equation*}
\end{enumerate}
This shows that the elements $(r,0)$ for $r\in\mathbb{R}$, with operations $\oplus$ and $\otimes$, ``act like'' the real numbers with the usual addition and multiplication operations $+$ and $\times$. As such, we can regard $\mathbb{R}$ as being contained within $\mathcal{C}$ by identifying $r\in\mathbb{R}$ with $(r,0)\in\mathcal{C}$, and then $\otimes$ and $\otimes$ extend $+$ and $\times$ from $\mathbb{R}$ to all of $\mathcal{C}$. As such, when $r$ is a real number we simply write $r$ instead of $(r,0)$. Moreover, from now on, we write $+$ and $\times$ (or $\cdot$) instead of $\oplus$ and $\otimes$. We also introduce the subtraction and division operations as $z - w = z + (-w)$ and $z/w = z\cdot w^{-1}$.
\begin{enumerate}\setcounter{enumii}{2}
\item Show that $(0,1)\times (0,1) = -1$ and $(0,-1)\times (0,-1) = -1$.
\end{enumerate}
This shows that $\mathcal{C}$ has square roots of $-1$, as expected. We can now recover the usual notation by defining $i = (0,1)$ and then observing that $(x,y) = x + y\cdot i$. Henceforth, we can forget about the underlying ordered pairs and replace $\mathcal{C}$ with the usual $\mathbb{C}$.
\item A function $f:\mathbb{C}\to\mathbb{C}$ is an \emph{$\mathbb{R}$-automorphism of $\mathbb{C}$} if
\begin{equation*}
f(z + w) = f(z) + f(w)\quad\text{and}\quad f(zw) = f(z)\cdot f(w)
\end{equation*}
for all $z,w\in\mathbb{C}$ and $f(r) = r$ for all $r\in\mathbb{R}$.
\begin{enumerate}
\item Show that if $f:\mathbb{C}\to\mathbb{C}$ is an $\mathbb{R}$-automorphism of $\mathbb{C}$, then $f(i) = i$ or $f(i) = -i$.
\item Show that the only two $\mathbb{R}$-automorphisms of $\mathbb{C}$ are the identity function $f(z) = z$ and the conjugation function $f(z) = \bar{z}$.
\end{enumerate}
\end{enumerate}


\newpage
\subsection{Answers}

\begin{enumerate}
\item \begin{enumerate}
\item $-3$
\item $-2$
\item $-7 + i$
\item $1 + 5i$
\item $18 - 6i$
\item $\frac{3}{10} - \frac{9}{10}i$
\end{enumerate}
\item $2 + i$ and $2 - i$
\item Let $z = a + bi$. Then $(-4 + 2i)(a + bi) = (-4a - 2b) + (2a - 4b)i$. For this product to be real, we need $2a - 4b = 0$, so $a = 2b$. Hence $z = 2b + bi = b(2 + i)$, so $\boxed{\text{any real multiple of }2 + i}$ would do the job.
\item \begin{enumerate}
\item We have $i^2 = -1$, so then $i^3 = -i$, so then $i^4 = -i^2 = -(-1) = 1$.
\item $1 = i^0$, $i = i^1$, $-1 = i^2$, and $-i = i^3$.
\item The right hand side is also 4-periodic, so it suffices to show that there exist $a,b,c,d$ with
\begin{equation*}
z_n = a + b\cdot i^n + c\cdot i^{2n} + d\cdot i^{3n}
\end{equation*}
for $n = 1, 2, 3, 4$. This gives us the system of linear equations
\begin{align*}
a + ib - c - id &= z_1, \\
a - b + c - d &= z_2, \\
a - ib - c + id &= z_3, \\
a + b + c + d &= z_4.
\end{align*}
This system does indeed have a solution, namely
\begin{align*}
a &= \frac{z_1 + z_2 + z_3 + z_4}{4}, \\
b &= \frac{-iz_1 - z_2 + iz_3 + z_4}{4}, \\
c &= \frac{-z_1 + z_2 - z_3 + z_4}{4}, \\
d &= \frac{iz_1 - z_2 - iz_3 + z_4}{4}.
\end{align*}
\end{enumerate}
\item \begin{enumerate}
\item $943 + 319i$
\item Let $z = x + yi$ and $w = a + bi$ throughout.
\begin{enumerate}
\item $\bar{(\bar{z})} = \bar{x - yi} = x + yi = z$
\item $\bar{z + w} = \bar{(x + a) + (y + b)i} = (x + a) - (y + b)i$\par
$\bar{z} + \bar{w} = (x - yi) + (a - bi) = (x + a) - (y + b)i$
\item $\bar{z\cdot w} = \bar{(xa - yb) + (xb + ya)i} = (xa - yb) - (xb + ya)i$\par
$\bar{z}\cdot\bar{w} = (x - yi)(a - bi) = (xa - yb) - (xb + ya)i$
\item $\frac{z + \bar{z}}{2} = \frac{(x + yi) + (x - yi)}{2} = x = \Re z$
\item $\frac{z - \bar{z}}{2i} = \frac{(x + yi) - (x - yi)}{2i} = y = \Im z$
\end{enumerate}
\end{enumerate}
\item \begin{enumerate}
\item 29
\item Let $z = x + yi$ and $w = a + bi$ throughout.
\begin{enumerate}
\item $z\cdot\bar{z} = (x + yi)(x - yi) = x^2 - (yi)^2 = x^2 + y^2 = \lvert z\rvert^2$
\item $\lvert zw\rvert = \sqrt{\lvert zw\rvert^2} = \sqrt{zw\cdot\bar{zw}} = \sqrt{z\bar{z}\cdot w\bar{w}} = \sqrt{\lvert z\rvert^2\cdot\lvert w\rvert^2} = \lvert z\rvert\cdot\lvert w\rvert$
\end{enumerate}
\end{enumerate}
\item \begin{enumerate}
\item Let $w = a + bi$, so then $w^2 = (a^2 - b^2) + 2abi$. Therefore, we need $a^2 - b^2 = -16$ and $ab = 15$. Substituting $b = 15/a$,
\begin{equation*}
a^2 - \frac{225}{a^2} = -16\iff a^4 + 16a^2 - 225 = 0.
\end{equation*}
Since $a$ is real, $a^2 > 0$, so the only viable solution to the quadratic in $a^2$ is
\begin{equation*}
a^2 = \frac{-16 + \sqrt{16^2 - 4(1)(-225)}}{2} = \frac{-16 + \sqrt{1156}}{2} = \frac{-16 + 34}{2} = 9.
\end{equation*}
Therefore, $a = 3$, in which case $b = 5$, or $a = -3$, in which case $b = -5$. Hence the two square roots of $-16 + 30i$ are $\pm (3 + 5i)$.
\item By the quadratic formula,
\begin{align*}
z &= \frac{(8 + 4i)\pm\sqrt{(8 + 4i)^2 - 4(2)(14 - 7i)}}{2(2)} \\
&= \frac{(8 + 4i)\pm\sqrt{(64 + 64i - 16) - (112 - 56i)}}{4} \\
&= \frac{(8 + 4i)\pm\sqrt{-64 + 120i}}{4} \\
&= \frac{(8 + 4i)\pm 2\sqrt{-16 + 30i}}{4} \\
&= \frac{(4 + 2i)\pm (3 + 5i)}{2} \\
&= \frac{7 + 7i}{2}\text{ or }\frac{1 - 3i}{2}.
\end{align*}
\item If $z = x + yi$ and $w = a + bi$ satisfies $w^2 = z$, then following the same procedure as in part (a) yields, when $y\neq 0$,
\begin{equation*}
a^4 - xa^2 - \frac{y^2}{4} = 0.
\end{equation*}
The product of the roots of the quadratic $T^2 - xT - y^2/4$ is $-y^2/4 < 0$, so there is a positive root $\alpha$ and a negative root $\beta$. Taking $a^2 = \alpha$, so then $a = \sqrt{\alpha}$ and $b = y/2a$, gives us a square root of $z$.\par 
In the case $y = 0$, either $a = 0$ or $b = 0$. If $x\geq 0$, then take $a = \sqrt{x}$ and $b = 0$, and if $x < 0$, then take $a = 0$ and $b = \sqrt{-x}$.
\end{enumerate}
\item \begin{enumerate}
\item For the inequality,
\begin{equation*}
\lvert z\rvert = \sqrt{(\Re z)^2 + (\Im z)^2}\geq\sqrt{(\Re z)^2} = \lvert\Re z\rvert\geq\Re z.
\end{equation*}
For the first inequality step, we have equality if and only if $\Im z = 0$. For the second inequality step, we have equality if and only if $\Re z\geq 0$. Putting these together, equality holds in the overall inequality if and only if $z$ is a non-negative real number.
\item As both sides are non-negative, it suffices to prove the squared inequality. We have
\begin{align*}
\lvert z + w\rvert^2 &= (z + w)\cdot\bar{(z + w)} \\
&= z\bar{z} + z\bar{w} + w\bar{z} + w\bar{w} \\
&= \lvert z\rvert^2 + 2\Re(z\bar{w}) + \lvert w\rvert^2 \\
&\leq\lvert z\rvert^2 + 2\lvert z\rvert\cdot\lvert w\rvert + \lvert w\rvert^2 \\
&= (\lvert z\rvert + \lvert w\rvert)^2,
\end{align*}
as required. Equality holds when we have equality in $\Re (z\bar{w}) = \lvert z\bar{w}\rvert$, and by part (a), we know that this requires $z\bar{w}\geq 0$. If $w = 0$, this condition holds. Otherwise, $w\bar{w} > 0$ and $z\bar{w}\geq 0$, so $z/w\geq 0$. That is, $z = \lambda w$ for a non-negative real number $\lambda$.
\end{enumerate}
\item \begin{enumerate}
\item $u\oplus (v\oplus w) = (u\oplus v)\oplus w = (-5,1)$\par
$u\otimes (v\otimes w) = (u\otimes v)\otimes w = (-40,40)$\par
$u\oplus v = v\oplus u = (-5,-3)$\par
$u\otimes v = v\otimes u = (10,10)$\par
$u\otimes (v\oplus w) = (u\otimes v)\oplus (u\otimes w) = (26,2)$\par
$a_v = (3,-1)$\par
$m_u = (-1/10, /5)$
\item $(x,0)\oplus (y,0) = (x + y, 0 + 0) = (x + y, 0)$\par
$(x,0)\otimes (y,0) = (x\cdot y - 0\cdot 0, x\cdot 0 + 0\cdot y) = (xy,0)$
\item $(0,1)\times (0,1) = (0\cdot 0 - 1\cdot 1, 0\cdot 1 + 1\cdot 0) = (-1,0) = -1$\par
$(0,-1)\times (0,-1) = (0\cdot 0 - (-1)\cdot (-1), 0\cdot (-1) + (-1)\cdot 0) = (-1,0) = -1$
\end{enumerate}
\item \begin{enumerate}
\item We have
\begin{equation*}
f(i)^2 = f(i)\cdot f(i) = f(i\cdot i) = f(-1) = -1,
\end{equation*}
so either $f(i) = i$ or $f(i) = -i$.
\item Let $z = x + yi$. Then
\begin{equation*}
f(z) = f(x + yi) = f(x) + f(yi) = x + f(y)\cdot f(i) = x + y\cdot f(i).
\end{equation*}
If $f(i) = i$, then $f(z) = x + yi = z$. If $f(i) = -i$, then $f(z) = x - yi = \bar{z}$.
\end{enumerate}
\end{enumerate}
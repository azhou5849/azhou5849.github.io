\section{Matrices in 2D}

\subsection{Review Problems}

\emph{Review problems are meant to cover ``standard'' definitions and calculations as well as the use of important results.}

Throughout, $\unit{i} = \begin{pmatrix} 1 \\ 0 \end{pmatrix}$ and $\unit{j} = \begin{pmatrix} 0 \\ 1 \end{pmatrix}$ are the standard unit vectors while $\vec{0} = \begin{pmatrix} 0 \\ 0 \end{pmatrix}$ is the zero vector. We also let $\mathsf{I} = \begin{pmatrix} 1 & 0 \\ 0 & 1 \end{pmatrix}$ be the ($2\times 2$) identity matrix and $\mathsf{0} = \begin{pmatrix} 0 & 0 \\ 0 & 0 \end{pmatrix}$ be the zero matrix.

\begin{enumerate}
\item \emph{Vector calculations.} Let $\vec{u} = \begin{pmatrix} 2 \\ 3 \end{pmatrix}$ and $\vec{v} = \begin{pmatrix} 4 \\ -1 \end{pmatrix}$. Compute each of the following.
\begin{enumerate}
\item $\vec{u} + \vec{v}$
\item $2\vec{v}$
\item $\vec{u}\cdot\vec{v}$ and $\vec{v}\cdot\vec{u}$
\item $\|\vec{u}\|$, $\|\vec{v}\|$, and $\|\vec{u} + \vec{v}\|$
\item The angle between $\vec{u}$ and $\vec{v}$
\item $\proj_{\vec{v}}(\vec{u})$ and $\proj_{\vec{u}}(\vec{v})$
\end{enumerate}
\item \emph{Applying matrices to vectors.} Let $\mathsf{A} = \begin{pmatrix} 2 & 4 \\ 1 & 1 \end{pmatrix}$ and $\vec{v} = \begin{pmatrix} 5 \\ 2 \end{pmatrix}$.
\begin{enumerate}
\item Compute $\mathsf{A}\vec{v}$
\item Find a vector $\vec{u}$ for which $\mathsf{A}\vec{u} = \vec{v}$, or show that none exists.
\end{enumerate}
\item \emph{Matrix operations.} Let $\mathsf{A} = \begin{pmatrix} 2 & 4 \\ 1 & 1 \end{pmatrix}$ and $\mathsf{B} = \begin{pmatrix} -3 & 4 \\ 5 & -7 \end{pmatrix}$. Compute each of the following.
\begin{enumerate}
\item $\mathsf{A} + \mathsf{B}$
\item $-3\mathsf{A}$
\item $\mathsf{AB}$
\item $\mathsf{BA}$
\item $\mathsf{B}^T$ (the transpose of $\mathsf{B}$)
\end{enumerate}
\item \emph{Geometric transformations.} Write down matrices for each of the following.
\begin{enumerate}
\item Dilation about the origin by a factor of 4
\item Horizontal dilation by a factor of 3 and vertical dilation by a factor of 2
\item Rotation about the origin by $\pi/4$ counterclockwise
\item Projection onto the line $y = (3/2)x$
\item Reflection across the line $y = (3/2)x$
\end{enumerate}
\item \emph{Matrix determinants.} Let $\mathsf{A} = \begin{pmatrix} 2 & 4 \\ 1 & 1 \end{pmatrix}$ and $\mathsf{B} = \begin{pmatrix} -3 & 4 \\ 5 & -7 \end{pmatrix}$. Compute each of the following.
\begin{enumerate}
\item $\det\mathsf{A}$ and $\det\mathsf{B}$
\item $\det(\mathsf{AB})$
\item $\det(\mathsf{A}^T)$
\item $\det(\mathsf{A} + \mathsf{B})$
\item The area of the ellipse formed by applying $\mathsf{A}$ to the unit circle
\end{enumerate}
\item \emph{Matrix inverses.} Let $\mathsf{A} = \begin{pmatrix} 2 & 4 \\ 1 & 1 \end{pmatrix}$ and $\mathsf{B} = \begin{pmatrix} -3 & 4 \\ 5 & -7 \end{pmatrix}$. Compute each of the following.
\begin{enumerate}
\item $\mathsf{A}^{-1}$ and $\mathsf{B}^{-1}$
\item $\mathsf{A}^{-1}\mathsf{B}^{-1}$ and $\mathsf{B}^{-1}\mathsf{A}^{-1}$
\item $(\mathsf{AB})^{-1}$
\item $(\mathsf{A}^T)^{-1}$
\item $(\mathsf{A} + \mathsf{B})^{-1}$
\item $\det(\mathsf{A}^{-1})$
\end{enumerate}
\item \emph{Shear transformations.} A \textbf{horizontal shear} is given by a matrix of the form $\begin{pmatrix} 1 & k \\ 0 & 1 \end{pmatrix}$.
\begin{enumerate}
\item Describe the image of the unit square with vertices $(0,0)$, $(1,0)$, $(1,1)$, and $(0,1)$ when the horizontal shear $\begin{pmatrix} 1 & 2 \\ 0 & 1 \end{pmatrix}$ is applied.
\item By what factor does a horizontal shear multiply areas?
\item Find real constants $a,b,k,\theta$ for which
\begin{equation*}
\begin{pmatrix} 4 & 1 \\ 3 & 7 \end{pmatrix} = \begin{pmatrix} \cos\theta & -\sin\theta \\ \sin\theta & \cos\theta \end{pmatrix}\begin{pmatrix} a & 0 \\ 0 & b \end{pmatrix}\begin{pmatrix} 1 & k \\ 0 & 1 \end{pmatrix}.
\end{equation*}
(The constant $\theta$ can be expressed in terms of an inverse trig function.)
\end{enumerate}
\end{enumerate}


\newpage
\subsection{Challenge Problems}

\emph{Challenge problems are meant to provide optional extensions of the ideas from class.}

\begin{enumerate}\setcounter{enumi}{7}
\item The \textbf{trace} of a square matrix is the sum of its main diagonal entries,
\begin{equation*}
\tr\begin{pmatrix} a & b \\ c & d \end{pmatrix} = a + d.
\end{equation*}
\begin{enumerate}
\item For the matrices $\mathsf{A}$ and $\mathsf{B}$ in problems 3, 5, and 6, compute $\tr\mathsf{A}$, $\tr\mathsf{B}$, and $\tr(\mathsf{AB})$.
\item Show that for any $2\times 2$ matrices $\mathsf{P}$ and $\mathsf{Q}$, we have $\tr(\mathsf{PQ}) = \tr(\mathsf{QP})$.
\item In general, must it be true that $\tr(\mathsf{ABC}) = \tr(\mathsf{ACB})$?
\end{enumerate}
\item Two matrices $\mathsf{A},\mathsf{B}$ are \textbf{similar}, written $\mathsf{A}\sim\mathsf{B}$, if there is an invertible $\mathsf{P}$ with $\mathsf{B} = \mathsf{P}^{-1}\mathsf{AP}$.
\begin{enumerate}
\item Show that the only matrix similar to $\mathsf{I}$ is $\mathsf{I}$.
\item Show that if $\mathsf{A}\sim\mathsf{B}$, then $\det\mathsf{A} = \det\mathsf{B}$ and $\tr\mathsf{A} = \tr\mathsf{B}$.
\item Let $\mathsf{A} = \begin{pmatrix} 3 & 1 \\ 2 & 2 \end{pmatrix}$. There is exactly one diagonal matrix $\mathsf{D} = \begin{pmatrix} d_1 & 0 \\ 0 & d_2 \end{pmatrix}$ with $d_1\geq d_2$ for which $\mathsf{D}\sim\mathsf{A}$. Find $\mathsf{D}$.
\end{enumerate}
\item If $\mathsf{A}$ is a square matrix, the \textbf{characteristic polynomial} of $\mathsf{A}$ is defined by
\begin{equation*}
f_{\mathsf{A}}(X) = \det(\mathsf{A} - X\mathsf{I}).
\end{equation*}
\begin{enumerate}
\item Compute the characteristic polynomial $f_{\mathsf{A}}(X)$ of the matrix $\mathsf{A} = \begin{pmatrix} 3 & 1 \\ 2 & 2 \end{pmatrix}$.
\item Find the two roots $\lambda_1\geq\lambda_2$ of $f_{\mathsf{A}}(X)$.
\item Find non-zero vectors $\vec{v}_1,\vec{v}_2$ for which $\mathsf{A}\vec{v}_j = \lambda_j\vec{v}_j$ for $j = 1,2$. (In general, if $\mathsf{A}\vec{v} = \lambda\vec{v}$ and $\vec{v}\neq\vec{0}$, we call $\vec{v}$ an \textbf{eigenvector} of $\mathsf{A}$ corresponding to the \textbf{eigenvalue} $\lambda$.)
\item Let $\mathsf{P}$ be the matrix whose columns are $\vec{v}_1$ and $\vec{v}_2$. Compute $\mathsf{P}^{-1}\mathsf{AP}$.
\item \emph{Cayley-Hamilton theorem.} Suppose $f_{\mathsf{A}}(X) = a_0 + a_1X + a_2X^2$. (The values of $a_0, a_1, a_2$ are known from part (a).) Compute
\begin{equation*}
a_0\mathsf{I} + a_1\mathsf{A} + a_2\mathsf{A}^2.
\end{equation*}
\end{enumerate}
\end{enumerate}


\newpage
\subsection{Answers}

\begin{enumerate}
\item \begin{enumerate}
\item $\begin{pmatrix} 6 \\ 2 \end{pmatrix}$
\item $\begin{pmatrix} 8 \\ -2 \end{pmatrix}$
\end{enumerate}
\item \begin{enumerate}
\item $\vec{x}(t) = \begin{pmatrix} 2 \\ 1 \end{pmatrix} + t\begin{pmatrix} 2 \\ -9 \end{pmatrix}$ (there are many other options)
\item Let $p$ and $q$ be the value of the parameters for $\vec{x}_1$ and $\vec{x}_2$ at the point of intersection, so $\vec{x}_1(p) = \vec{x}_2(q)$. Then
\begin{equation*}
\begin{pmatrix} 0 \\ 3 \end{pmatrix} + p\begin{pmatrix} 1 \\ -1 \end{pmatrix} = \begin{pmatrix} -4 \\ 2 \end{pmatrix} + q\begin{pmatrix} 2 \\ 3 \end{pmatrix},
\end{equation*}
which means $p = -4 + 2q$ and $3 - p = 2 + 3q$. Solving the system, $q = 1$ and $p = -2$, and the point of intersection is $(-2,5)$.
\end{enumerate}
\item \begin{enumerate}
\item $\|\vec{u}\| = \sqrt{13}$\par
$\vec{u}\cdot\vec{v} = 5$
\item $\hat{\vec{v}} = \dfrac{1}{\sqrt{17}}\begin{pmatrix} 4 \\ -1 \end{pmatrix} = \begin{pmatrix} 4/\sqrt{17} \\ -1/\sqrt{17} \end{pmatrix}$
\item $\cos\theta = \frac{5}{\sqrt{221}}$
\item $\pm\dfrac{1}{\sqrt{13}}\begin{pmatrix} 3 \\ -2 \end{pmatrix}$
\end{enumerate}
\item \begin{enumerate}
\item If $a\vec{u} + b\vec{v} = 0$, then $2a + 4b = 0$ and $3a - b = 0$. Adding 4 times the second equation to the first, $14a = 0$, so $a = 0$ and hence $b = 0$ as well.
\item $(a,b,c,d) = (1/14, 3/14, 2/7, -1/7)$
\item Let $\vec{x} = \begin{pmatrix} x_1 \\ x_2 \end{pmatrix}$ and $\vec{y} = \begin{pmatrix} y_1 \\ y_2 \end{pmatrix}$. If $a\vec{x} + b\vec{y} = \vec{0}$, then
\begin{align}
x_1a + y_1b &= 0, \\
x_2a + y_2b &= 0.
\end{align}
Taking the combination $y_2\cdot (1) - y_1\cdot (2)$,
\begin{equation*}
(x_1y_2 - x_2y_1)a = 0.
\end{equation*}
If $x_1y_2 - x_2y_1 = 0$, then we consider further subcases.
\begin{itemize}
\item If $\vec{x} = \vec{0}$, then taking $(a,b) = (1,0)$ gives a non-zero solution, so $\vec{x}$ and $\vec{y}$ are linearly dependent. Moreover, linear combinations $c\vec{x} + d\vec{y} = d\vec{y}$ can only produce multiples of $\vec{y}$, hence will not span all of $\mathbb{R}^2$.
\item If $\vec{x}\neq\vec{0}$, either $x_1\neq 0$ or $x_2\neq 0$. If $x_1\neq 0$, then let $\lambda = y_1/x_1$, so $y_1 = \lambda x_1$. Substituting, we find $y_2 = \lambda x_2$, so $\vec{y} = \lambda\vec{x}$. This means that $\vec{x}$ and $\vec{y}$ are linearly dependent. Also, any linear combination $c\vec{x} + d\vec{y} = (c + d\lambda)\vec{x}$ will be a multiple of $\vec{x}$, hence will not span all of $\mathbb{R}^2$.
\end{itemize}
If $x_1y_2 - x_2y_1\neq 0$, then $a = 0$. At least one of $y_1$ and $y_2$ is non-zero in this case, and substituting into the relevant equation gives $b = 0$ as well. Therefore, $\vec{x}$ and $\vec{y}$ are linearly independent. To see that they span $\mathbb{R}^2$, let $\begin{pmatrix} p \\ q \end{pmatrix}$ be arbitrary. We look for coefficients $c,d$ such that $c\vec{x} + d\vec{y} = \begin{pmatrix} p \\ q \end{pmatrix}$, or
\begin{align*}
x_1c + y_1d &= p, \\
x_2c + y_2d &= q.
\end{align*}
This has a solution, namely $(c,d) = \left(\dfrac{py_2 - qy_1}{x_1y_2 - x_2y_1}, \dfrac{qx_1 - px_2}{x_1y_2 - x_2y_1}\right)$.
\item Let $\vec{x},\vec{y},\vec{z}$ be given. If $\vec{x}$ and $\vec{y}$ are linearly dependent, then $\vec{x}$, $\vec{y}$, and $\vec{z}$ are as well. Otherwise, by part (c), there exist $a,b$ for which $\vec{z} = a\vec{x} + b\vec{y}$. Then $a\vec{x} + b\vec{y} - \vec{z}$ is a non-trivial linear combination of the three which equals $\vec{0}$, so they are linearly dependent.
\end{enumerate}
\item \begin{enumerate}
\item $\proj_{\vec{u}}(\vec{v}) = \begin{pmatrix} 10/13 \\ 15/13 \end{pmatrix}$\par
$\proj_{\vec{v}}(\vec{u}) = \begin{pmatrix} 20/17 \\ -5/17 \end{pmatrix}$
\item It suffices to show $\vec{x}$ and $\vec{y}$ are linearly independent. Suppose $a\vec{x} + b\vec{y} = \vec{0}$. Then
\begin{equation*}
0 = \vec{x}\cdot (a\vec{x} + b\vec{y}) = a(\vec{x}\cdot\vec{x}) + b(\vec{x}\cdot\vec{y}) = a(1) + b(0) = a.
\end{equation*}
By a similar argument, $b = 0$.
\item Translating up by 1 unit, it suffices to find the distance between the head of $\vec{u} = \begin{pmatrix} 4 \\ 6 \end{pmatrix}$ and the line $y = 2x$, which is spanned by the vector $\vec{v} = \begin{pmatrix} 1 \\ 2 \end{pmatrix}$. We compute
\begin{equation*}
\|\proj_{\vec{v}}(\vec{u})\| = \frac{\lvert\vec{u}\cdot\vec{v}\rvert}{\|\vec{v}\|} = \frac{16}{\sqrt{5}}.
\end{equation*}
\end{enumerate}
\item \begin{enumerate}
\item $y = (-2/3)x + (4/3)$
\item $\vec{n} = \begin{pmatrix} 3 \\ 1 \end{pmatrix}$ and $d = 5$ (there are many choices that work)
\item If $\vec{x}_1,\vec{x}_2$ are two position vectors for points on the line, then $\vec{v} = \vec{x}_2 - \vec{x}_1$ points along the line. Since
\begin{equation*}
\hat{\vec{n}}\cdot\vec{v} = \hat{\vec{n}}\cdot (\vec{x}_2 - \vec{x}-1) = \hat{\vec{n}}\cdot\vec{x}_2 - \hat{\vec{n}}\cdot\vec{x}_1 = d - d = 0,
\end{equation*}
$\hat{\vec{n}}$ is perpendicular to the line. Then, for any position vector $\vec{x}$ on the line, we compute the distance from the origin to the line as
\begin{equation*}
\|\proj_{\hat{\vec{n}}}(\vec{x})\| = \frac{\lvert\hat{\vec{n}}\cdot\vec{x}\rvert}{\|\hat{\vec{n}}\|} = d.
\end{equation*}
\end{enumerate}
\item \begin{enumerate}
\item By the angle bisector theorem, $BD/DC = AB/AC = 1/2$. Therefore, $\vec{D} = \frac{2}{3}\vec{B} + \frac{1}{3}\vec{C}$.
\item The points on line $\overline{AD}$ are those with position vector of the form
\begin{equation*}
t\vec{A} + (1 - t)\vec{D} = t\vec{A} + (1 - t)\left(\frac{2}{3}\vec{B} + \frac{1}{3}\vec{C}\right)
\end{equation*}
for a real number $t$. Setting $t = 5/14$ gives us $\vec{I}$.
\end{enumerate}
\item \begin{enumerate}
\item We show that $\vec{y} = 2\proj_{\vec{v}}(\vec{x}) - \vec{x}$ has the defining properties of reflection, namely that the line $\ell$ spanned by $\vec{v}$ is orthogonal to $\vec{x} - \vec{y}$ and passes through the midpoint of the segment connecting $\vec{x}$ and $\vec{y}$. First, $(\vec{x} + \vec{y})/2 = \proj_{\vec{v}}(\vec{x})$, so $\ell$ passes through the midpoint. For orthogonality,
\begin{equation*}
\vec{v}\cdot (\vec{x} - \vec{y}) = 2\vec{v}\cdot (\vec{x} - \proj_{\vec{v}}(\vec{x})) = 0.
\end{equation*}
\item We compute
\begin{align*}
\refl_{\vec{v}}(a\vec{x} + b\vec{y}) &= 2\proj_{\vec{v}}(a\vec{x} + b\vec{y}) - (a\vec{x} + b\vec{y}) \\
&= 2(a\proj_{\vec{v}}(\vec{x}) + b\proj_{\vec{v}}(\vec{y})) - (a\vec{x} + b\vec{y}) \\
&= a(2\proj_{\vec{v}}(\vec{x}) - \vec{x}) + b(2\proj_{\vec{v}}(\vec{y}) - \vec{y}) \\
&= a\refl_{\vec{v}}(\vec{x}) + b\refl_{\vec{v}}(\vec{y}).
\end{align*}
\end{enumerate}
\item \begin{enumerate}
\item Linearity of $\proj_{\vec{v}}$ follows from (bi)linearity of the dot product.\par
To see that $\proj_{\vec{v}}\circ\proj_{\vec{v}} = \proj_{\vec{v}}$, we compute
\begin{align*}
(\proj_{\vec{v}}\circ\proj_{\vec{v}})(\vec{x}) &= \proj_{\vec{v}}\left(\left[\vec{x}\cdot\frac{\vec{v}}{\|\vec{v}\|^2}\right]\vec{v}\right) \\
&= \left(\left[\vec{x}\cdot\frac{\vec{v}}{\|\vec{v}\|^2}\right]\vec{v}\cdot\frac{\vec{v}}{\|\vec{v}\|^2}\right)\vec{v} \\
&= \left[\vec{x}\cdot\frac{\vec{v}}{\|\vec{v}\|^2}\right]\vec{v} = \proj_{\vec{v}}(\vec{x}).
\end{align*}
Finally, orthogonality is part of how $\proj_{\vec{v}}$ was defined.
\item Let $\vec{x}$ be a vector for which $P(\vec{x})\neq\vec{0}$ and let $\vec{y}$ be a vector for which $P(\vec{y})\neq\vec{y}$. (These conditions imply $\vec{x},\vec{y}\neq\vec{0}$.) Let $\vec{u} = P(\vec{x})$ and $\vec{v} = \vec{y} - P(\vec{y})$. These are non-zero, and
\begin{align*}
P(\vec{u}) &= P(P(\vec{x})) = P(\vec{x}) = \vec{u}, \\
P(\vec{v}) &= P(\vec{y} - P(\vec{y})) = P(\vec{y}) - P(P(\vec{y})) = P(\vec{y}) - P(\vec{y}) = \vec{0}.
\end{align*}
For linear independence, suppose $a\vec{u} + b\vec{v} = \vec{0}$. Applying $P$,
\begin{equation*}
\vec{0} = P(\vec{0}) = P(a\vec{u} + b\vec{v}) = aP(\vec{u}) + bP(\vec{v}) = a\vec{u},
\end{equation*}
so $a = 0$. Then, $b\vec{v} = \vec{0}$, so $b = 0$ as well.
\item Since $\vec{u},\vec{v}$ are linearly independent in $\mathbb{R}^2$, they also span $\mathbb{R}^2$. Given any vector $\vec{x}\in\mathbb{R}^2$, let $\vec{x} = a\vec{u} + b\vec{v}$. Then
\begin{equation*}
P(\vec{x}) = P(a\vec{u} + b\vec{v}) = aP(\vec{u}) + bP(\vec{v}) = a\vec{u}.
\end{equation*}
As $a$ and $b$ range over all real numbers, we find the range of $P$ is the span of $\vec{u}$.
\end{enumerate}
\item \begin{enumerate}
\item We proceed by induction. By definition, $s_0 = ax_0 + by_0$ and $s_1 = ax_1 + by_1$. Then, if $s_n = ax_n + by_n$ and $s_{n + 1} = ax_{n + 1} + by_{n + 1}$,
\begin{align*}
s_{n + 2} &= s_{n + 1} + 2s_n \\
&= (ax_{n + 1} + by_{n + 1}) + 2(ax_n + by_n) \\
&= a(x_{n + 1} + 2x_n) + b(y_{n + 1} + 2y_n) \\
&= ax_{n + 2} + by_{n + 2}.
\end{align*}
\item We need $\lambda^n(\lambda^2 - \lambda - 2) = 0$ for all $n\geq 0$. This holds when $\lambda = -1$ and when $\lambda = 2$.
\item The sequence $e_n = (-1)^n$ corresponds to $\begin{pmatrix} 1 \\ -1 \end{pmatrix}$, while the sequence $f_n = 2^n$ corresponds to $\begin{pmatrix} 1 \\ 2 \end{pmatrix}$. We can find
\begin{equation*}
\begin{pmatrix} 4 \\ 7 \end{pmatrix} = \frac{1}{3}\begin{pmatrix} 1 \\ -1 \end{pmatrix} + \frac{11}{3}\begin{pmatrix} 1 \\ 2 \end{pmatrix},
\end{equation*}
so
\begin{equation*}
x_n = \frac{1}{3}e_n + \frac{11}{3}f_n = \frac{1}{3}\cdot (-1)^n + \frac{11}{3}\cdot 2^n.
\end{equation*}
\end{enumerate}
\end{enumerate}
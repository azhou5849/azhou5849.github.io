\section{Matrices in 2D}

\subsection{Review Problems}

\emph{Review problems are meant to cover ``standard'' definitions and calculations as well as the use of some important results.}

Throughout, $\unit{i} = \begin{pmatrix} 1 \\ 0 \end{pmatrix}$ and $\unit{j} = \begin{pmatrix} 0 \\ 1 \end{pmatrix}$ are the standard unit vectors while $\vec{0} = \begin{pmatrix} 0 \\ 0 \end{pmatrix}$ is the zero vector. We also let $\mathsf{I} = \begin{pmatrix} 1 & 0 \\ 0 & 1 \end{pmatrix}$ be the ($2\times 2$) identity matrix and $\mathsf{0} = \begin{pmatrix} 0 & 0 \\ 0 & 0 \end{pmatrix}$ be the zero matrix.

\begin{enumerate}
\item \emph{Vector calculations.} Let $\vec{u} = \begin{pmatrix} 2 \\ 3 \end{pmatrix}$ and $\vec{v} = \begin{pmatrix} 4 \\ -1 \end{pmatrix}$. Compute each of the following.
\begin{enumerate}
\item $\vec{u} + \vec{v}$
\item $2\vec{v}$
\item $\vec{u}\cdot\vec{v}$ and $\vec{v}\cdot\vec{u}$
\item $\|\vec{u}\|$, $\|\vec{v}\|$, and $\|\vec{u} + \vec{v}\|$
\item The angle between $\vec{u}$ and $\vec{v}$ (in terms of an inverse trig function)
\item $\proj_{\vec{v}}(\vec{u})$ and $\proj_{\vec{u}}(\vec{v})$
\end{enumerate}
\item \emph{Applying matrices to vectors.} Let $\mathsf{A} = \begin{pmatrix} 2 & 4 \\ 1 & 1 \end{pmatrix}$ and $\vec{v} = \begin{pmatrix} 5 \\ 2 \end{pmatrix}$.
\begin{enumerate}
\item Compute $\mathsf{A}\vec{v}$
\item Find a vector $\vec{u}$ for which $\mathsf{A}\vec{u} = \vec{v}$, or show that none exists.
\end{enumerate}
\item \emph{Matrix operations.} Let $\mathsf{A} = \begin{pmatrix} 2 & 4 \\ 1 & 1 \end{pmatrix}$ and $\mathsf{B} = \begin{pmatrix} -3 & 4 \\ 5 & -7 \end{pmatrix}$. Compute each of the following.
\begin{enumerate}
\item $\mathsf{A} + \mathsf{B}$
\item $-3\mathsf{A}$
\item $\mathsf{AB}$
\item $\mathsf{BA}$
\item $\mathsf{B}^T$ (the transpose of $\mathsf{B}$)
\end{enumerate}
\item \emph{Geometric transformations.} Write down matrices for each of the following.
\begin{enumerate}
\item Dilation about the origin by a factor of 4
\item Horizontal dilation by a factor of 3 and vertical dilation by a factor of 2
\item Rotation about the origin by $\pi/4$ counterclockwise
\item Projection onto the line $y = (3/2)x$
\item Reflection across the line $y = (3/2)x$
\end{enumerate}
\item \emph{Matrix determinants.} Let $\mathsf{A} = \begin{pmatrix} 2 & 4 \\ 1 & 1 \end{pmatrix}$ and $\mathsf{B} = \begin{pmatrix} -3 & 4 \\ 5 & -7 \end{pmatrix}$. Compute each of the following.
\begin{enumerate}
\item $\det\mathsf{A}$ and $\det\mathsf{B}$
\item $\det(\mathsf{AB})$
\item $\det(\mathsf{A}^T)$
\item $\det(\mathsf{A} + \mathsf{B})$
\item The area of the ellipse formed by applying $\mathsf{A}$ to the unit circle
\end{enumerate}
\item \emph{Matrix inverses.} Let $\mathsf{A} = \begin{pmatrix} 2 & 4 \\ 1 & 1 \end{pmatrix}$ and $\mathsf{B} = \begin{pmatrix} -3 & 4 \\ 5 & -7 \end{pmatrix}$. Compute each of the following.
\begin{enumerate}
\item $\mathsf{A}^{-1}$ and $\mathsf{B}^{-1}$
\item $\mathsf{A}^{-1}\mathsf{B}^{-1}$ and $\mathsf{B}^{-1}\mathsf{A}^{-1}$
\item $(\mathsf{AB})^{-1}$
\item $(\mathsf{A}^T)^{-1}$
\item $(\mathsf{A} + \mathsf{B})^{-1}$
\item $\det(\mathsf{A}^{-1})$
\end{enumerate}
\item \emph{Shear transformations.} A \textbf{horizontal shear} is given by a matrix of the form $\begin{pmatrix} 1 & k \\ 0 & 1 \end{pmatrix}$.
\begin{enumerate}
\item Describe the image of the unit square with vertices $(0,0)$, $(1,0)$, $(1,1)$, and $(0,1)$ when the horizontal shear $\begin{pmatrix} 1 & 2 \\ 0 & 1 \end{pmatrix}$ is applied.
\item By what factor does a horizontal shear multiply areas?
\item Find real constants $a,b,k,\theta$ for which
\begin{equation*}
\begin{pmatrix} 4 & 1 \\ 3 & 7 \end{pmatrix} = \begin{pmatrix} \cos\theta & -\sin\theta \\ \sin\theta & \cos\theta \end{pmatrix}\begin{pmatrix} a & 0 \\ 0 & b \end{pmatrix}\begin{pmatrix} 1 & k \\ 0 & 1 \end{pmatrix}.
\end{equation*}
(The constant $\theta$ can be expressed in terms of an inverse trig function.)
\end{enumerate}
\end{enumerate}


\newpage
\subsection{Challenge Problems}

\emph{Challenge problems are meant to provide optional extensions of the ideas from class.}

\begin{enumerate}\setcounter{enumi}{7}
\item The \textbf{trace} of a square matrix is the sum of its main diagonal entries,
\begin{equation*}
\tr\begin{pmatrix} a & b \\ c & d \end{pmatrix} = a + d.
\end{equation*}
\begin{enumerate}
\item For the matrices $\mathsf{A},\mathsf{B}$ in problems 3, 5, 6, compute $\tr\mathsf{A}$, $\tr\mathsf{B}$, $\tr(\mathsf{A} + \mathsf{B})$, and $\tr(\mathsf{AB})$.
\item Show that for any $2\times 2$ matrices $\mathsf{P}$ and $\mathsf{Q}$, we have $\tr(\mathsf{PQ}) = \tr(\mathsf{QP})$.
\item In general, must it be true that $\tr(\mathsf{ABC}) = \tr(\mathsf{ACB})$?
\end{enumerate}
\item Two matrices $\mathsf{A},\mathsf{B}$ are \textbf{similar}, written $\mathsf{A}\sim\mathsf{B}$, if there is an invertible $\mathsf{P}$ with $\mathsf{B} = \mathsf{P}^{-1}\mathsf{AP}$.
\begin{enumerate}
\item Show that the only matrix similar to $\mathsf{I}$ is $\mathsf{I}$.
\item Show that if $\mathsf{A}\sim\mathsf{B}$, then $\det\mathsf{A} = \det\mathsf{B}$ and $\tr\mathsf{A} = \tr\mathsf{B}$.
\item Let $\mathsf{A} = \begin{pmatrix} 3 & 1 \\ 2 & 2 \end{pmatrix}$. There is exactly one diagonal matrix $\mathsf{D} = \begin{pmatrix} d_1 & 0 \\ 0 & d_2 \end{pmatrix}$ with $d_1\geq d_2$ for which $\mathsf{D}\sim\mathsf{A}$. Find $\mathsf{D}$.
\end{enumerate}
\item If $\mathsf{A}$ is a square matrix, the \textbf{characteristic polynomial} of $\mathsf{A}$ is defined by
\begin{equation*}
f_{\mathsf{A}}(X) = \det(\mathsf{A} - X\mathsf{I}).
\end{equation*}
\begin{enumerate}
\item Compute the characteristic polynomial $f_{\mathsf{A}}(X)$ of the matrix $\mathsf{A} = \begin{pmatrix} 3 & 1 \\ 2 & 2 \end{pmatrix}$.
\item Find the two roots $\lambda_1\geq\lambda_2$ of $f_{\mathsf{A}}(X)$.
\item Find non-zero vectors $\vec{v}_1,\vec{v}_2$ for which $\mathsf{A}\vec{v}_j = \lambda_j\vec{v}_j$ for $j = 1,2$. (In general, if $\mathsf{A}\vec{v} = \lambda\vec{v}$ and $\vec{v}\neq\vec{0}$, we call $\vec{v}$ an \textbf{eigenvector} of $\mathsf{A}$ corresponding to the \textbf{eigenvalue} $\lambda$.)
\item Let $\mathsf{P}$ be the matrix whose columns are $\vec{v}_1$ and $\vec{v}_2$. Compute $\mathsf{P}^{-1}\mathsf{AP}$.
\item Find $\mathsf{A}^{100}$.
\item \emph{Cayley-Hamilton theorem.} Suppose $f_{\mathsf{A}}(X) = a_0 + a_1X + a_2X^2$. (The values of $a_0, a_1, a_2$ are known from part (a).) Compute
\begin{equation*}
a_0\mathsf{I} + a_1\mathsf{A} + a_2\mathsf{A}^2.
\end{equation*}
\end{enumerate}
\end{enumerate}


\newpage
\subsection{Answers}

\begin{enumerate}
\item \begin{enumerate}
\item $\begin{pmatrix} 6 \\ 2 \end{pmatrix}$
\item $\begin{pmatrix} 8 \\ -2 \end{pmatrix}$
\item Both are $5$. In general, $\vec{a}\cdot\vec{b} = \vec{b}\cdot\vec{a}$.
\item $\|\vec{u}\| = \sqrt{13}$\par
$\|\vec{v}\| = \sqrt{17}$\par
$\|\vec{u} + \vec{v}\| = \sqrt{40} = 2\sqrt{10}$
\item $\arccos\left(\frac{5}{\sqrt{221}}\right)$
\item $\proj_{\vec{v}}(\vec{u}) = \begin{pmatrix} 20/17 \\ -5/17 \end{pmatrix}$\par
$\proj_{\vec{u}}(\vec{v}) = \begin{pmatrix} 10/13 \\ 15/13 \end{pmatrix}$
\end{enumerate}
\item \begin{enumerate}
\item $\begin{pmatrix} 18 \\ 7 \end{pmatrix}$
\item Let $\vec{u} = \begin{pmatrix} a \\ b \end{pmatrix}$. Then
\begin{equation*}
\mathsf{A}\vec{u} = \begin{pmatrix} 2 & 4 \\ 1 & 1 \end{pmatrix}\begin{pmatrix} a \\ b \end{pmatrix} = \begin{pmatrix} 2a + 4b \\ a + b \end{pmatrix},
\end{equation*}
so we require $2a + 4b = 5$ and $a + b = 2$. The solution to this system is that $a = 3/2$ and $b = 1/2$, so then $\vec{u} = \begin{pmatrix} 3/2 \\ 1/2 \end{pmatrix}$.\par
\textit{Remark:} We can also compute $\vec{u} = \mathsf{A}^{-1}\vec{v}$ once we have $\mathsf{A}^{-1}$ (see Problem 6).
\end{enumerate}
\item \begin{enumerate}
\item $\begin{pmatrix} -1 & 8 \\ 6 & -6 \end{pmatrix}$
\item $\begin{pmatrix} -6 & -12 \\ -3 & -3 \end{pmatrix}$
\item $\begin{pmatrix} 14 & -20 \\ 2 & -3 \end{pmatrix}$
\item $\begin{pmatrix} -2 & -8 \\ 3 & 13 \end{pmatrix}$
\item $\begin{pmatrix} -3 & 5 \\ 4 & -7 \end{pmatrix}$
\end{enumerate}
\item \begin{enumerate}
\item $4\mathsf{I} = \begin{pmatrix} 4 & 0 \\ 0 & 4 \end{pmatrix}$
\item $\begin{pmatrix} 3 & 0 \\ 0 & 2 \end{pmatrix}$
\item $\begin{pmatrix} \cos(\pi/4) & -\sin(\pi/4) \\ \sin(\pi/4) & \cos(\pi/4) \end{pmatrix} = \begin{pmatrix} 1/\sqrt{2} & -1/\sqrt{2} \\ 1/\sqrt{2} & 1/\sqrt{2} \end{pmatrix}$
\item $\mathsf{P} = \begin{pmatrix} 4/13 & 6/13 \\ 6/13 & 9/13 \end{pmatrix}$
\item $2\mathsf{P} - \mathsf{I} = \begin{pmatrix} -5/13 & 12/13 \\ 12/13 & 5/13 \end{pmatrix}$
\end{enumerate}
\item \begin{enumerate}
\item $\det\mathsf{A} = -2$\par 
$\det\mathsf{B} = 1$
\item $\det(\mathsf{AB}) = \det(\mathsf{A})\cdot\det(\mathsf{B}) = -2$
\item $\det(\mathsf{A}^T) = \det\mathsf{A} = -2$
\item $\det(\mathsf{A} + \mathsf{B}) = \det\begin{pmatrix} -1 & 8 \\ 6 & -6 \end{pmatrix} = -42$
\item $\lvert\det\mathsf{A}\rvert\cdot(\text{unit circle area}) = 2\pi$
\end{enumerate}
\item \begin{enumerate}
\item $\mathsf{A}^{-1} = \dfrac{1}{\det\mathsf{A}}\begin{pmatrix} 1 & -4 \\ -1 & 2 \end{pmatrix} = \begin{pmatrix} -1/2 & 2 \\ 1/2 & -1 \end{pmatrix}$\par
$\mathsf{B}^{-1} = \dfrac{1}{\det\mathsf{B}}\begin{pmatrix} -7 & -4 \\ -5 & -3 \end{pmatrix} = \begin{pmatrix} -7 & -4 \\ -5 & -3 \end{pmatrix}$
\item $\mathsf{A}^{-1}\mathsf{B}^{-1} = \begin{pmatrix} -13/2 & -4 \\ 3/2 & 1 \end{pmatrix}$\par
$\mathsf{B}^{-1}\mathsf{A}^{-1} = \begin{pmatrix} 3/2 & -10 \\ 1 & -7 \end{pmatrix}$
\item $(\mathsf{AB})^{-1} = \mathsf{B}^{-1}\mathsf{A}^{-1} = \begin{pmatrix} 3/2 & -10 \\ 1 & -7 \end{pmatrix}$
\item $(\mathsf{A}^T)^{-1} = (\mathsf{A}^{-1})^T = \begin{pmatrix} -1/2 & 1/2 \\ 2 & -1 \end{pmatrix}$
\item $(\mathsf{A} + \mathsf{B})^{-1} = \dfrac{1}{\det(\mathsf{A} + \mathsf{B})}\begin{pmatrix} -6 & -8 \\ -6 & -1 \end{pmatrix} = \begin{pmatrix} 1/7 & 4/21 \\ 1/7 & 1/42 \end{pmatrix}$
\item $\det(\mathsf{A}^{-1}) = 1/\det\mathsf{A} = -1/2$
\end{enumerate}
\item \begin{enumerate}
\item A parallelogram with vertices $(0,0), (1,0), (3,1), (2,1)$
\item $\det\begin{pmatrix} 1 & k \\ 0 & 1 \end{pmatrix} = 1$
\item Multiplying the right two matrices, $\begin{pmatrix} 4 & 1 \\ 3 & 7 \end{pmatrix} = \begin{pmatrix} \cos\theta & -\sin\theta \\ \sin\theta & \cos\theta \end{pmatrix}\begin{pmatrix} a & ak \\ 0 & b \end{pmatrix}$. Looking at the image of vector $\unit{i}$, we need $\begin{pmatrix} a \\ 0 \end{pmatrix}$ to rotate to $\begin{pmatrix} 4 \\ 3 \end{pmatrix}$. This can be achieved with a rotation by $\theta = \arccos(4/5)$ and $a = 5$. To find $b$, taking the determinant on both sides and noting that rotations have determinant 1, we require $ab = 25$, so $b = 5$. Finally, to get $k$, we need $\begin{pmatrix} 5k \\ 5 \end{pmatrix}$ to rotate to $\begin{pmatrix} 1 \\ 7 \end{pmatrix}$. Comparing lengths and noting that $\begin{pmatrix} 5k \\ 5 \end{pmatrix}$ must be in the first quadrant, $k = 1$.
\end{enumerate}
\item \begin{enumerate}
\item $\tr\mathsf{A} = 3$\par
$\tr\mathsf{B} = -10$\par
$\tr(\mathsf{A} + \mathsf{B}) = \tr\mathsf{A} + \tr\mathsf{B} = -7$\par
$\tr(\mathsf{AB}) = 11$
\item Let $\mathsf{P} = \begin{pmatrix} a & b \\ c & d \end{pmatrix}$ and $\mathsf{Q} = \begin{pmatrix} e & f \\ g & h \end{pmatrix}$. Then
\begin{equation*}
\mathsf{PQ} = \begin{pmatrix} ae + bg & af + bh \\ ce + dg & cf + dh \end{pmatrix}\quad\text{and}\quad\mathsf{QP} = \begin{pmatrix} ae + cf & be + df \\ ag + ch & bg + dh \end{pmatrix},
\end{equation*}
so $\tr(\mathsf{PQ}) = \tr(\mathsf{QP}) = ae + bg + cf + dh$.
\item In general, the answer is \textbf{no}. For example, let
\begin{equation*}
\mathsf{A} = \begin{pmatrix} 2 & 4 \\ 1 & 1 \end{pmatrix},\quad\mathsf{B} = \begin{pmatrix} 1 & 1 \\ 0 & 1 \end{pmatrix},\quad\mathsf{C} = \begin{pmatrix} 2 & 0 \\ 0 & 1 \end{pmatrix}.
\end{equation*}
Then
\begin{align*}
\mathsf{ABC} &= \begin{pmatrix} 2 & 4 \\ 1 & 1 \end{pmatrix}\begin{pmatrix} 2 & 1 \\ 0 & 1 \end{pmatrix} = \begin{pmatrix} 4 & 6 \\ 2 & 2 \end{pmatrix}, \\
\mathsf{ACB} &= \begin{pmatrix} 2 & 4 \\ 1 & 1 \end{pmatrix}\begin{pmatrix} 2 & 2 \\ 0 & 1 \end{pmatrix} = \begin{pmatrix} 4 & 8 \\ 2 & 3 \end{pmatrix},
\end{align*}
so $\tr(\mathsf{ABC}) = 6$ while $\tr(\mathsf{ACB}) = 7$.
\end{enumerate}
\item \begin{enumerate}
\item Suppose $\mathsf{I}\sim\mathsf{B}$. Then there is an invertible matrix $\mathsf{P}$ such that $\mathsf{B} = \mathsf{P}^{-1}\mathsf{IP}$, but the right hand side simplifies to $\mathsf{P}^{-1}\mathsf{P} = \mathsf{I}$.
\item If $\mathsf{A}\sim\mathsf{B}$ with $\mathsf{B} = \mathsf{P}^{-1}\mathsf{AP}$, then
\begin{equation*}
\det\mathsf{B} = \det(\mathsf{P}^{-1}\mathsf{AP}) = \det(\mathsf{P})^{-1}\cdot\det\mathsf{A}\cdot\det\mathsf{P} = \det\mathsf{A}.
\end{equation*}
For the trace, Problem 8b gives us
\begin{equation*}
\tr\mathsf{B} = \tr(\mathsf{P}^{-1}(\mathsf{AP})) = \tr((\mathsf{AP})\mathsf{P}^{-1}) = \tr\mathsf{A}.
\end{equation*}
\item We have $\det\mathsf{A} = 4$ and $\tr\mathsf{A} = 5$, so
\begin{equation*}
\det\mathsf{D} = d_1d_2 = 4\quad\text{and}\quad\tr\mathsf{D} = d_1 + d_2 = 5.
\end{equation*}
This is satisfied by $d_1 = 4$ and $d_2 = 1$, so $\mathsf{D} = \begin{pmatrix} 4 & 0 \\ 0 & 1 \end{pmatrix}$.
\end{enumerate}
\item \begin{enumerate}
\item We compute
\begin{equation*}
f_{\mathsf{A}}(X) = \det(\mathsf{A} - X\mathsf{I}) = \det\begin{pmatrix} 3 - X & 1 \\ 2 & 2 - X \end{pmatrix} = (3 - X)(2 - X) - 2 = X^2 - 5X + 4.
\end{equation*}
\item The roots are $\lambda_1 = 4$ and $\lambda_2 = 1$.
\item Note that the equation $\mathsf{A}\vec{v} = \lambda\vec{v}$ is equivalent to $(\mathsf{A} - \lambda\mathsf{I})\vec{v} = \vec{0}$, which has a non-zero solution if and only if $\det(\mathsf{A} - \lambda\mathsf{I}) = 0$. Moreover, we can use this version of the equation to find solutions more easily.\par
For $\lambda_1 = 4$, we have $\mathsf{A} - \lambda_1\mathsf{I} = \begin{pmatrix} -1 & 1 \\ 2 & -2 \end{pmatrix}$, so we can take $\vec{v}_1 = \begin{pmatrix} 1 \\ 1 \end{pmatrix}$ (or any non-zero scalar multiple) as a solution to $(\mathsf{A} - \lambda_1\mathsf{I})\vec{v} = \vec{0}$.\par
For $\lambda_2 = 1$, we have $\mathsf{A} - \lambda_2\mathsf{I} = \begin{pmatrix} 2 & 1 \\ 2 & 1 \end{pmatrix}$, so we can take $\vec{v}_2 = \begin{pmatrix} 1 \\ -2 \end{pmatrix}$ (or any non-zero scalar multiple) as a solution to $(\mathsf{A} - \lambda_2\mathsf{I})\vec{v} = \vec{0}$.
\item Here $\mathsf{P} = \begin{pmatrix} 1 & 1 \\ 1 & -2 \end{pmatrix}$, so then $\mathsf{P}^{-1} = -\dfrac{1}{3}\begin{pmatrix} -2 & -1 \\ -1 & 1 \end{pmatrix} = \dfrac{1}{3}\begin{pmatrix} 2 & 1 \\ 1 & -1 \end{pmatrix}$. We compute
\begin{align*}
\mathsf{P}^{-1}\mathsf{AP} &= \frac{1}{3}\begin{pmatrix} 2 & 1 \\ 1 & -1 \end{pmatrix}\begin{pmatrix} 3 & 1 \\ 2 & 2 \end{pmatrix}\begin{pmatrix} 1 & 1 \\ 1 & -2 \end{pmatrix} \\
&= \frac{1}{3}\begin{pmatrix} 2 & 1 \\ 1 & -1 \end{pmatrix}\begin{pmatrix} 4 & 1 \\ 4 & -2 \end{pmatrix} \\
&= \frac{1}{3}\begin{pmatrix} 12 & 0 \\ 0 & 3 \end{pmatrix} = \begin{pmatrix} 4 & 0 \\ 0 & 1 \end{pmatrix}.
\end{align*}
\textit{Remark 1:} If we produced different valid choices of $\vec{v}_1$ and $\vec{v}_2$ from part (c), $\mathsf{P}$ and $\mathsf{P}^{-1}$ would change, but the end result would be the same. If we swapped the order of the columns of $\mathsf{P}$, then we would swap the order of the diagonal entries correspondingly.\par
\textit{Remark 2:} The fact that we got a diagonal matrix with entries $\lambda_1, \lambda_2$, the same one as in Problem 9c, is not a coincidence. The process we went through in this problem is called \textbf{diagonalisation}. (Not all $n\times n$ matrices are diagonalisable, but one sufficient condition for diagonalisability is that the characteristic polynomial has $n$ distinct roots.)
\item Let $\mathsf{D} = \mathsf{P}^{-1}\mathsf{AP} = \begin{pmatrix} 4 & 0 \\ 0 & 1 \end{pmatrix}$, so then $\mathsf{A} = \mathsf{PDP}^{-1}$. Then
\begin{align*}
\mathsf{A}^{100} &= \mathsf{PDP}^{-1}\cdot\mathsf{PDP}^{-1}\cdot\mathsf{PDP}^{-1}\cdot\ldots\cdot\mathsf{PDP}^{-1}\cdot\mathsf{PDP}^{-1} \\
&= \mathsf{PD}\cdot\mathsf{D}\cdot\mathsf{D}\cdot\ldots\cdot\mathsf{D}\cdot\mathsf{DP}^{-1} = \mathsf{PD}^{100}\mathsf{P}^{-1} \\
&= \begin{pmatrix} 1 & 1 \\ 1 & -2 \end{pmatrix}\begin{pmatrix} 4^{100} & 0 \\ 0 & 1 \end{pmatrix}\cdot\frac{-1}{3}\begin{pmatrix} -2 & -1 \\ -1 & 1 \end{pmatrix} \\
&= \frac{1}{3}\begin{pmatrix} 4^{100} & 1 \\ 4^{100} & -2 \end{pmatrix}\begin{pmatrix} 2 & 1 \\ 1 & -1 \end{pmatrix} \\
&= \frac{1}{3}\begin{pmatrix} 2\cdot 4^{100} + 1 & 4^{100} - 1 \\ 2\cdot 4^{100} -2 & 4^{100} + 2 \end{pmatrix}.
\end{align*}
\item Here $(a_0, a_1, a_2) = (4, -5, 1)$, so
\begin{align*}
a_0\mathsf{I} + a_1\mathsf{A} + a_2\mathsf{A}^2 &= \begin{pmatrix} 4 & 0 \\ 0 & 4 \end{pmatrix} + \begin{pmatrix} -15 & -5 \\ -10 & -10 \end{pmatrix} + \begin{pmatrix} 3 & 1 \\ 2 & 2 \end{pmatrix}\begin{pmatrix} 3 & 1 \\ 2 & 2 \end{pmatrix} \\
&= \begin{pmatrix} -11 & -5 \\ -10 & -6 \end{pmatrix} + \begin{pmatrix} 11 & 5 \\ 10 & 6 \end{pmatrix} = \mathsf{0}.
\end{align*}
\end{enumerate}
\end{enumerate}
\section{Laws of Sines and Cosines}

\subsection{Review problems}

Calculators are recommended for this section. Throughout, if $ABC$ is a triangle, then we use $a$, $b$, and $c$ to denote the side lengths $BC$, $CA$, and $AB$, respectively. (That is, $a$ is the length of the side opposite $A$, etc.) The notation $[ABC]$ denotes the area of $ABC$.

\begin{enumerate}
\item \emph{SAS congruence.} Let $ABC$ be a triangle with $a = 1$, $b = 5$, and $\angle C = 104^{\circ}$.
\begin{enumerate}
\item Find $[ABC]$.
\item Find $c$.
\item Using the law of sines, or otherwise, find $\sin A$ and $\sin B$.
\item Show that $\angle A = \arcsin(\sin A)$ and $\angle B = \arcsin(\sin B)$, and hence compute $\angle A$ and $\angle B$.\par
(Hint: For which angles does $\arcsin(\sin\theta) = \theta$ hold?)
\end{enumerate}
\item \emph{SSS congruence.} Let $ABC$ be a triangle with $a = 13$, $b = 14$, and $c = 15$.
\begin{enumerate}
\item Using the law of cosines, or otherwise, find $\cos A$, $\cos B$, and $\cos C$.
\item Compute $\angle A$, $\angle B$, and $\angle C$.
\item Find $[ABC]$.
\end{enumerate}
\item \emph{ASA/AAS congruence.} Let $ABC$ be a triangle with $c = 2$, $\angle A = 12^{\circ}$, and $\angle B = 77^{\circ}$.
\begin{enumerate}
\item Find $\angle C$.
\item Find $a$ and $b$.
\item Find $[ABC]$.
\end{enumerate}
\item \emph{SSA non-congruence.} Let $ABC$ be a triangle with $\angle A = 30^{\circ}$, $a = 6$, and $b = 9$.
\begin{enumerate}
\item Show that $c^2 - (9\sqrt{3})c + 45 = 0$.
\item Find all possible values of $c$.
\end{enumerate}
\item \emph{Extended law of sines.} If $ABC$ is a triangle with \href{https://en.wikipedia.org/wiki/Circumcircle}{circumradius} $R$, then the \textbf{extended law of sines} states that
\begin{equation*}
\frac{a}{\sin A} = \frac{b}{\sin B} = \frac{c}{\sin C} = 2R.
\end{equation*}
\begin{enumerate}
\item Express $\sin C$ in terms of $a$, $b$, and $[ABC]$.
\item Assuming the extended law of sines, show that $R = \dfrac{abc}{4[ABC]}$.
\item Given that $a = 13$, $b = 14$, and $c = 15$, compute $R$.
\end{enumerate}
\item \emph{Single-cevian computations.} Let $ABC$ be a triangle and let $D$ be a point on side $\overline{BC}$. (The line segment $\overline{AD}$ is a \emph{cevian} from $A$.)
\begin{enumerate}
\item Express $\angle ADB$ in terms of $\angle CDA$.
\item \emph{Ratio lemma.} Using the law of sines, or otherwise, show that
\begin{equation*}
\frac{BD}{DC} = \frac{AB}{AC}\cdot\frac{\sin(\angle BAD)}{\sin(\angle DAC)}.
\end{equation*}
\item \emph{Angle bisector theorem.} Show that if $\overline{AD}$ bisects $\angle BAC$, then $\dfrac{AB}{BD} = \dfrac{AC}{DC}$.
\item \emph{Stewart's theorem.} Let $\angle ADB = \theta$ and let $AD = d$, $BD = x$, and $DC = y$. Show that
\begin{align*}
c^2 &= d^2 + x^2 - 2dx\cos\theta, \\
b^2 &= d^2 + y^2 + 2dy\cos\theta,
\end{align*}
and conclude that
\begin{equation*}
b^2x + c^2y = a(d^2 + xy).
\end{equation*}
\end{enumerate}
\item \emph{Concurrent cevians.} Let $ABC$ be a triangle and let $D$, $E$, $F$ lie on $\overline{BC}$, $\overline{CA}$, $\overline{AB}$ respectively.
\begin{enumerate}
\item Show that
\begin{equation*}
\frac{AF}{FB}\cdot\frac{BD}{DC}\cdot\frac{CE}{EA} = \frac{\sin(\angle ACF)}{\sin(\angle FCB)}\cdot\frac{\sin(\angle BAD)}{\sin(\angle DAC)}\cdot\frac{\sin(\angle CBE)}{\sin(\angle EBA)}.
\end{equation*}
\item \emph{Ceva's theorem.} Show that $\overline{AD}$, $\overline{BE}$, and $\overline{CF}$ are concurrent if and only if the two sides of the above equation are equal to 1.
\end{enumerate}
\end{enumerate}


\subsection{Challenge problems}

\begin{enumerate}\setcounter{enumi}{7}
\item Points $O$, $A$, $B$, and $C$ are placed in the plane so that $AO = BO = CO = 4$, $AB = 2$, and $AC = 1$. Find all possible lengths of $BC$.
\item In triangle $ABC$, point $D$ lies on $\overline{BC}$ so that $\overline{AD}$ bisects $\angle BAC$. Assuming that $BD = 7$, $BA = 8$, and $AD = 5$, find $CD$.
\item (Eisenstein triples) An \emph{Eisenstein triple} is a triple of positive integers $(a,b,c)$ for which a triangle with side lengths $a$, $b$, and $c$ has an angle of measure either $60^{\circ}$ or $120^{\circ}$. If the Eisenstein triple $(a,b,c)$ corresponds to a triangle with an angle of measure $60^{\circ}$, we will call it an Eisenstein triple of \emph{acute type}, and otherwise, we call it an Eisenstein triple of \emph{obtuse type}. (The ``acute type'' and ``obtuse type'' names are non-standard.)
\begin{enumerate}
\item Let $(a,b,c)$ be an Eisenstein triple of obtuse type with $a < b < c$. Show that $(a, a + b, c)$ and $(a + b, b, c)$ are Eisenstein triples of acute type.
\item Conversely, show that every Eisenstein triple of acute type either corresponds to an equilateral triangle or arises from an Eisenstein triple of obtuse type in the above manner.
\item Show that if $(a,b,c)$ is an Eisenstein triple of obtuse type with $\gcd(a,b,c) = 1$, then there are relatively prime positive integers $m$ and $n$ such that
\begin{equation*}
\{a,b,c\} = \{m^2 + mn + n^2, 2mn + n^2, m^2 - n^2\}.
\end{equation*}
(Hint: See Section 1 Problem 10 from the Midterm 1 review.)
\end{enumerate}
\end{enumerate}


\newpage
\subsection{Answers}

\begin{enumerate}
\item \begin{enumerate}
\item $[ABC] = \frac{1}{2}ab\sin C = \frac{5}{2}\sin(104^{\circ})\approx 2.426$
\item $c = \sqrt{a^2 + b^2 - 2ab\cos C} = \sqrt{26 - 10\cos(104^{\circ})}\approx 5.331$
\item $\sin A = \frac{a\sin C}{c}\approx\frac{\sin(104^{\circ})}{5.331}\approx 0.182$\par
$\sin B = \frac{b\sin C}{c}\approx\frac{5\sin(104^{\circ})}{5.331}\approx 0.910$
\item Since $\angle C$ is obtuse, $\angle A$ and $\angle B$ are acute. Since acute angles are included in the range of $\arcsin$, we have $\angle A = \arcsin(\sin A)$ and $\angle B = \arcsin(\sin B)$.\par
$\angle A = \arcsin(\sin A)\approx\arcsin(0.182)\approx 10.49^{\circ}$\par 
$\angle B = \arcsin(\sin B)\approx\arcsin(0.910)\approx 65.51^{\circ}$
\end{enumerate}
\item \begin{enumerate}
\item $\cos A = \frac{b^2 + c^2 - a^2}{2bc} = \frac{3}{5}$\par
$\cos B = \frac{c^2 + a^2 - b^2}{2ca} = \frac{33}{65}$\par
$\cos C = \frac{a^2 + b^2 - c^2}{2ab} = \frac{5}{13}$
\item The range of $\arccos$ is $[0^{\circ}, 180^{\circ}]$, so we can always use it to extract triangle angles.\par
$\angle A = \arccos(\cos A) = \arccos(\frac{3}{5})\approx 53.13^{\circ}$\par
$\angle B = \arccos(\cos B) = \arccos(\frac{33}{65})\approx 59.49^{\circ}$\par
$\angle C = \arccos(\cos C) = \arccos(\frac{5}{13})\approx 67.38^{\circ}$
\item $[ABC] = \frac{1}{2}bc\sin A = \frac{14\cdot 15}{2}\sin(\arccos(\frac{3}{5})) = 7\cdot 15\cdot\frac{4}{5} = 84$
\end{enumerate}
\item \begin{enumerate}
\item $\angle C = 180^{\circ} - \angle A - \angle B = 91^{\circ}$
\item $a = \frac{c}{\sin C}\cdot\sin A = \frac{2\sin 12^{\circ}}{\sin 91^{\circ}}\approx 0.416$\par
$b = \frac{c}{\sin C}\cdot\sin B = \frac{2\sin 77^{\circ}}{\sin 91^{\circ}}\approx 1.949$
\item $[ABC] = \frac{1}{2}ac\sin B\approx 0.416\sin 77^{\circ}\approx 0.405$
\end{enumerate}
\item \begin{enumerate}
\item By the law of cosines,
\begin{equation*}
a^2 = b^2 + c^2 - 2bc\cos A\implies 36 = 81 + c^2 - 18\cos(30^{\circ})c.
\end{equation*}
Evaluating $\cos(30^{\circ}) = \sqrt{3}/2$ and rearranging gives us $c^2 - (9\sqrt{3})c + 45 = 0$.
\item By the quadratic formula,
\begin{equation*}
c = \frac{9\sqrt{3}\pm\sqrt{(9\sqrt{3})^2 - 4\cdot 1\cdot 45}}{2} = \frac{9\sqrt{3}\pm 3\sqrt{7}}{2}.
\end{equation*}
\end{enumerate}
\item \begin{enumerate}
\item $\sin C = \frac{2[ABC]}{ab}$.
\item $R = \dfrac{c}{2\sin C} = \dfrac{c}{\frac{4[ABC]}{ab}} = \dfrac{abc}{4[ABC]}$
\item $R = 65/8$
\end{enumerate}
\item \begin{enumerate}
\item $\angle ADB = 180^{\circ} - \angle CDA$
\item By the law of sines,
\begin{align*}
\frac{AB}{\sin(\angle ADB)} = \frac{BD}{\sin(\angle BAD)}\quad\text{and}\quad\frac{AC}{\sin(\angle CDA)} = \frac{DC}{\sin(\angle DAC)}.
\end{align*}
Dividing one equation by the other and using the fact that $\sin(\angle ADB) = \sin(\angle CDA)$ gets us the desired result after rearranging.
\item When $\overline{AD}$ bisects $\angle BAC$, we have $\angle BAD = \angle DAC$, so the sines cancel in part (b).
\item The law of cosines in triangle $ADB$, using $\angle ADB$, gives us
\begin{equation*}
(AB)^2 = (AD)^2 + (BD)^2 - 2(AD)(BD)\cos(\angle ADB)\implies c^2 = d^2 + x^2 - 2dx\cos\theta,
\end{equation*}
while the law of cosines in triangle $ADC$, using $\angle CDA$, gives us
\begin{equation*}
b^2 = d^2 + y^2 - 2dy\cos(180^{\circ} - \theta) = d^2 + y^2 + 2dy\cos\theta.
\end{equation*}
Adding $y$ times the first equation to $x$ times the second equation, so as to eliminate $\cos\theta$,
\begin{align*}
b^2x + c^2y &= (d^2x + y^2 + 2dy\cos\theta)x + (d^2y + x^2 - 2dx\cos\theta)y \\
&= d^2(x + y) + xy(x + y) = a(d^2 + xy).
\end{align*}
\end{enumerate}
\item \begin{enumerate}
\item By the ratio lemma (problem 6b),
\begin{align*}
\frac{AF}{FB} &= \frac{CA}{CB}\cdot\frac{\sin(\angle ACF)}{\sin(\angle FCB)}, \\
\frac{BD}{DC} &= \frac{AB}{AC}\cdot\frac{\sin(\angle BAD)}{\sin(\angle DAC)}, \\
\frac{CE}{EA} &= \frac{BC}{BA}\cdot\frac{\sin(\angle CBE)}{\sin(\angle EBA)}.
\end{align*}
Multiplying these equations together gives us the desired result.
\item First suppose $\overline{AD}$, $\overline{BE}$, and $\overline{CF}$ concur at point $P$. Then by the law of sines,
\begin{align*}
\frac{\sin(\angle ACF)}{\sin(\angle DAC)} &= \frac{\sin(\angle ACP)}{\sin(\angle PAC)} = \frac{AP}{CP}, \\
\frac{\sin(\angle BAD)}{\sin(\angle EBA)} &= \frac{\sin(\angle BAP)}{\sin(\angle PBA)} = \frac{BP}{AP}, \\
\frac{\sin(\angle CBE)}{\sin(\angle FCB)} &= \frac{\sin(\angle CBP)}{\sin(\angle PCB)} = \frac{CP}{BP}.
\end{align*}
Multiplying these equations,
\begin{align*}
\frac{\sin(\angle ACF)}{\sin(\angle FCB)}\cdot\frac{\sin(\angle BAD)}{\sin(\angle DAC)}\cdot\frac{\sin(\angle CBE)}{\sin(\angle EBA)} &= \frac{\sin(\angle ACF)}{\sin(\angle DAC)}\cdot\frac{\sin(\angle BAD)}{\sin(\angle EBA)}\cdot\frac{\sin(\angle CBE)}{\sin(\angle FCB)} \\
&= \frac{AP}{CP}\cdot\frac{BP}{AP}\cdot\frac{CP}{BP} = 1.
\end{align*}
Conversely, suppose both sides of the equation from part (a) are 1, so in particular
\begin{equation*}
\frac{AF}{FB}\cdot\frac{BD}{DC}\cdot\frac{CE}{EA} = 1.
\end{equation*}
Let $\overline{AD}$ and $\overline{BE}$ intersect at point $Q$, and let line $\overline{CQ}$ intersect side $\overline{AB}$ at point $F'$. Then using what we just showed,
\begin{equation*}
\frac{AF'}{F'B}\cdot\frac{BD}{DC}\cdot\frac{CE}{EA} = 1.
\end{equation*}
This means $AF/FB = AF'/F'B$. With $F$ and $F'$ both interior to segment $\overline{AB}$, this can only happen if $F = F'$, which means that $\overline{AD}$, $\overline{BE}$, and $\overline{CF}$ concur (at $Q$) as desired.
\end{enumerate}
\item Fix points $A$ and $B$ on a circle of radius $4$ centered at $O$ so that $AB = 2$. By the law of cosines, we can find
\begin{equation*}
\cos(\angle AOB) = \frac{7}{8}\quad\text{and}\quad\cos(\angle AOC) = \frac{31}{32},
\end{equation*}
from which we find
\begin{equation*}
\sin(\angle AOB) = \frac{\sqrt{15}}{8}\quad\text{and}\quad\sin(\angle AOC) = \frac{3\sqrt{7}}{32}.
\end{equation*}
Since $CO = 4$, we know $C$ lies on this circle as well, and since $AC = 1$, there are two possible locations for $C$, one on either side of $\overline{OA}$. When $C$ and $B$ lie on the same side of $\overline{OA}$, we have $\angle BOC = \angle AOB - \angle AOC$, which gives us
\begin{align*}
BC &= \sqrt{32 - 32\cos(\angle AOB - \angle AOC)} \\
&= 4\sqrt{2 - 2\left(\frac{7}{8}\cdot\frac{31}{32} + \frac{\sqrt{15}}{8}\cdot\frac{3\sqrt{7}}{32}\right)} \\
&= 4\sqrt{2 - \frac{217 + 3\sqrt{105}}{128}}\approx 1.016.
\end{align*}
When $C$ and $B$ lie on opposite sides of $\overline{OA}$, we instead have $\angle BOC = \angle BOA + \angle AOC$. A similar calculation to the first case yields
\begin{equation*}
BC = 4\sqrt{2 - \frac{217 - 3\sqrt{105}}{128}}\approx 2.953.
\end{equation*}
\item Let $CD = 7x$, so that $AC = 8x$ by the angle bisector theorem. From the law of cosines,
\begin{equation*}
\cos(\angle BAD) = \frac{8^2 + 5^2 - 7^2}{2\cdot 8\cdot 5} = \frac{1}{2},
\end{equation*}
so $\cos(\angle DAC) = 1/2$ as well. Using the law of cosines at $\angle DAC$ gives us
\begin{equation*}
(7x)^2 = (8x)^2 + 5^2 - 2\cdot 8x\cdot 5\cdot\frac{1}{2}\implies 15x^2 - 40x + 25 = 0.
\end{equation*}
This quadratic factors as $5(3x - 5)(x - 1)$, so there are two solutions, $x = 1$ or $x = 5/3$. When $x = 1$, we end up with $\triangle DAB\cong\triangle DAC$. However, this together with $D$ lying on segment $\overline{BC}$ implies that $\angle ADB = 90^{\circ}$, a contradiction. Hence the only valid solution is that $x = 5/3$, in which case $CD = 35/3$.
\item Suppose without loss of generality that $c$ is the side opposite the $120^{\circ}$ angle, so that by the law of cosines,
\begin{equation*}
c^2 = a^2 + b^2 - 2ab\cos 120^{\circ} = a^2 + ab + b^2.
\end{equation*}
Dividing through by $c^2$ and letting $x = a/c$ and $y = b/c$, finding Eisenstein triples is equivalent to finding points with positive rational coordinates on the conic
\begin{equation*}
x^2 + xy + y^2 = 1.
\end{equation*}
Graphing the conic, we see that it is an ellipse passing through the points $(\pm 1, 0)$ and $(0, \pm 1)$, and that every point with positive rational coordinates can be connected to $(0,-1)$ by a line of rational slope greater than 1.\par
Let $t = m/n$ be a rational number greater than 1, where $m$ and $n$ are relatively prime positive integers. The line of slope $t$ through $(0,-1)$ is $y = tx - 1$, so to find the other point where the line intersects the conic, we substitute to get the equation
\begin{align*}
x^2 + x\cdot (tx - 1) + (tx - 1)^2 &= 1, \\
(t^2 + t + 1)x^2 - (2t + 1)x &= 0.
\end{align*}
One solution is $x = 0$, corresponding to $y = -1$, and the other solution is
\begin{equation*}
x = \frac{2t + 1}{t^2 + t + 1},
\end{equation*}
corresponding to
\begin{equation*}
y = tx - 1 = \frac{t^2 - 1}{t^2 + t + 1}.
\end{equation*}
Substituting $t = m/n$ and clearing nested denominators gives us
\begin{equation*}
(x,y) = \left(\frac{a}{c},\frac{b}{c}\right) = \left(\frac{2mn + n^2}{m^2 + mn + n^2}, \frac{m^2 - n^2}{m^2 + mn + n^2}\right).
\end{equation*}
To finish, we need to check whether the fractions on the right hand side are fully reduced. To start, since $\gcd(m,n) = 1$,
\begin{align*}
\gcd(2mn + n^2, m^2 + mn + n^2) &= \gcd(n\cdot (2m + n), m^2 + mn + n^2) \\
&= \gcd(2m + n, m^2 + mn + n^2) \\
&= \gcd(2m + n, m^2 + mn + n^2 - n\cdot (2m + n)) \\
&= \gcd(2m + n, m^2 - mn) = \gcd(2m + n, m\cdot (m - n)) \\
&= \gcd(2m + n, m - n) = \gcd(3n, m - n).
\end{align*}
If $m\equiv n\pmod{3}$, then let $m = n + 3k$. Then $\gcd(n,k) = 1$ and
\begin{equation*}
\gcd(3n, m - n) = \gcd(3n,3k) = 3\gcd(n,k) = 3.
\end{equation*}
Otherwise,
\begin{equation*}
\gcd(3n, m - n) = \gcd(n, m - n) = \gcd(n,m) = 1.
\end{equation*}
Thus we are done in the case that $m\neq n\pmod{3}$, while in the case that $m\equiv n\pmod{3}$,
\begin{equation*}
a = \frac{2mn + n^2}{3},\quad b = \frac{m^2 - n^2}{3},\quad c = \frac{m^2 + mn + n^2}{3}.
\end{equation*}
Let $r = \frac{m + 2n}{3}$ and $s = \frac{m - n}{3}$, so that $n = r - s$ and $m = r + 2s$. Then
\begin{align*}
a &= \frac{2(r + 2s)(r - s) + (r - s)^2}{3} = r^2 - s^2, \\
b &= \frac{(r + 2s)^2 - (r - s)^2}{3} = 2rs + s^2, \\
c &= \frac{(r + 2s)^2 + (r + 2s)(r - s) + (r - s)^2}{3} = r^2 + rs + s^2,
\end{align*}
so the result still holds with $r$ and $s$ in place of $m$ and $n$.
\end{enumerate}
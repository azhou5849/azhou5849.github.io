\section{Laws of Sines and Cosines}

\subsection{Review problems}

Calculators are recommended for this section. Throughout, if $ABC$ is a triangle, then we use $a$, $b$, and $c$ to denote the side lengths $BC$, $CA$, and $AB$, respectively. (That is, $a$ is the length of the side opposite $A$, etc.) The notation $[ABC]$ denotes the area of $ABC$.

\begin{enumerate}
\item \emph{SAS congruence.} Let $ABC$ be a triangle with $a = 1$, $b = 5$, and $\angle C = 104^{\circ}$.
\begin{enumerate}
\item Find $[ABC]$.
\item Find $c$.
\item Using the law of sines, or otherwise, find $\sin A$ and $\sin B$.
\item Show that $\angle A = \arcsin(\sin A)$ and $\angle B = \arcsin(\sin B)$, and hence compute $\angle A$ and $\angle B$.\par
(Hint: For which angles does $\arcsin(\sin\theta) = \theta$ hold?)
\end{enumerate}
\item \emph{SSS congruence.} Let $ABC$ be a triangle with $a = 13$, $b = 14$, and $c = 15$.
\begin{enumerate}
\item Using the law of cosines, or otherwise, find $\cos A$, $\cos B$, and $\cos C$.
\item Compute $\angle A$, $\angle B$, and $\angle C$.
\item Find $[ABC]$.
\end{enumerate}
\item \emph{ASA/AAS congruence.} Let $ABC$ be a triangle with $c = 2$, $\angle A = 12^{\circ}$, and $\angle B = 77^{\circ}$.
\begin{enumerate}
\item Find $\angle C$.
\item Find $a$ and $b$.
\item Find $[ABC]$.
\end{enumerate}
\item \emph{SSA non-congruence.} Let $ABC$ be a triangle with $\angle A = 30^{\circ}$, $a = 6$, and $b = 9$.
\begin{enumerate}
\item Show that $c^2 - (9\sqrt{3})c + 45 = 0$.
\item Find all possible values of $c$.
\end{enumerate}
\item \emph{Extended law of sines.} If $ABC$ is a triangle with \href{https://en.wikipedia.org/wiki/Circumcircle}{circumradius} $R$, then the \textbf{extended law of sines} states that
\begin{equation*}
\frac{a}{\sin A} = \frac{b}{\sin B} = \frac{c}{\sin C} = 2R.
\end{equation*}
\begin{enumerate}
\item Prove that $R = \dfrac{abc}{4[ABC]}$.
\item Given that $a = 13$, $b = 14$, and $c = 15$, find $R$.
\item Prove the extended law of sines for acute triangles.
\end{enumerate}
\item Let $ABC$ be a triangle and let $D$ be a point on side $\overline{BC}$.
\begin{enumerate}
\item \emph{Ratio lemma.} Prove that
\begin{equation*}
\frac{BD}{DC} = \frac{AB}{AC}\cdot\frac{\sin(\angle BAD)}{\sin(\angle DAC)}.
\end{equation*}
\item \emph{Angle bisector theorem.} Show that if $\overline{AD}$ bisects $\angle BAC$, then $\dfrac{AB}{BD} = \dfrac{AC}{DC}$.
\end{enumerate}
\item \emph{Heron's formula.} Let $ABC$ be a triangle.
\begin{enumerate}
\item Show that
\begin{equation*}
[ABC]^2 = \frac{1}{4}a^2b^2(1 - \cos^2 C) = \frac{4a^2b^2 - (a^2 + b^2 - c^2)^2}{16}.
\end{equation*}
\item Conclude that
\begin{equation*}
[ABC] = \sqrt{s(s - a)(s - b)(s - c)},
\end{equation*}
where $s = (a + b + c)/2$ is the \emph{semiperimeter} of triangle $ABC$.
\end{enumerate}
\end{enumerate}


\subsection{Challenge problems}

\begin{enumerate}\setcounter{enumi}{7}
\item Points $O$, $A$, $B$, and $C$ are placed in three-dimensional space so that $AO = BO = CO = 4$, $AB = 2$, and $AC = 1$. What are the shortest and longest possible lengths of $BC$?
\item In triangle $ABC$, point $D$ lies on $\overline{BC}$ so that $\overline{AD}$ bisects $\angle BAC$. Given that $BD = 7$, $BA = 8$, and $AD = 5$, find $CD$.
\item (Eisenstein triples) An \emph{Eisenstein triple} is a triple of positive integers $(a,b,c)$ for which a triangle with side lengths $a$, $b$, and $c$ has an angle of measure either $60^{\circ}$ or $120^{\circ}$. If the Eisenstein triple $(a,b,c)$ corresponds to a triangle with an angle of measure $60^{\circ}$, we will call it an Eisenstein triple of \emph{acute type}, and otherwise, we call it an Eisenstein triple of \emph{obtuse type}. (The ``acute type'' and ``obtuse type'' names are non-standard.)
\begin{enumerate}
\item Let $(a,b,c)$ be an Eisenstein triple of obtuse type with $a < b < c$. Show that $(a, a + b, c)$ and $(a + b, b, c)$ are Eisenstein triples of acute type.
\item Conversely, show that every Eisenstein triple of acute type arises from an Eisenstein triple of obtuse type in the above manner.
\item Show that if $(a,b,c)$ is an Eisenstein triple of obtuse type with $\gcd(a,b,c) = 1$, then there are relatively prime positive integers $m$ and $n$ such that
\begin{equation*}
\{a,b,c\} = \{m^2 + mn + n^2, 2mn + n^2, m^2 - n^2\}.
\end{equation*}
(Hint: See Section 1 Problem 10 from the Midterm 1 review.)
\end{enumerate}
\end{enumerate}


\newpage
\subsection{Answers}

\begin{enumerate}
\item \begin{enumerate}
\item $[ABC] = \frac{1}{2}ab\sin C = \frac{5}{2}\sin(104^{\circ})\approx 2.426$
\item $c = \sqrt{a^2 + b^2 - 2ab\cos C} = \sqrt{26 - 10\cos(104^{\circ})}\approx 5.331$
\item $\angle A = \arcsin(\frac{a\sin C}{c})\approx 10.49^{\circ}$\par 
$\angle B = \arcsin(\frac{b\sin C}{c})\approx 65.51^{\circ}$\par
\emph{These angles can also be found with the law of cosines.}
\end{enumerate}
\item \begin{enumerate}
\item $\angle A = \arccos(\frac{b^2 + c^2 - a^2}{2bc}) = \arccos(\frac{3}{5})\approx 53.13^{\circ}$
\item $\angle B = \arccos(\frac{a^2 + c^2 - b^2}{2ac}) = \arccos(\frac{33}{65})\approx 59.49^{\circ}$\par 
$\angle C = \arccos(\frac{a^2 + b^2 - c^2}{2ab}) = \arccos(\frac{5}{13})\approx 67.38^{\circ}$\par
\emph{These angles can also be found with the law of sines.}
\item $[ABC] = \frac{1}{2}bc\sin A = \frac{14\cdot 15}{2}\sin(\arccos(\frac{3}{5})) = 7\cdot 15\cdot\frac{4}{5} = 84$
\end{enumerate}
\item \begin{enumerate}
\item $\angle C = 91^{\circ}$
\item $a = \frac{c}{\sin C}\cdot\sin A = \frac{2\sin 12^{\circ}}{\sin 91^{\circ}}\approx 0.416$\par
$b = \frac{c}{\sin C}\cdot\sin B = \frac{2\sin 77^{\circ}}{\sin 91^{\circ}}\approx 1.949$
\item $[ABC] = \frac{1}{2}ac\sin B = \frac{2\sin 12^{\circ}\sin 77^{\circ}}{\sin 91^{\circ}}\approx 0.405$
\end{enumerate}
\item \begin{enumerate}
\item By the law of cosines,
\begin{equation*}
a^2 = b^2 + c^2 - 2bc\cos A\implies 36 = 81 + c^2 - (18\cos 20^{\circ})c.
\end{equation*}
Solving the resulting quadratic yields
\begin{equation*}
c = \frac{18\cos 20^{\circ}\pm\sqrt{324\cos^2(20^{\circ}) - 180}}{2} = 9\cos 20^{\circ}\pm 3\sqrt{9\cos^2(20^{\circ}) - 5}.
\end{equation*}
One solution is $\approx 3.307$ and the other solution is $\approx 13.607$.
\item When $c\approx 3.307$, we have $\angle B = \arccos(\frac{a^2 + c^2 - b^2}{2ac})\approx 149.13^{\circ}$.\par 
When $c\approx 13.607$, we have $\angle B = \arccos(\frac{a^2 + c^2 - b^2}{2ac})\approx 30.87^{\circ}$.
\item Let $y = XZ$ be the missing side length. By the law of cosines,
\begin{equation*}
x^2 = 81 + y^2 - (18\cos 20^{\circ})y\implies y^2 - (18\cos 20^{\circ})y + (81 - x^2) = 0.
\end{equation*}
For there to be only one triangle with the given properties, there must be exactly one positive solution for $y$. This can occur in two ways.
\begin{description}
\item[Case 1 (exactly one real solution, which is positive).] If there is exactly one real solution, then it must be $y = 9\cos 20^{\circ}$, which is positive as required. This situation occurs when $81 - x^2 = (9\cos 20^{\circ})^2 = 81\cos^2(20^{\circ})$, which holds when $x = 9\sin 20^{\circ}$. (This corresponds to ``HL congruence.'')
\item[Case 2 (two real solutions, only one of which is positive).] The quadratic has a leading coefficient of 1, so this situation occurs precisely when the constant term is negative. Thus we need $81 - x^2 < 0$, and since $x$ is a side length, we have $x > 9$.
\end{description}
\end{enumerate}
\item \begin{enumerate}
\item From $[ABC] = \frac{1}{2}ab\sin C$, we have $\sin C = \frac{2[ABC]}{ab}$. Then,
\begin{equation*}
R = \frac{c}{2\sin C} = \frac{c}{\frac{4[ABC]}{ab}} = \frac{abc}{4[ABC]}.
\end{equation*}
\item $R = 65/8$
\item See \href{https://artofproblemsolving.com/wiki/index.php/Law_of_Sines}{this link}.
\end{enumerate}
\item \begin{enumerate}
\item We have
\begin{align*}
[ABD] &= \frac{1}{2}\cdot AB\cdot AD\cdot\sin(\angle BAD), \\
[ADC] &= \frac{1}{2}\cdot AC\cdot AD\cdot\sin(\angle DAC),
\end{align*}
and dividing the two equations yields
\begin{equation*}
\frac{[ABD]}{[ADC]} = \frac{AB}{AC}\cdot\frac{\sin(\angle BAD)}{\sin(\angle DAC)}.
\end{equation*}
The conclusion follows from the fact that triangles $ABD$ and $ADC$ share a height from $A$, so that then $\frac{[ABD]}{[ADC]} = \frac{BD}{DC}$.
\item When $\overline{AD}$ bisects $\angle BAC$, we have $\angle BAD = \angle DAC$, so the sines cancel in part (a).
\end{enumerate}
\item \begin{enumerate}
\item We compute
\begin{align*}
[ABC]^2 &= \left(\frac{1}{2}ab\sin C\right)^2 = \frac{1}{4}a^2b^2\sin^2 C \\
&= \frac{1}{4}a^2b^2(1 - \cos^2 C) \\
&= \frac{1}{4}a^2b^2\left[1 - \left(\frac{a^2 + b^2 - c^2}{2ab}\right)^2\right] \\
&= \frac{1}{4}a^2b^2\left[\frac{4a^2b^2 - (a^2 + b^2 - c^2)^2}{4a^2b^2}\right] \\
&= \frac{4a^2b^2 - (a^2 + b^2 - c^2)^2}{16}.
\end{align*}
\item From here, we observe some differences of squares to obtain
\begin{align*}
[ABC]^2 &= \frac{[2ab - (a^2 + b^2 - c^2)][2ab + (a^2 + b^2 - c^2)]}{16} \\
&= \frac{[c^2 - (a^2 - 2ab + b^2)][(a^2 + 2ab + b^2) - c^2]}{16} \\
&= \frac{[c - (a - b)][c + (a - b)][(a + b) - c][(a + b) + c]}{16} \\
&= \frac{a + b + c}{2}\cdot\frac{b + c - a}{2}\cdot\frac{a + c - b}{2}\cdot\frac{a + b - c}{2} \\
&= s(s - a)(s - b)(s - c).
\end{align*}
\end{enumerate}
\item By the law of cosines, we can find
\begin{equation*}
\cos(\angle AOB) = \frac{7}{8}\quad\text{and}\quad\cos(\angle AOC) = \frac{31}{32},
\end{equation*}
from which we find
\begin{equation*}
\sin(\angle AOB) = \frac{\sqrt{15}}{8}\quad\text{and}\quad\sin(\angle AOC) = \frac{3\sqrt{7}}{32}.
\end{equation*}
The smallest possible value of $\angle BOC$ is $\angle AOB - \angle AOC$, so the smallest possible $BC$ is
\begin{align*}
\min BC &= \sqrt{32 - 32\cos(\angle AOB - \angle AOC)} \\
&= 4\sqrt{2 - 2\left(\frac{7}{8}\cdot\frac{31}{32} + \frac{\sqrt{15}}{8}\cdot\frac{3\sqrt{7}}{32}\right)} \\
&= 4\sqrt{2 - \frac{217 + 3\sqrt{105}}{128}}\approx 1.016.
\end{align*}
By a similar argument, the largest possible $BC$ is
\begin{equation*}
\max BC = 4\sqrt{2 - \frac{217 - 3\sqrt{105}}{128}}\approx 2.953.
\end{equation*}
\item Let $CD = 7x$, so that $AC = 8x$ by the angle bisector theorem. From the law of cosines,
\begin{equation*}
\cos(\angle BAD) = \frac{8^2 + 5^2 - 7^2}{2\cdot 8\cdot 5} = \frac{1}{2},
\end{equation*}
so $\cos(\angle DAC) = 1/2$ as well. Using the law of cosines at $\angle DAC$ gives us
\begin{equation*}
(7x)^2 = (8x)^2 + 5^2 - 2\cdot 8x\cdot 5\cdot\frac{1}{2}\implies 15x^2 - 40x + 25 = 0.
\end{equation*}
This quadratic factors as $5(3x - 5)(x - 1)$, so there are two solutions, $x = 1$ or $x = 5/3$. When $x = 1$, we end up with $\triangle DAB\cong\triangle DAC$. However, this together with $D$ lying on segment $\overline{BC}$ implies that $\angle ADB = 90^{\circ}$, a contradiction. Hence the only valid solution is that $x = 5/3$, in which case $CD = 35/3$.
\item Suppose without loss of generality that $c$ is the side opposite the $120^{\circ}$ angle, so that by the law of cosines,
\begin{equation*}
c^2 = a^2 + b^2 - 2ab\cos 120^{\circ} = a^2 + ab + b^2.
\end{equation*}
Dividing through by $c^2$ and letting $x = a/c$ and $y = b/c$, finding Eisenstein triples is equivalent to finding points with positive rational coordinates on the conic
\begin{equation*}
x^2 + xy + y^2 = 1.
\end{equation*}
Graphing the conic, we see that it is an ellipse passing through the points $(\pm 1, 0)$ and $(0, \pm 1)$, and that every point with positive rational coordinates can be connected to $(0,-1)$ by a line of rational slope greater than 1.\par
Let $t = m/n$ be a rational number greater than 1, where $m$ and $n$ are relatively prime positive integers. The line of slope $t$ through $(0,-1)$ is $y = tx - 1$, so to find the other point where the line intersects the conic, we substitute to get the equation
\begin{align*}
x^2 + x\cdot (tx - 1) + (tx - 1)^2 &= 1, \\
(t^2 + t + 1)x^2 - (2t + 1)x &= 0.
\end{align*}
One solution is $x = 0$, corresponding to $y = -1$, and the other solution is
\begin{equation*}
x = \frac{2t + 1}{t^2 + t + 1},
\end{equation*}
corresponding to
\begin{equation*}
y = tx - 1 = \frac{t^2 - 1}{t^2 + t + 1}.
\end{equation*}
Substituting $t = m/n$ and clearing nested denominators gives us
\begin{equation*}
(x,y) = \left(\frac{a}{c},\frac{b}{c}\right) = \left(\frac{2mn + n^2}{m^2 + mn + n^2}, \frac{m^2 - n^2}{m^2 + mn + n^2}\right).
\end{equation*}
To finish, we need to check whether the fractions on the right hand side are fully reduced. To start, since $\gcd(m,n) = 1$,
\begin{align*}
\gcd(2mn + n^2, m^2 + mn + n^2) &= \gcd(n\cdot (2m + n), m^2 + mn + n^2) \\
&= \gcd(2m + n, m^2 + mn + n^2) \\
&= \gcd(2m + n, m^2 + mn + n^2 - n\cdot (2m + n)) \\
&= \gcd(2m + n, m^2 - mn) = \gcd(2m + n, m\cdot (m - n)) \\
&= \gcd(2m + n, m - n) = \gcd(3n, m - n).
\end{align*}
If $m\equiv n\pmod{3}$, then let $m = n + 3k$. Then $\gcd(n,k) = 1$ and
\begin{equation*}
\gcd(3n, m - n) = \gcd(3n,3k) = 3\gcd(n,k) = 3.
\end{equation*}
Otherwise,
\begin{equation*}
\gcd(3n, m - n) = \gcd(n, m - n) = \gcd(n,m) = 1.
\end{equation*}
Thus we are done in the case that $m\neq n\pmod{3}$, while in the case that $m\equiv n\pmod{3}$,
\begin{equation*}
a = \frac{2mn + n^2}{3},\quad b = \frac{m^2 - n^2}{3},\quad c = \frac{m^2 + mn + n^2}{3}.
\end{equation*}
Let $r = \frac{m + 2n}{3}$ and $s = \frac{m - n}{3}$, so that $n = r - s$ and $m = r + 2s$. Then
\begin{align*}
a &= \frac{2(r + 2s)(r - s) + (r - s)^2}{3} = r^2 - s^2, \\
b &= \frac{(r + 2s)^2 - (r - s)^2}{3} = 2rs + s^2, \\
c &= \frac{(r + 2s)^2 + (r + 2s)(r - s) + (r - s)^2}{3} = r^2 + rs + s^2,
\end{align*}
so the result still holds with $r$ and $s$ in place of $m$ and $n$.
\end{enumerate}
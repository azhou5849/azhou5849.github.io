\section{Laws of Sines and Cosines}

\subsection{Review problems}

Calculators are recommended for this section. Throughout, if $ABC$ is a triangle, then we use $a$, $b$, and $c$ to denote the side lengths $BC$, $CA$, and $AB$, respectively. (That is, $a$ is the length of the side opposite $A$, etc.) The notation $[ABC]$ denotes the area of $ABC$.

\begin{enumerate}
\item (SAS congruence) Let $ABC$ be a triangle with $a = 1$, $b = 5$, and $\angle C = 104^{\circ}$.
\begin{enumerate}
\item Find $[ABC]$.
\item Find $c$.
\item Find $\angle A$ and $\angle B$.
\end{enumerate}
\item (SSS congruence) Let $ABC$ be a triangle with $a = 13$, $b = 14$, and $c = 15$.
\begin{enumerate}
\item Find $\angle A$.
\item Find $\angle B$ and $\angle C$.
\item Find $[ABC]$.
\end{enumerate}
\item (ASA/AAS congruence) Let $ABC$ be a triangle with $c = 2$, $\angle A = 12^{\circ}$, and $\angle B = 77^{\circ}$.
\begin{enumerate}
\item Find $\angle C$.
\item Find $a$ and $b$.
\item Find $[ABC]$.
\end{enumerate}
\item (SSA non-congruence) Let $ABC$ be a triangle with $\angle A = 20^{\circ}$, $a = 6$, and $b = 9$.
\begin{enumerate}
\item Find all possible values of $c$.
\item For each possible value of $c$, find $\angle B$.
\item For what values of $x$ does there exist exactly one triangle $XYZ$ with $\angle X = 20^{\circ}$, $XY = 9$, and $YZ = x$?
\end{enumerate}
\item (Extended law of sines) If $ABC$ is a triangle with \href{https://en.wikipedia.org/wiki/Circumcircle}{circumradius} $R$, then the \emph{extended law of sines} states that
\begin{equation*}
\frac{a}{\sin A} = \frac{b}{\sin B} = \frac{c}{\sin C} = 2R.
\end{equation*}
\begin{enumerate}
\item Prove that $R = \dfrac{abc}{4[ABC]}$.
\item Given that $a = 13$, $b = 14$, and $c = 15$, find $R$.
\item Prove the extended law of sines for acute triangles.
\end{enumerate}
\item Let $ABC$ be a triangle and let $D$ be a point on side $\overline{BC}$.
\begin{enumerate}
\item (Ratio lemma) Prove that
\begin{equation*}
\frac{BD}{DC} = \frac{AB}{AC}\cdot\frac{\sin(\angle BAD)}{\sin(\angle DAC)}.
\end{equation*}
\item (Angle bisector theorem) Show that if $\overline{AD}$ bisects $\angle BAC$, then $\dfrac{AB}{BD} = \dfrac{AC}{DC}$.
\end{enumerate}
\item (Heron's formula) Let $ABC$ be a triangle.
\begin{enumerate}
\item Show that $\cos C = \dfrac{a^2 + b^2 - c^2}{2ab}$.
\item Show that
\begin{equation*}
[ABC]^2 = \frac{1}{4}a^2b^2(1 - \cos^2 C) = \frac{4a^2b^2 - (a^2 + b^2 - c^2)^2}{16}.
\end{equation*}
\item Conclude that
\begin{equation*}
[ABC] = \sqrt{s(s - a)(s - b)(s - c)},
\end{equation*}
where $s = (a + b + c)/2$ is the \emph{semiperimeter} of triangle $ABC$.
\end{enumerate}
\end{enumerate}


\subsection{Challenge problems}

\begin{enumerate}\setcounter{enumi}{7}
\item Points $O$, $A$, $B$, and $C$ are placed in three-dimensional space so that $AO = BO = CO = 4$, $AB = 2$, and $AC = 1$. What are the shortest and longest possible lengths of $BC$?
\item In triangle $ABC$, point $D$ lies on $\overline{BC}$ so that $\overline{AD}$ bisects $\angle BAC$. Given that $BD = 7$, $BA = 8$, and $AD = 5$, find $CD$.
\item (Eisenstein triples) An \emph{Eisenstein triple} is a triple of positive integers $(a,b,c)$ for which a triangle with side lengths $a$, $b$, and $c$ has an angle of measure either $60^{\circ}$ or $120^{\circ}$. If the Eisenstein triple $(a,b,c)$ corresponds to a triangle with an angle of measure $60^{\circ}$, we will call it an Eisenstein triple of \emph{acute type}, and otherwise, we call it an Eisenstein triple of \emph{obtuse type}. (The ``acute type'' and ``obtuse type'' names are non-standard.)
\begin{enumerate}
\item Let $(a,b,c)$ be an Eisenstein triple of obtuse type with $a < b < c$. Show that $(a, a + b, c)$ and $(a + b, b, c)$ are Eisenstein triples of acute type.
\item Conversely, show that every Eisenstein triple of acute type arises from an Eisenstein triple of obtuse type in the above manner.
\item Show that if $(a,b,c)$ is an Eisenstein triple of obtuse type with $\gcd(a,b,c) = 1$, then there are relatively prime positive integers $m$ and $n$ such that
\begin{equation*}
\{a,b,c\} = \{m^2 + mn + n^2, 2mn + n^2, m^2 - n^2\}.
\end{equation*}
(Hint: See Section 1 Problem 10 from the Midterm 1 review.)
\end{enumerate}
\end{enumerate}


\newpage
\subsection{Answers}

\begin{enumerate}
\item \begin{enumerate}
\item $[ABC] = \frac{1}{2}ab\sin C = \frac{5}{2}\sin(104^{\circ})\approx 2.426$
\item $c = \sqrt{a^2 + b^2 - 2ab\cos C} = \sqrt{26 - 10\cos(104^{\circ})}\approx 5.331$
\item $A = \arcsin(\frac{a\sin C}{c})\approx 10.49^{\circ}$ and $B = \arcsin(\frac{b\sin C}{c})\approx 65.51^{\circ}$
\end{enumerate}
\item 
\end{enumerate}
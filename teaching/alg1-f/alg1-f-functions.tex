\section{Functions}

Throughout, the notation $f:D\to\mathbb{R}$ means that $f$ is a function whose domain is $D$ and whose output values are real numbers. By the ``natural domain'' of a formula, we mean the largest subset of the reals which can be the domain of a function given by that formula. For example, the natural domain of $1/x$ is the set of all real numbers other than $0$.

\subsection{Review problems}

\begin{enumerate}
\item Let $f:\mathbb{R}\to\mathbb{R}$ be the function given by $f(x) = x^2 - 6$. Evaluate each of the following:
\begin{enumerate}
\item $f(0)$
\item $f(f(1))$
\item $f^{6}(3)$
\item $(f(3))^6$
\item $f(3 + 4) - (f(3) + f(4))$
\end{enumerate}
\item Let $f:\mathbb{R}\to\mathbb{R}$ be the function given by $f(x) = x^2 - 6$ and let $g:\mathbb{R}\to\mathbb{R}$ be the function given by $g(x) = 2x$. Evaluate each of the following:
\begin{enumerate}
\item $(f + g)(2)$
\item $(f\cdot g)(2)$
\item $(f\circ g)(2)$
\item $(g\circ f)(2)$
\item $f(g(f(2)))$
\item $(f^2\circ g)(2)$
\end{enumerate}
\item Determine the natural domains of each of the following formulas.
\begin{enumerate}
\item $x^3 - x$
\item $\tfrac{x + 2}{(x - 2)(x - 3)}$
\item $\sqrt{2x - 8}$
\item $\sqrt[4]{x^2 - 5x - 14}$
\end{enumerate}
\item For each of the following functions, determine the range. If the domain is unspecified and only a formula is given, assume that the corresponding function has the natural domain.
\begin{enumerate}
\item $x^2$
\item $x^3$
\item $\frac{1}{x - 1} + 4$
\item $\sqrt{x + 3} - 2$
\item $x^2$ with domain $(-2,3]$
\item $\frac{2x}{x^2 + 1}$
\end{enumerate}
\item For each of the following functions, determine whether the function is invertible. If so, find the inverse function (including domain specification as needed). If not, find two input values which produce the same output.
\begin{enumerate}
\item $x^2$
\item $x^2$ with domain $(-2,3]$
\item $x^2 - 6x + 8$ with domain $(3, +\infty)$
\item $\frac{1}{x - 1} + 4$
\item $\sqrt{x + 3} - 2$
\item $\frac{2x}{x^2 + 1}$ with domain $[-1,1]$
\end{enumerate}
\item Let $f:\mathbb{R}\to\mathbb{R}$ be a function. Describe a sequence of transformations that would transform the graph of $f$ into the graph of the given equation.
\begin{enumerate}
\item $y = f(x) - 3$
\item $y = -2f(x)$
\item $y = f(x - 5)$
\item $y = f(x/4)$
\item $y = 3f(2x - 1) + 4$
\item $(y + 1)/2 = f(-x + 6)$
\end{enumerate}
\item Let $f:\mathbb{R}\to\mathbb{R}$ be a function satisfying $f(4) = 7$ and let $g:\mathbb{R}\to\mathbb{R}$ be given by the formula $g(x) = 3f(x^2) - 4$.
\begin{enumerate}
\item Find two points on the graph of $g$.
\item Does $g$ have an inverse?
\end{enumerate}
\end{enumerate}


\subsection{Challenge problems}

\begin{enumerate}[resume]
\item Let $f:\{1,2,3,4,5,6,7,8\}\to\{1,2,3,4,5,6,7,8\}$ be given by
\begin{align*}
f(1) &= 6, & f(2) &= 4, & f(3) &= 2, & f(4) &= 3, \\
f(5) &= 7, & f(6) &= 8, & f(7) &= 1, & f(8) &= 5.
\end{align*}
Find the smallest positive integer $n$ such that $f^n(k) = k$ for all valid inputs $k$.
\item 
\item (Even more challenging than usual) Describe a function $f:\mathbb{R}\to\mathbb{R}$ with the property that for any real numbers $a$, $b$, and $c$ with $a < b$, there is a real number $x$ between $a$ and $b$ such that $f(x) = c$. \emph{Hint: Digits and \href{https://en.wikipedia.org/wiki/Positional_notation}{base number arithmetic}.}
\end{enumerate}


\newpage
\subsection{Answers}
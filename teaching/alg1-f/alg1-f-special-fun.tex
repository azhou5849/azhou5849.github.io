\section{Special Functions}

\subsection{Review problems}

\begin{enumerate}
\item Let $f(x) = 4x^3 - 6x^4 + 2x - 1$ and $g(x) = 3x^4 + x^2 + 9$ and $h(x) = \sqrt{x^2 + 1}$.
\begin{enumerate}
\item Which of $f$, $g$, and $h$ are polynomial functions?
\item Evaluate $f(0)$ and $g(1)$.
\item Simplify $f(x) + g(x)$ and $f(x)\cdot g(x)$.
\item Compute $\deg f(x)$ and $\deg g(x)$.
\item Is there a constant $a$ for which $f(x) + a\cdot g(x)$ has degree less than $4$? If so, what is the value of $a$ and the degree of $f(x) + a\cdot g(x)$ for that value? If not, explain why not.
\end{enumerate}
\item Let $p(x) = x^4 + 9x^3 + 28x^2 + 39x + 21$. This polynomial has no integer roots, but there are positive integers $a,b,c,d$ such that 
\begin{equation*}
p(x) = (x^2 + ax + b)(x^2 + cx + d).
\end{equation*}
Find all \underline{real} roots of $p(x)$.
\item Solve each of the following equations for $x$.
\begin{enumerate}
\item $\sqrt[6]{625} = 5^x$
\item $9^{1 + x} = 27^{1 + 1/x}$
\item $4^x - 2^x = 56$
\end{enumerate}
\item (Calculator allowed) Janelle puts \$10,000 into an account that earns $4\%$ interest compounded annually. How much money will there be in the account after $18$ years?
\item Compute each of the following logarithms without a calculator (some are undefined).
\begin{enumerate}
\item $\log_7(49)$
\item $\log_2(64)$
\item $\log_{27}(9)$
\item $\log_5(1)$
\item $\log_4(0)$
\item $\log_6(-6)$
\item $\log_8(1/4)$
\item $\log_{12}\left(2\sqrt[3]{18}\right)$
\end{enumerate}
\item Find the domain and range of the function $f(x) = \log_{10}\left(\sqrt{100 - x^2}\right)$.
\item Find the values of $a$, $b$, and $c$ which make the following function continuous:
\begin{equation*}
f(x) = \begin{cases} x^2 - 3 & x\leq -1; \\ ax + b & -1 < x\leq 2; \\ \frac{2x^2 - 3x - 9}{x^2 - 4x + 3} & x > 2\text{ and }x\neq 3; \\ c & x = 3. \end{cases}
\end{equation*}
\end{enumerate}


\subsection{Challenge problems}

\begin{enumerate}[resume]
\item Let $p(x) = (1 + x)^{20}$.
\begin{enumerate}
\item Find the sum of the coefficients of $p(x)$.
\item Find the sum of the coefficients of $p(x - 1)$.
\item (Very challenging) Find the sum of the coefficients of the terms of $p(x)$ whose degrees are multiples of $4$.
\end{enumerate}
\item (Calculator allowed)
\begin{enumerate}
\item Let $E(i,m)$ denote the \emph{effective annual interest rate} for a nominal annual interest rate $i$ compounded $m$ times per year. That is, for any whole number of years, compounding annually at a rate of $E(i,m)$ results in the same balance as compounding $m$ times per year with a nominal annual interest rate $i$.\par Write down a formula for $E(i,m)$ in terms of $i$ and $m$, then compute $E(4\%, 12)$.
\item Let $F(i,m)$ be the solution to the equation $E(F(i,m),m) = i$. Compute $F(4\%, 12)$.
\item If we hold $i$ fixed, then as $m$ gets larger and larger, $F(i,m)$ approaches a value called the \emph{force of interest}. For the interest rate $i = 4\%$, compute the force of interest to three decimal places.
\end{enumerate}
\item Compute $(\log_4 5)(\log_5 6)(\log_6 7)(\log_7 8)$ without a calculator.
\end{enumerate}


\newpage
\subsection{Answers}

\begin{enumerate}
\item \begin{enumerate}
\item $f$ and $g$ are polynomials, $h$ is not 
\item $f(0) = -1$ and $g(1) = 13$
\item \begin{align*}
f(x) + g(x) &= -3x^4 + 4x^3 + x^2 + 2x + 8 \\
f(x)\cdot g(x) &= -18x^8 + 12x^7 - 6x^6 + 10x^5 - 57x^4 + 38x^3 - x^2 + 18x - 9
\end{align*}
\item $\deg f(x) = \deg g(x) = 4$
\item When $a = 2$,
\begin{equation*}
f(x) + a\cdot g(x) = 4x^3 + 2x^2 + 2x + 17
\end{equation*}
so $\deg (f(x) + a\cdot g(x)) = 3$.
\end{enumerate}
\item We can suppose that $b\leq d$, as otherwise we simply swap the two factors of $p(x)$. Expanding,
\begin{align*}
a + c &= 9, \\
b + ac + d &= 28, \\
ad + bc &= 39, \\
bd &= 21.
\end{align*}
Since $a,b,c,d$ are positive integers, either $b = 1$ and $d = 21$ or $b = 3$ and $d = 7$. In the first case, we have $a + c = 9$ and $21a + b = 39$, but this is not satisfied by positive integers. Thus we are in the second case, so $a + c = 9$ and $7a + 3b = 39$. This system has solution $a = 3$ and $c = 6$, and we can check the other equations (or by expanding from scratch) to see that
\begin{equation*}
p(x) = (x^2 + 3x + 3)(x^2 + 6x + 7).
\end{equation*}
If $r$ is a root of $p(x)$, then $p(r) = 0$, so either $r^2 + 3r + 3 = 0$ or $r^2 + 6r + 7 = 0$. By the quadratic formula, the first case gives us
\begin{equation*}
r = \frac{-3\pm\sqrt{-3}}{2},
\end{equation*}
which are non-real roots, while the second case gives us
\begin{equation*}
r = \frac{-6\pm\sqrt{8}}{2} = \boxed{-3\pm\sqrt{2}}.
\end{equation*}
\item 
\end{enumerate}